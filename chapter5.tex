\documentclass[main.tex]{subfiles}
\begin{document}

\chapter{A Bond Forged of Fire}

\par \nar After the dissolution of their natal star cluster, \rmmaia~ and \rmmerope~ wandered through the Galactic Disk.  Millenia passed without much in the way of interruption; \rmmaia~ had only \rmmerope~ to keep her company, and vice versa.  Now and again, the sisters encountered a passing star off in the distance, too far away to exchange a conversation.  Regardless, they yelled relentlessly, eager to meet a new face.  After a few million years of this, the pair had grown bored, but inseparable.

\section{An Old Star...}

\par \nar \rmmerope~'s age was now showing.  The most massive of her siblings, she was destined to burn through her nuclear fuel the fastest.\footnote{To remind the reader, the most massive stars have the shortest lifetimes since, in spite of having more nuclear fuel, they are able to burn through their entire energy reservoirs faster.}  She was now reaching the end of her short life.  As such, she had begun to expand, slowly entering the red giant phase of stellar evolution.\footnote{Recall that during the red giant branch phase of stellar evolution, stars can expand up to several hundred times their size on the main-sequence.  When the Sun eventually reaches the red giant branch phase, several billion years from now, it is likely to expand out beyond the orbit of Earth, engulfing the planet.}  

\par \Maia Uh... I don't mean to alarm you, \rmmerope, but I've noticed you've been expanding lately.  I mean, just a little bit.  No big deal, really.  But, uh, where do you suppose this is heading?

\par \Merope Uh... Well, I hadn't noticed, to be honest.  But now that you mention it, I do feel a little bloated.  

\par \Maia Well, you look \textit{great}, that's for sure.

\par \Merope Thank you, sister.  But I fear this is going to get worse before it gets better.  Much much worse, in fact.  I seem to be expanding further...

\par \Maia Oh dear.  Well, don't panic.  We'll figure this out.  How do you feel?

\par \Merope I feel... Cold.  Do I look cold to you?

\par \nar \rmmerope~ was shivering noticeably.

\par \Maia Eeesh... Yeah, kind of.  You've been getting steadily redder in color.  It's not drastic, but it is noticeable.\footnote{As stars expand on the giant branch, their effective or surface temperatures decreases.  The reason for this stems from a formal relationship between the radius, luminosity and surface temperature of a star, called the Stefan-Boltzmann law.  The general trend of larger radii corresponding to cooler surface temperatures, at a given mass and evolutionary stage, is consistent with the basic idea that larger radii correspond to surfaces with a larger displacement from the core, which is where energy is generated in stars.  Hence, the total energy diffusing through the stellar surface per unit time (i.e., the total luminosity) is smaller, and so is the effective temperature. CHECK}

\par \Merope Um... Okay, maybe if I hold my breath \textit{that} will stop the expansion.  Worth a try, I suppose.  Here I go...

\par \nar \rmmerope~ drew in a deep breath, and held it.  The minutes passed.  Her face blue, \rmmerope~ eventually exhaled in defeat.  

\par \Merope Oh no.  That was uncomfortable.  And I am still expanding.

\par \Maia Don't panic.  We'll sort this out.

\par \Merope You already said that!

\par \Maia Well, frankly, I can't emphasize it enough.  Just... Don't panic.

\par \Merope You aren't making me feel any better over here.

\par \Maia I'm sorry.  I'm doing my best... I know I said not to panic, but I'm losing it over here myself...  

\par \nar \rmmaia~ was noticeably agitated, pacing back and forth.\footnote{This pacing occurs in the radial direction, and can ultimately be characterized by the orbital eccentricity:  more eccentric orbits have larger (fractional) radial amplitudes, at a given orbital separation.  In other words, more eccentric binaries are more agitated.  PICK OF ORBITAL DYNAMICS AND MORE THOROUGH DEFINITION OF ORBITAL ELEMENTS AND ECCENTRICITY?}

\par \Maia Yep, I'm dropping the ball.  Big time.

\par \Merope  Jeez, \rmmaia, get ahold of yourself.  I'm the one who seems to be expanding to certain doom, and yet you are the one hyperventilating.

\par \nar \rmmaia~ was rapidly fluctuating in brightness, caused by the hyperventilation.

\par \nar Flabbergasted, \rmmaia~ struggled to regain her composure.

\par \Maia It's fine.  I'm fine.  How are you doing?

\par \Merope Well, quite honestly, I continue to expand.  How do I stop?

\par \Maia Have you tried thinking really hard about \textit{contracting}?

\par \Merope Really?  I'm about to die, and your big idea is to think good thoughts?  

\par \Maia Well, it can't hurt...

\par \Merope You aren't helping!  

\par \nar By this time, \rmmerope~ had already expanded by over an order of magnitude, and her radius was still growing rapidly.  

\par \Maia I'll go get help!

\par \Merope And how are you going to do that?  We're gravitationally bound.

\par \Maia Oh right.  Gravity.  Okay, new plan:  I'll think good thoughts?

\par \nar \rmmerope~ sighs in defeat.

\par \Merope I'm going to die for sure.

\par \Maia Don't say that, \rmmerope~  You're not going to die.

\par \nar Eventually, \rmmerope~ stopped expanding and stabilized.

\par \Maia Whoa!  I think you've stopped expanding! 

\par \Merope Oh thank the Heavens!  I do indeed seem to have stopped.  Phew!  What a relief!

\par \Maia Well, look at you!  You've definitely stopped expanding.  You're much much redder and, well, larger than you used to be...but none of that is changing anymore.  I think you're going to be okay!

\par \nar \rmmerope~ breathed a long sigh of relief.  The life of a massive star was anything but easy, but \rmmerope~ remained optimistic that the aging process would grow easier as time went on.

\section{Blown Out of Proportion}

\par \nar For some time, it seemed \rmmaia~ had proved right.  But eventually \rmmerope~ began expanding again.  Over the next few million years, she eventually grew accustomed to it.  Just another day in the life of a massive star.

\par \nar But then, one day, suddenly and without warning, \rmmerope~ exploded.  A supernova!  Plasma and gore flew passed \rmmaia~, spraying her face with the guts of \rmmerope~ and knocking her backward.  The blast was asymmetric, escaping mostly from a preferred side which happened to oppose \rmmerope~'s orbital motion almost perfectly.  Due to angular momentum conservation, this meant a significant and almost instantaneous reduction in \rmmerope~'s orbital velocity and a loss of orbital angular momentum.\footnote{EXPLAIN THIS USING CONSERVATION OF MOMENTUM, AS WELL AS CONSERVATION OF BOTH ENERGY AND ANGULAR MOMENTUM.}  \rmmerope~ drifted inward, closer to \rmmaia~ and on a much more eccentric orbit than before.\footnote{EXPLAIN THE CONCEPT OF ORBITAL PARAMETERS, AND PROVIDE THE DEFINITION OF ECCENTRICITY BY DISCUSSING ELLIPSES.}  

\par \Maia \rmmerope! Are you alright?!  

\par \Merope Uh... I think so.  Give me a minute here.  I feel much lighter.

\par \Maia Well you did just eject your outer envelope and most of your mass in a rather dramatic supernova explosion.  So it makes sense that you are much less massive now.  You look so small and compact!  

\par \Merope Thank you!  I feel pretty svelte over here.  Looking good, feeling good.  Who would have thought exploding could be such a positive experience?

\par \nar A worried expression appeared suddenly on \rmmerope~'s face.

\par \Maia What's wrong?

\par \Merope Does your belly always do that?

\par \nar \rmmaia~ looked down to discover her midsection had grown distended.  A bulge pointed directly toward \rmmerope~.\footnote{Gravity exerts a differential force across the extent of finite-sized objects.  In practice, for example, consider the Earth-Moon system.  The gravitational force exerted by the Moon on the Earth is strongest on the side facing the Moon, and weakest on the opposing side, since the force of gravity falls off as the inverse of distance squared.  This causes the side facing the Moon to become extended, or to "bulge out" toward the Moon.  Called a "tidal bulge", this is the mechanism responsible for tides in the Earth's oceans. PROVIDE AN ILLUSTRATION FOR CLARITY.}  As \rmmerope~ re-approached the point along the orbit corresponding to closest approach with \rmmerope,\footnote{Orbits can be either circular or elliptic.  In the former case, both objects maintain a constant distance from their mutual center of mass.  In the latter, or "eccentric", case, the distance of each object changes with respect to their mutual center of mass.  The point along the orbit corresponding to the distance of closest approach between the objects is called "pericenter" or "periastron", whereas the opposing side (corresponding to the point of furthest approach) is called "apocenter" or "apoastron".} the bulge grew and a thin stream of material began to flow from \rmmaia~'s surface toward \rmmerope~.\footnote{Consider a line connecting the centers of mass of two objects in orbit about each other.  Somewhere along this line, the force of gravity exerted by each object is exactly balanced (the location of this point depends only on the masses of the two objects).  If the radius of one star expands beyond this point, then the matter on the outer surface of the star is more strongly attracted gravitationally by its binary companion.  In such a scenario, the mass flows from the surface of the large star onto the surface of the other, flowing through the point where their mutual gravitational force is precisely balanced.  Mass transfer ensues.  This critical surface is called the "Roche lobe".}

\par \Maia Holy cow!  Now I'm losing mass! Okay, don't panic.

\par \Merope I swear, if you say that one more time I'm going to scream.

\end{document}
  

