\documentclass[main.tex]{subfiles}
\begin{document}

\section{The Triples...}

%THIS CHAPTER DESCRIBES THE OUTER TRIPLE EXCITING LK OSCILLATIONS IN THE INNTER TWO BINARY COMPONENTS, EVENTUALLY CAUSING THEM TO MERGE IN A HORRIFIC AND GORY FINAL FATE.  THE NEW EMERGING SIBLING, WHILE MUCH MORE MASSIVE, IS SWEET AND KIND, AND TWEEDLEDOH IS VERY HAPPY TO MEET HIS NEW SISTER TWEEDLEDAH.
 
%Alone at last ...well, depending on how you looked at it.  
Now gravitationally unbound from their siblings, $\Taygete$, $\Alcyone$ and $\Celaeno$ drifted off in to the vastness of empty spce.  Three siblings alone at last.  Well, two twins and a target; the inner twins continuted to gossip rudely about their outer, omni-present sister.

\Taygete She's a little bulbous don't you think?

\Alcyone  Ha ha.  Yeah, you nailed it!

\Celaeno  I can hear you two, just so you know.  And I am \textit{not} bulbous.  A little round perhaps, but never bulbous.  And I shine brither than each of you.  So take that!

\Alcyone You may shine brighter than each of us individually, but together \textit{we} outshine \textit{you}.  

\Taygete Yeah!  Take \textit{that}!

$\Celaeno$ lets out a long, exasperrated sigh.

\Celaeno You guys, I really don't want to fight with you.  First, we're sisters from the same Mother.  Second, we're pretty much stuck together in this configuration for the next several hundred million years.  So we had might as well make the most of it.

\Taygete Fat chance, you bulbous sphere.

\Alcyone Ah ha, another good one, sister!

\Celaeno Well, fine.  I guess I'll take a nap then.  Better than listening to the two of you.

\subsection{The Demise of $\Taygete$ and $\Alcyone$}

After some time, $\Celaeno$ awoke from a peaceful slumber.  

\Celaeno Yaaaaawn... What a lovely day, wouldn't you say, $\Taygete$ and $\Alcyone$?

$\Celaeno$ turned her gaze toward her sisters, but was horrified by what she saw.  Her sisters' relative configuration had changed over the course of their rest.  First, it seemed to $\Celaeno$ That the relative angle of inclination between the inner orbital plane formed by his sisters and her own outer orbital plane had changed.  They were now much more co-planar than before.  In fact, currently, $\Celaeno$ could only see $\Alcyone$ directly; presumably, $\Taygete$ was behind $\Alcyone$, her light being eclipsed by the foreground presence of her twin sister, along the line of sight to $\Celaeno$.

More importantly, $\Taygete$ and $\Alcyone$ now shared a much more eccentric orbit than before.  First they swung out to larger distances relative to their common center of mass, where they sit patiently due to their much slower orbital velocities here (i.e., apoastron).  But, they eventually swung to much closer approaches than before, their surfaces almost touching at the point of closest approach (i.e., periastron).  Each time they complete one orbital revolution, it seems they get a little bit closer to colliding directly with each.  Their orbit was definitely getting more compact, and fast.

After a particularly close periastron passage, $\Taygete$ began to notice the changes.

\Taygete  Whoa!  What the hell is going on?!  I almost just colliding with $\Alcyone$ there!

\Alcyone Ah, yeah no kidding!  What the hell \textit{is} going on?!

\Celaeno  Well, don't panic just yet.  We need to figure this out, since I can't very well run off to get and bring back help.

\Alcyone Useless to the bitter end...

\Celaeno Go hump yourself.  But while you are doing that, I am going to try to figure out a way to save your miserable ass.  I suspect that the shift toward more co-planar orbits is directly related to the increase in your orbital eccentricity.  If you think about, these two things in combination should conserve total angular momentum, and might arise if the outer or inner orbit spends more or less time above or below the orbital plane of, respectively, the inner or outer orbit.

$\Celaeno$ found herself lost in thought.  Several minutes passed.  Finally, an epiphany strikes her.

\Celaeno  Wait, I've got it!  I know what is going on!

\Taygete  Super!  Well, please do enlighten us.

\Celaeno  Okay, so our unique three body configuration consists of two orbital planes.  You two orbit in one of those planes, and I orbit in another about our mutual center of mass.

\Alcyone  You lost me.

\Taygete What's the point exactly?

\Celaeno The point is that your mutual orbital motion is such that its plane spends a net excess amount of time above or below my orbital plane.  It depends on our exact configuration, but that's the basic idea.  This is critical though, since it means that a net torque is being applied between our orbits.

\Alcyone And why should we care about that?

\Celaeno Well, unless I miss my guess, this will cause your orbital plane to be torqued toward mine, so that we are in the end orbiting roughly co-planar to each other.  Unfortunately, in order to conserve total angular momentum, this also causes the eccentricity of your orbital motion to increase.  

\Taygete  Huh?  Say that again?

\Celaeno Okay, basically, I suspect that the shift toward more co-planar orbits is directly related to the increase in your orbital eccentricity, which we are clearly seeing.  If you think about, these two things in combination should conserve the total angular momentum of the three-body system, and might arise naturally if the outer or inner orbit spends more or less time above or below the orbital planeof, respectively, the inner or outer orbit.  Apart from that, well, there's not much to say, really.  Your mutual periastron approach is already comparable to the sum of your radii.  And it is still decreasing due to the aforementioned effect.  And we haven't even considered tidal dissipation acting at periastron due to your \textit{extensive} radii being finite, which will only accelerate the rate of dissipation.  In short, you're about to collide with each other.  You'll probably merge.

\Taygete  Wait, WHAT?!

\Alcyone  WHAT?!

Just then, $\Taygete$ smashes in to $\Alcyone$ as they both re-approach their mutual periastron.  The collision is violent; the relative velocity is of order the sum of their local orbital speed, which is several hundred kilometers per second.  Needless to say, the sisters never stood a chance.  Their innards and organs (i.e., massive globs of hot gas) are flung violently away from the merging pair.  It is a grizzly scene.

\Celaeno Oh, Dear Lord!  I think I'm going to throw up.  SO MUCH PLASMA!  Gross, disgusting, and emotionally it's a lot to handle. ...Yep, here comes the vomit...

$\Celaeno$ suddenly vomits, spraying a particularly potent coronal mass ejection (or CME, for short) in to her immediate vicinty.  The vomited CME extends in an arc spanning about 120 degrees.

\Celaeno Oooooooh my God, they've got to be dead after that.  So gross!  

$\Celaeno$ nearly vomits a second time, but manages to stifle the urge.

\Celaeno ...Uh...I guess I should check to see if everybody is okay? ...CAN YOU HEAR ME?!  ARE YOU STILL THERE?! 

Gravity was in the process of taking the remains of what had once been $\Taygete$ and $\Alcyone$, and re-shaping them into a brand new star.  Formed from two stellar corpses, this new star was quickly turning out to be nearly twice as massive as each of her individual progenitors.  Suddenly, the product of this coalescence awakes.

%was spinning rapidly, but not for long; intense magnetic fields had been amplified during the collision and were rapidly spinning the new star to lower rotational frequencies.  A few millions years from now, all signs of the simultaneous deaths of two brothers, needed to give life to a new sister, will have vanished. 

\Lacedaemon Uh...Hello, there!  My name is...uh...$\Lacedaemon$, I do believe.  I appear to be new to the scene...of which I know absolutely nothing about.  Where are we exactly?  Uh...\texit{What} are we?

\Celaeno Hello!  You are my new brother.  We are stars; spheres of gas and dust contracting into this configuration by gravity, which in turn rose the temperature in our cores \texit{a lot}.  We are slowly undergoing nuclear fusion in our bellies, converting hydrogen into helium and emitting energy in the form of light or photons in the process.  The outward momentum supplied by the photons provides the outward pressure we need to balance the inward force from gravity.  Mother called it ``hydrostatic equilibrium''.

\Lacedaemon Wow, that was a lot of very technical information to start with.  Still, I appreciate it very much!  In fact, I'm quite impressed.  Now that you mention it, I am starting to feel very balanced overall.  

\Celaeno I am glad to hear it!

\Lacedaemon ...Well, except for this excess rotation I seem to be holding on to. And it goes right to the gut, let me tell you.  Wow, this bulge is really extending outward at my equator.  SUPER!  
%You look very svelt for a star, I must say.  I seem to be holding on to all this extra crud around my mid-center.  Rotational weight, that's the culprit.  Who would have guessed that angular momentum conservation would be this much work?

\Celaeno Don't worry, I've seen it lots of times before.  New stars spin down as they grow out of infancy.  So, the ``gut'' will go away.  You'll settle in to a more sphere-like shape in no time.  I think the whole thing has something to do with ``magnetic fields'', or so I have heard.  As near as I can tell, this is some magical force that slows your spin rate down as you mature into a beautiful new star.

\Lacedaemon Hey, alright!  Good news.  I am sold.  I mean, who doesn't appreciate spherical symmetry?  Weirdos, that's who.  I'm in for sure.  Uh...So now what?

$\Celaeno$ liked her new brother, convinced she would appreciate having him around.

\Celaeno We focus on the journey ahead, of course.  Who knows where our fates will take us.  But now that we are free of our siblings, I suppose almost anything \textit{could} happen.  Here is hoping for mostly good things.

\Lacedaemon  Wow, this is exciting!  A new chapter in our lives.  What adventures will befall us?  What obstacles will we overcome?  I can't wait to find out!

%A worried expression became evident on $\Celaeno's$ face.  
%
%\Lacedaemon  You look like you just saw something terrifying.  What's up?
%
%\Celaeno  Well, if you take our current trajectory through the Galaxy and propagate it forward through time, it looks to me like we are ultimately heading for the Galactic Center.
%
%\Lacedaemon  Great!  Wait... What is a ``Galactic Center''?
%
%\Celaeno  It is the very center of our Galaxy.  Here, a dense nuclear star cluster lives, jam packed with stars spanning all sort of masses and ages.  We are in the field of our Galaxy currently, where the mean stellar density is about 1 star per cubic parsec.  For example the distance separating the Sun and her next nearest neighbor, Proxima Centauri, is about a parsec.  But if you take the Sun and Proxima Centauri as they are observed, and drop them in to the very central regions of the nuclear cluster at the heart of our Galaxy, then more than 100 stars will fall between the pair.  The densities are that much higher.  
%
%\Lacedaemon  Wow.  That sounds crowded.  I bet you can hear \textit{everything} \textit{everyone} is doing, at any given moment.  *shudders*  
%
%\Celaeno But the really terrifying thing, if you ask me, is the super-massive black hole lurking at the heart of the nuclear cluster.    
%
%\Lacedaemon Super-massive black hole, you say?  What the hell is that?
%
%\Celaeno It is a dense dark object that does not emit light, typically way smaller than Jupter in size.  But it's total mass can range anywhere from 10$^6$ to 10$^{10}$ times the mass of the Sun.  In our own Milky Way, the super-massive black hole is known to be about 4 $\times$ 10$^6$ times the mass of the Sun.
%
%\Lacedaemon Holy cow!  That sounds....potentially violent.
%
%\Celaeno Uh... Yeah, it is.  Technically, if you wander too close to a super-massive black hole, it can eat you whole.  And, once inside its belly, you can never escape.
%
%\Lacedaemon  Oh, wonderful!  That doesn't sound even remotely terrifying...
%
%\Celaeno Um... And even if you don't get eaten, if you do wander close to a super-massive black hole, you are almost certainly going to get accelerated to extremely high velocities.  Like, we are talking thousands of kilometers per second.  
%
%\Lacedaemon  WHAT?!  Is that even possible?
%
%\Celaeno  Yep, such fast stars have definitely been seen whipping by in our very own Galaxy.  
%
%\Lacedaemon Well, this is all very terrifying.  How can we possibly prepare for a journey through the central nuclear star cluster of the Milky Way?
%
%\Celaeno Well, I'm going to take another nap, personally.  I figure we'll need to be fresh and alert, if nothing else.
%  
%
%\Lacedaemon Oh my gawd, you don't have a plan!  We are so going to die!
%
%$\Celaeno$ had already fallen asleep, and was now snoring loudly.

\end{document}

  

