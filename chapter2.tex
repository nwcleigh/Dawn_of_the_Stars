\documentclass[main.tex]{subfiles}
%\usepackage[T1]{fontenc}
%\usepackage{ae,aecompl}


%%%%% AUTHORS - PLACE YOUR OWN PACKAGES HERE %%%%%

% Only include extra packages if you really need them. Common packages are:
%\usepackage{graphicx}	% Including figure files
%\usepackage{amsmath}	% Advanced maths commands
%\usepackage{amssymb}	% Extra maths symbols


%\newcommand\levelone[1]{{\color{red}\bf}}
%\newcommand\leveltwo[1]{{\color{blue}\bf}}
%\newcommand\levelthree[1]{{\color{green}\bf}}
\begin{document}


%INSERT NEW CHAPTER 2 HERE.

\chapter{A Gust of Wind...}

\par \nar \rmsterope~ awoke when a gust of wind brushed past her face.  The gas was dense enough to temporarily obscure her vision.  She could feel the wind against her stellar surface as it rushed past, gently caressing her fiery skin.  All in all, the wind carried enough momentum to startle her out of slumber,


\begin{tcolorbox}[sharp corners, colback=blue!30, colframe=blue!80!blue, title=Linear Momentum]
\par \textcolor{blue} {The momentum $\vec{p}$ an object possesses is defined as the product of the object's mass and its velocity, or:
\begin{equation}
%\begin{center}
\vec{p} = m\vec{v},
%\end{center}
\end{equation}
hwere $m$ is the mass of the object and $\vec{p}$ is its three-dimensional linear momentum vector.  Momentum is in many ways complementary to the term inertia; momentum quantifies how difficult it is to alter an object's trajectory.  The more momentum an object possesses, the larger the total applied force must be (over a given interval of time) in order to change the object's trajectory.  More specifically, the total change in momentum can be calculated using the product of the applied force and the total time spent applying that force to the object.  The corresponding change in momentum is larger if either the magnitude of the applied force is larger, or the total time spent applying that force is longer.  Linear momentum is always a conserved quantity, even if inelastic collisions are taken into account (in which total kinetic energy is not conserved, since some energy is absorbed internally by the particles during the collisions)}.  
\end{tcolorbox}

%\leveltwo{
\begin{tcolorbox}[sharp corners, colback=blue!30, colframe=blue!80!blue, title=Angular Momentum]
\par \textcolor{blue} {Angular momentum is the rotational analog of linear momentum.  The angular momentum $\vec{L}$ an object possesses is defined as the product of the object's mass and its velocity, or:
\begin{equation}
%\begin{center}
\vec{L} = m\vec{r} \times \vec{v} = rmv_{\rm \perp} = rmvsin(\theta),
%\end{center}
\end{equation}
where $r$ is the distance of the object from the system center or mass, $m$ is the mass of the object and $v_{\rm \perp}$ is the velocity component orthogonal to $\vec{r}$, and $\theta$ is the angle between these two vectors. 
%Momentum is in many ways complementary to the term inertia; momentum quantifies how difficult it is to alter an object's trajectory.  The more momentum an object possesses, the larger the total applied force must be (over a given interval of time) in order to change the object's trajectory.  More specifically, the total change in momentum can be calculated using the product of the applied force and the total time spent applying that force to the object.  The corresponding change in momentum is larger if either the magnitude of the applied force is larger, or the total time spent applying that force is longer.  
Briefly, we describe qualitatively the conservation of angular momentum, deferring a more detailed derivation to later.  The total scalar angular momentum is always a conserved quantity, independent of the details of its redistribution (e.g., orbital angular momentum in binaries can be converted to spin angular momentum in the stars; if the orbit loses angular momentum, then the stars must gain enough spin angular momentum to compensate).  Hence, $\vec{}L^2} is a conserved quantity.  If we consider point particles all orbiting in a plane perpendicular to the $z$-axis, then it can be shown that $\vec{L_{\rm z}$ is also a conserved quantity.}.  
\end{tcolorbox}

\par \nar \rmsterope~ coughed, clearing the gas from her face.  Her surroundings now revealed, \rmsterope~ looked around, confirming her suspicions; the gas had become substantially less dense since she had fallen asleep, now disconcertingly sparse. 

%\levelone{
\begin{tcolorbox}[sharp corners, colback=red!30, colframe=red!80!blue, title=Mass Density]
\par \textcolor{red} {\nar Density is defined as the total amount of mass per unit volume; lower densities imply less mass occupies a given unit of 3-D volume.  Consider a spherical volume of radius $r$ with constant mass density $\rho$ and containing a total mass $M$.  In this simple case, the mass density is:
\begin{equation}
\rho = \frac{4M}{3{\pi}r^3}
\end{equation}
}  
\end{tcolorbox}
%Something was causing the remaining gas to leave, and fast.

\par \Sterope My sisters, you must wake up!  The remnants of our Mother are leaving us.

\par \nar The other six siblings awoke to the scene described by \rmsterope.  

\par \Electra Whoa!  What's going on?  We're all drifting apart.  And where did Mother go?

\par \Maia It's okay, young ones.  The last vestiges of Mother have now left us.  Upon giving birth to us, she activated our metabolisms.  We've been spewing light out ever since, in the form of photons.  Each photon carries momentum, and transfers some of it to any gas molecule or atom upon collision.  Our light has been banging in to the gas and dust of our Mother for quite some time now, pushing her outward and away.

%\levelone{
\begin{tcolorbox}[sharp corners, colback=red!30, colframe=red!80!blue, title=Momentum Carried by Light]
\par \textcolor{red} {``Radiation pressure'', they call it.}
\end{tcolorbox}

\par \Electra Okay, but then why are \textit{we} drifting apart from \textit{each other}?

\par \Maia Mother was made up of gas and dust, which came along with significant \textit{mass}.  And with mass comes gravity.  Now that her mass is gone, it can no longer contribute to the inward pull of gravity.  As this happens, we move further and further away from our common center of mass. This is because, if we add up all the same in stars, it is not enough from gravity to keep us in close proximity.  Only with the gas mass supplied by our Mother was this possible, but it is now gone.

%\levelone{
\begin{tcolorbox}[sharp corners, colback=red!30, colframe=red!80!blue, title=Gravitational Potential Energy]
\par \textcolor{red} {Gravity provides an attractive force that keeps mass effectively "glued" together.  The strength of this glue is often quantified by something called the gravitational potential energy of the system, often denoted $V$.  Consider a distribution of $N$ identical particles, each with a mass $m_{\rm i}$.  The total system mass is $M = Nm_{\rm i}$.  Let us also assume that the mass distribution is spherically symmetric, for simplicity.  The mass distribution defines the gravitational field, which is a function only of the distance $r$ from the system center of mass due to our assumption of spherical symmetry.  At every point in the gravitational field, the gravitational force exerted on particle $i$ is:
\begin{equation}
\vec{F}_{\rm i} = m_{\rm i}\frac{d^2{r_{\rm i}}_{\rm i}}{dt^2} = -Gm_{\rm i}\sum_{\rm j=1,j \ne i}^{\rm N}m_{\rm j}\frac{\vec{r_{\rm i}} - \vec{r}_{\rm j}}{|\vec{r_{\rm i}} - \vec{r_{\rm j}}|^3},
\label{eqn:force}
\end{equation}
where $t$ denotes time.  Note that the gravitational force $\vec{F}_{\rm i}$ is a vector quantity, and so a direction must be specified for the applied acceleration.  The gravitational force is equal to the gradient of the gravitational potential at the location of the particle, or:
\begin{equation}
\nabla_{\rm i}V = \vec{F}_{\rm i}.
\end{equation}
We can compute the total gravitational potential energy of the system by summing over all particles:
\begin{equation}
V = -\frac{1}{2}\sum_{\rm j=1}^{\rm N}\sum_{\rm k=1, k \ne j}^{\rm N}\frac{m_{\rm j}m_{\rm k}}{|\vec{r_{\rm j}} - \vec{r_{\rm k}}|}.
\end{equation}
Potential energy is by definition a negative quantity.}
\end{tcolorbox}


\par \Maia In other words, without our dispersed Mother, between the seven of us we no longer have enough mass in our mutually occupied volume to keep us gravitationally bound.  We are now free to drift apart.  And drift apart we are destined to do. 

%leveltwo

\par \Taygete Yeah, yeah, yeah.  But what does all that even \textit{mean}?

\par \Sterope I think it means that this is goodbye...  With our Mother's mass now lost, we are no longer gravitationally bound.  Relative to each other, we are energetically \textit{unbound}.

%\leveltwo{
\begin{tcolorbox}[sharp corners, colback=blue!30, colframe=blue!80!blue, title=Kinetic Energy]
\par \textcolor{blue} {In addition to the gravitational potential energy, particles have individual kinetic energies which can be summed over to compute a total kinetic energy for the system, usually denoted as $T$.  In our hypothetic spherically symmetric distribution of particles, gravity is causing the particles to move around within the gravitational potential with finite velocities.  These particles are orbiting within the gravitational field of the system, since gravity imparts unto every particle an acceleration at every instant in time.  The kinetic energy of a given particle is proportional to the square of its velocity $v_{\rm i}$ relative to the system center of mass.  By summing over all particles, we can compute the total kinetic energy of the system:
\begin{equation}
T = \sum_{\rm i}^{N} T_{\rm i} = \frac{1}{2}m_{\rm i}v_{\rm i}^2
\end{equation}
Kinetic energy is by definition a positive quantity.}  
\end{tcolorbox}

\begin{tcolorbox}[sharp corners, colback=blue!30, colframe=blue!80!blue, title=Total Energy]
\par \textcolor{blue} {The total system energy, usually denoted $E$, is defined as the sum of the total kinetic and gravitational potential energies, or:
\begin{equation}
E = T + V.
\end{equation}
The total system energy can be either positive or negative.  The zero-energy boundary (i.e., $E = 0$) is critical to deciding the ultimate fate of a self-gravitating system of particles.}  
\end{tcolorbox}

\begin{tcolorbox}[sharp corners, colback=blue!30, colframe=blue!80!blue, title=Gravitational Binding Energy]
\par \textcolor{blue} {If the total system energy is negative, then the system is said to be gravitationally bound.  In practice, this means that the self-gravitating system will not disperse on a short timescale.  For positive total energies, the particles will begin to move farther and farther away from each other, until eventually they are hardly interacting via gravity at all.  This is because the typical relative distance between particles grows over time, reducing the denominator in Equation~\ref{eqn:force} and along with it the gravitational acceleration acting on each particle.  For negative total energies, the particles will remain grouped closely together, and continue to interact strongly due to gravity.  The system remains effectively "glued" together for a much longer period of time relative to analogous systems with positive total energies.  It is important to note, however, that in nature the system is never truly isolated and can lose mass over time.  This changes the total kinetic and potential energies, and the total system energy can change from negative to positive.  This marks the critical transition at which point infant star clusters enter a state of rapid dispersal or dissociation.}
%DERIVE CONDITION FOR A CLUSTER TO REMAIN BOUND.  EXTEND DERIVATION SO THE SOLUTION IS GIVEN AS A FUNCTION OF THE FRACTION OF ITS INITIAL MASS LOST. Quantify the timescale, and then relate it back to the lifetimes of the seven sisters to show that the cluster will disperse on a timescale much shorter than their lifetimes.}   
\end{tcolorbox}

%\begin{tcolorbox}[ams equation,sharp corners, colback=blue!30, colframe=blue!80!blue, title=Virial Theorem]
\begin{tcolorbox}[sharp corners, colback=blue!30, colframe=blue!80!blue, title=Virial Theorem]
\par \textcolor{blue} {If the total energy is negative, then it would seem that gravity out balances the effective outward pressure or heat provided by the particle kinetic energies.  So what prevents the whole system from collapsing inward to a singular point at the system center of mass?  This is where the Virial Theorem comes in, which provides a simple relation between the total kinetic and potential energies.  In its simplest form, the Virial Theorem states that for a self-gravitating system of particles, twice the total time-averaged kinetic energy plus the total time-averaged gravitational potential energy is equal to zero.  To see why this is the case, let us once again consider a self-gravitating system of $N$ particles.  For the $i$-th particle, we let the mass, distance from the system center of mass, velocity relative to the center of mass and linear momentum be denoted, respectively, $m_{\rm i}$, $\vec{r}_{\rm i}$, $\vec{v}_{\rm i}$ and $\vec{p}_{\rm i}$.   Now, we define the following quantity:
\begin{equation}
H = \sum_{\rm i=1}^{\rm N} \vec{p}_{\rm i} \bullet \vec{r}_{\rm i}.
\end{equation}
Differentiating with respect to time gives:
\begin{equation}
\frac{dH}{dt} =  \sum_{\rm i=1}^{\rm N} \frac{d\vec{p}_{\rm i}}{dt} \bullet \vec{r}_{\rm i} + \sum_{\rm i=1}^{\rm N} \vec{p}_{\rm i} \bullet \frac{d\vec{r}_{\rm i}}{dt}.
\end{equation}
With a few simple substitutions, this becomes:
\begin{equation}
\frac{dH}{dt} =  \sum_{\rm i=1}^{\rm N} \vec{F}_{\rm i} \bullet \vec{r}_{\rm i} + \sum_{\rm i=1}^{\rm N} m_{\rm i}\vec{v}_{\rm i}^2,
\end{equation}
since $\vec{F}_{\rm i} = d\vec{p}_{\rm i}/dt$ and $\vec{p}_{\rm i} = m_{\rm i}\vec{v}_{\rm i} = m_{\rm i}\frac{d\vec{r}_{\rm i}}{dt}$,
which can be re-written as:
\begin{equation}
\frac{dH}{dt} =  \sum_{\rm i=1}^{\rm N} \vec{F}_{\rm i} \bullet \vec{r}_{\rm i} + 2K.
\label{eqn:dhdt}
\end{equation}
Now, in order to arrive at the Virial Theorem, we must take a time-average of the above equation.  First, note that in general, the time-average of a variable $x$ is:
\begin{equation}
\bar{x} = \frac{1}{\tau}\int_{0}^{\rm \tau} x(t)dt.
\end{equation}
Hence, the time-average of the left-hand side of Equation~\ref{eqn:dhdt} is:
\begin{equation}
%\begin{empheq}
\bar{\frac{dH}{dt}} = \frac{1}{\tau}\int_{0}^{\rm \tau} [H(\tau) - H(0)].
%\end{empheq}
\end{equation}
Thus, for example, if the system is periodic and returns to its initial state after some time interval, then $\tau$ can be chosen to be equal to the characteristic period such that $\bar{dH/dt} = 0$.  Other reasons exist why $\bar{dH/dt} = 0$ might apply.  For example, if the system state is non-periodic but instead long-lived while remaining approximately stable, or at least deviations from the initial state are not too extreme, then over long timescales we have $\bar{dH/dt} \rightarrow 0$.  In the end, taking a time-average of Equation~\ref{eqn:dhdt} and setting the result equal to zero yields:
%\footnote{We note that this derivation is valid only for power-law potentials of the general form $V(r) = {\beta}r^{\alpha}$.}
\begin{equation}
2\bar{K} + \bar{U} = 0.
\label{eqn:vir}
\end{equation}
Equation~\ref{eqn:vir} is the Virial Theorem for a self-gravitating system of particles.}
%\end{tcolorbox}
\end{tcolorbox}

\par \Sterope We are all fated to wander independently through the Cosmos.  Utterly and completely alone.  Well, except for \rmmaia~ and \rmmerope, I suppose, who form a binary.  Oh, and the triplets.  Those three are also still gravitationally bound.

\par \nar \rmalcyone's shoulders slump.  She begins to cry.

\par \Alcyone Already, I miss each and every one of you.

\par \Taygete Well, at least you have me.

\par \nar \rmalcyone rolls her eyes.

\par \Taygete Hey!  I saw that!

\par \Alcyone I'm sorry, sister.  You are right.  I am grateful for your presence.  Even if it \textit{is} all the time.  Without any breaks.  Ever.

\par \Maia I'm afraid \rmalcyone~ is right.  It is now time for each of us to follow our own paths through the Universe.  Or, equivalently, to follow our own trajectories through space-time.  At least in your case, \rmtaygete~, your sister \rmalcyone~ will be accompanying you.

\par \Electra Hold on a second!  I don't like the sound of this one bit!

\par \Sterope Me neither! \rmelectra~ and I are going to be completely alone!

\par \Alphab AHEM!

\par \Betab You will have us with you, Mother!  We will follow you on your journeys!

\par \Sterope Of course, I do apologize.  Your words comfort me, thank you.  And \rmelectra~ will of course have the companionship or her three planets and moon, as soon as they finish forming from the protoplanetary disk around her equator.

\par \Electra I am sure that together, my satellites and I will meet all sorts of interesting characters, maybe even make a few new friends along the way.  

\par \nar Both \rmsterope~ and \rmelectra~ lock eyes, exchanging a sympathetic glance as they continue to drift apart.  
%They drift farther and farther away from their sisters and each other.

\par \Maia Do not worry, my fresh new stars. This is all a part of the Circle of Life.  As are you.  Something tells me it will not be long before you hear from me again. Keep your eyes peeled to the horizon, and I will soon be there.  

\par \Electra Sigh...  

\par \Taygete So... Uh... Wow.  This is awkward.  I guess we'll see you guys later.  I'm not sure how or when that will happen.  I can only assume it will involve some miraculous and possibly mysterious act of fate.  But I'm sure it \textit{will} happen.
  
\par \nar \rmtaygete~ and \rmalcyone~ snicker quietly to themselves, exchanging a glance of mutual understanding, doubting \rmmaia's prediction.  Ever the pessimists.  \rmelectra~ begins to cry.  \rmsterope~ joins suit.  They cry together for a while, before \rmsterope~ stops and says to her sister, sniffling loudly:

\par \Sterope Do not worry, \rmelectra.  It is an exciting time!  A new chapter in our lives.  What adventures will befall us?  What obstacles will we overcome?  

\par \nar \rmelectra~ interrupts her sister, the sarcasm rich in her voice:

\par \Electra How many times will I be overwhelmed by the situation, unsuccessfully trying to manage my anxiety by crying and blubbering uncontrollably? I \textit{can't wait} to find out!

\par \nar Despite the brave face, \rmmaia~ was every bit as terrified as her sisters.  The oldest among them, \rmmaia intently sought to calm her panicking siblings as she slowly drifted from view.

\par \Taygete Well, \rmalcyone.  It looks like we're stuck with 'ol \rmcelaeno~ over there.

\par \Alcyone Yep, looks like it.  She was already gravitationally bound to us pretty significantly before the gas left, so I guess it's no surprise that she's still here.  Perhaps a disappointment, but not a surprise.

\par \Celaeno Heeeeelllloooo over there!  Did you know that I can hear you?  I wish I couldn't.  But I can.  

\par \Taygete We know. 

\par \nar \rmtaygete~ and \rmalcyone~ exchange a wink of understanding.  \rmcelaeno~ mutters under her breath:

\par \Celaeno I hate you.  Both.  Profoundly.

\par \Taygete What was that?

\par \nar \rmcelaeno~ speaks louder, so her sisters can hear:

\par \Celaeno I \textit{love} you both.  Profoundly.

\par \Alcyone Aw.


\textbf{NEED AN ILLUSTRATION OF THE CLUSTER DISSOLVING, AND THE SISTERS CLEARLY DRIFTING FAR APART (I.E., SOME VERY FAR AWAY, SOME STILL CLOSE AND IN THE FOREGROUND, AND SO ON.}
%\begin{figure}
%\includegraphics[width=\columnwidth]{fig3.png}
%\caption{\rmmaia~ (blue), \rmelectra~ (red), \rmtaygete~ and \rmalcyone~ (both purple), \rmcelaeno~ (green) and \rmsterope~ (orange).  %Note that, as explained in the text, green and purple stars do not exist in the Universe.  Illustrations by Joshua Leigh.
%\label{fig:fig3}}
%\end{figure}


\end{document}
