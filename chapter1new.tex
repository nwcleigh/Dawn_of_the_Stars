\documentclass[main.tex]{subfiles}
\usepackage[T1]{fontenc}
\usepackage{ae,aecompl}


%%%%% AUTHORS - PLACE YOUR OWN PACKAGES HERE %%%%%

% Only include extra packages if you really need them. Common packages are:
\usepackage{graphicx}	% Including figure files
\usepackage{amsmath}	% Advanced maths commands
\usepackage{amssymb}	% Extra maths symbols
%\usepackage{lipsum}
%\usepackage{lmodern}
%\usepackage{tcolorbox}

%\newcommand\levelone[1]{{\color{red}\bf}}
%\newcommand\leveltwo[1]{{\color{blue}\bf}}
%\newcommand\levelthree[1]{{\color{green}\bf}}
\begin{document}

\section{Guidelines} \label{guide}

\par \nar Before we begin, let us first consider a few features in this book implemented to help communicate its contents.  


The first thing you will notice is that the text and story line have been segmented into a “box” structure.  This is used to define and communicate different levels of education needed as a minimum by the reader to understand certain key concepts contained within each “level”.  For example, level zero is shown in black text, and is not confined to any such boxes.  This “zeroth level” contains most of the underlying story line, and can be understood by readers having a base level of education attained in grade school (i.e., up to and including high school).  It does not require any knowledge of any text appearing in any other “levels” in order to understand its contents.  If the reader chooses to read only the black text, then they will be able to follow along with the story but will neglect access to some of the higher-level education materials presented in this book.  

Subsequent “higher” levels require more education to understand their contents.  All of the corresponding “higher level” education materials are contained within, and isolated to, the box format described above.  Level one is indicated by red boxes and lettering, and requires a basic undergraduate STEM-based education to understand the content.  All text appearing in red boxes assumes that the reader is following along with all text appearing in both red and black fonts.  Level two is indicated via blue boxes, and requires an education in undergraduate physics to comprehend.  All text appearing in blue boxes assumes that the reader is following along with all text appearing in blue, red and black boxes.  Level three is indicated via green boxes \note{A.S.H.: I find the green boxes hard to read. Maybe we should tweak the colors a bit? Also, we should bear in mind that having a lot of color in the book (text boxes plus illustrations) could be costly when publishing!} and lettering, and is aimed at the graduate level.  And so on.  

In summary, each level of text, indicated by a different type of box and lettering, and requiring the indicated background education, requires knowledge of the content of each level below it, but no knowledge of any level above it.  In principle, the reader can begin to appreciate the book from a very young age and, if they so choose, return to it again and again over the course of their life as their own personal education develops.  Each time the reader re-discovers this book, we hope they will discover new things, with new content emerging as it becomes accessible to them.

PYRAMID DIAGRAM SHOWING THE COLORS AND CORRESPONDING EDUCATION LEVELS?  Obviously, the highest level of education goes at the top, indicating all levels below it are required to follow the story development in these sections.  The base of the pyramid should be in black, given the above descriptions.

The second thing you will notice is that the characters in the story appear in corresponding illustrations.  These are included not just for entertainment purposes, but are instead primarily used to help convey the underlying story and certain key concepts.  We do our best to ensure that the illustrations are accurate and precise in terms of the underlying physics, to help communicate important concepts typically related to the physics governing the birth and subsequent evolution of stars.  For example, the sizes and colors of the stars in the story are accurate, given their relative masses.

%\textbf{NL:  I tried to define a few "rules" above, which would also extend to the color-coding of the text in footnotes, or on a different page or in the margin or whatever.  I propose we try to define these rules, and decide on the corresponding education levels, before we even begin working on the book.  Thoughts?  I also tried to include stuff where Pleione teaches Maia how to count, add, subtract, etc.  My hope is that this introduces the key concepts of "mathematics and numbers" and "computing", and will help you guys having fun doing whatever the hell you want with this and/or subsequent chapters.  With all of us operating under the same over-arching rules, I think it should be fine to do things like:  contribute at whatever level you want (and even request of me trying to integrate the underlying concepts into the story, from the bottom up, and such things.  I mean, I hope this will turn into "our book" instead of "my project" pretty quickly, but please keep honest.  That is my intention, anyways, or at least to be as sincere as possible when asking if you guys wanted to work on this.  Okay, crap, this got "sappy"...  Also, I still need to go through the rest of this, including this chapter, in more detail to correct for these ideas, if we choose to keep them.  But I figured we should all decide on the rules before doing that, maybe assigning chapters.  For example, chapter 2, I think it is, might work well for Adrian since it is focused on the triplets.  There is also cluster dynamics in globulars in one chapter, which could go in lots of directions.  Ditto when the black holes appear in the story line.  There is also the possibility of including planets around one or some of the stars, and trying to integrate those characters into things.  And so on.  Just brain storming a bit, or trying to.}
%\comment
%Chapter 1 (star formation, basic stellar physics related to concepts like hydrodynamic equilibrium, introducing the main concepts we wish to build on and come back to over the course of the book, such as mathematics and computation, if we want to stick with those overarching themes)
%Main author:  Nathan

%In this chapter, we introduce the main characters appearing throughout the book, namely the Seven Sisters.  A giant molecular cloud %named Pleione births seven stars, namely Maia, Merope, Sterope, Taygete, Celaeno, Elektra and Alcyone.  Maia and Merope form a %binary pair, and are the two most massive of the seven siblings (Merope > Maia).  Taygete and Alcyone form a compact binary, with %Celaeno orbiting the pair as an outer tertiary companion.  Elektra and Sterope are two isolated single stars.  Merope (= 23 solar masses) %> Maia > Celaeno > Sterope > Taygete = Alcyone > Elektra (DOUBLE CHECK THIS!).  

%This chapter begins with the births of the seven sisters, and their mother Pleione instructing them on the basics of communication, counting and numbers, and the development of basic arithmetic.  The chapter ends with the dissociation of the newly formed cluster, once the gas has been dispersed due to radiation pressure, lowering the gravitational potential in the process and causing the natal stellar cluster to expand.  It expands to the point that the seven sisters become gravitationally unbound, drifting apart and hence beginning their individual journeys throughout the Galaxy.

%In this chapter, we must decide and describe how the stars will communicate.  As it is, we are assuming that they will use the EM spectrum to communicate, modulating their luminosities temporally to effectively communicate in an analogous way to using morse code.  BUT, this immediately presents the challenge of having to worry about delays in said communication due to the light travel time becoming exceedingly long.  Of course, I would view this as an opportunity to explain certain key concepts while maintaining an interesting and/or useful story on which we can draw out the physical concepts we want to get into.
%	We also introduce and explain the concepts of numbering systems, counting and basic arithmetic (i.e., computation).  This is needed so that the main characters will be able to perform calculations later in the book. 
%\comment


%PROPERTIES OF STARS ARE GIVEN BELOW.
%In this chapter, we begin with our introduction of the main characters appearing throughout the book, namely the Seven Sisters.  A giant molecular cloud named Pleione births seven stars, namely Maia, Merope, Sterope, Taygete, Celaeno, Elektra and Alcyone.  Maia and Merope form a binary pair, and are the two most massive of the seven siblings (Merope > Maia).  Taygete and Alcyone form a compact binary, with Celaeno orbiting the pair as an outer tertiary companion.  Elektra and Sterope are two isolated single stars.  Merope (= 23 solar masses; the true mass seems to be about 4.5 M_sun with a radius of 5.1 R_sun, from Wikipedia, also called 23 Tau) > Maia (= 5 M_sun, 6 R_sun, spectral type B8III, 850 L_sun, also called 20 Tau) > Celaeno (4 M_sun, 4.4 R_sun, 344 L_sun, also called 16 Tau) > Sterope (2.93 M_sun, B8V star; REF?) > Taygete (2.1 M_sun; Taygete is actually a binary star system or Tau Aa and Ab, with true masses of about 4.5 M_sun for Aa and 3.2 M_sun for Ab) = Alcyone (2.1 M_sun; Alcyone is actually a triple star system...it seems the brightest star is a B-type giant in the inner binary with a very low-mass companion and an orbital period of about 4 days, whereas the outer companion is roughly half the mass of the B-type giant with an orbital period of about 830 days) > Electra (0.4 M_sun; also called 17 Tau I do not see a mass...i.e., Wikipedia failed me). NOTE:  Only Sterope has the technically correct mass, thanks to Maxwell, and the rest I made up.  We also only mention Merope’s mass in the story line, so this needs to be edited.  I guess we should dig into the correct masses a bit better, and make sure we check for consistency with the colors and sizes of the stars in the illustrations.  
%Adrian: maybe be more specific and specify the masses of all the stars?
%Nathan:  Agreed.  This is on the todo list!  I did make up the rest of the masses now, except for Sterope’s mass.  Still wondering if we should just look up what is actually known about these stars in the literature, and then maybe edit components of the story line accordingly.
%Nathan: Note:  We need to edit this section, to make sure we birth planets around at least some of the stars.  How many stars should have planets?  How many planets should each star have?  This is another place I think it would be cool to try to be accurate, or at least consistent with current data.  Advice/thoughts welcome.
%Maxwell: Good idea to mention the masses indeed. If we use MSun, should we somehow explain why this is a more commonly used unit than kg? The introduction of the MSun can be done in a box, since the seven sisters may not be aware of our Sun and therefore it is unnatural for them to normalize their masses to our Sun. The introduction to planet formation seems to be a bit too fast, but we will see how the story develops.
%Javier: will it make sense to indicate that the seven Pleiades do exist in nature even though the fictitious storyline takes off from there? That may help readers understand that although the story is fictitious the properties of the stars are based on real numbers.
%Nathan: I started looking into the details of these stars, and have come to realize I should have done this a long time ago.  Sincerest apologies.  I wrote in some rough notes using mostly wikipedia, which does give specific references but they need to be checked.  We should discuss how we want to handle the details of these stars, and how accurate we want to be,  It would be much cooler to be accurate, in my opinion, but this will require some over-arching decisions to go along with the “rules” we decide on.  Once decided, wew then need to think about to edit or re-structure the book to accommodate the more accurate or observed stellar properties.  I mean, both Taygete and Alcyone are actually unresolved multiples, the latter a binary and the former a triplet.  So we could just work with that directly, which might be easiest.
%
%This chapter begins with the births of the seven sisters, and their mother Pleione instructing them on the basics of communication, counting and numbers, and the development of basic arithmetic.  The focus is on the first born os the siblings, namely Maia.  The chapter should end with the birth of the next sister(s) in line, but save the births of the remaining siblings for the next chapter.  Pleione disperses at the end of this chapter.



\chapter{Life emerges...}

\par \nar For a long while, there was only darkness.  Well, mostly darkness.  Photons are to the Universe what microorganisms are to the world.  In a nutshell, that stuff is pretty much \textit{everywhere}.

%\footnote{Photons are particles of light.  They travel freely through vacuum at a speed of about 299792458 meters per second or 299792 kilometers per second or, if you prefer, 671000000 miles per hour; basically photons travel an unfathomable distance each and every second.  Photons are defined according to their energy or, equivalently, wavelength or frequency.  The spectrum of energies characterizing photons is called the electromagnetic (EM) spectrum.  Photons are produced in the cores of stars, eventually working their way up to escape from the surface.  These photons, often mostly from the visible portion of the EM spectrum, are transparent to the Earth's atmosphere.  They are detected by our eyes, revealing a wonderfully brilliant and colorful night sky on Earth.}  

%LEVEL ONE:
\begin{tcolorbox}[sharp corners, colback=red!30, colframe=red!80!blue, title=Photons]
%\lipsum[2]
\par \textcolor{red} {Photons are particles of light.  They travel freely through vacuum at a speed of about 299792458 meters per second or 299792 kilometers per second or, if you prefer, 671000000 miles per hour; basically photons travel an unfathomable distance each and every second.  Photons are defined according to their energy or, equivalently, wavelength or frequency.  The spectrum of energies characterizing photons is called the electromagnetic (EM) spectrum.  Photons are produced in the cores of stars, eventually working their way up to escape from the surface.  These photons, often mostly from the visible portion of the EM spectrum, are transparent to the Earth's atmosphere.  They are detected by our eyes, revealing a wonderfully brilliant and colorful night sky on Earth.}
\end{tcolorbox}

\section{Hello World} \label{hello}

\par \nar But on a very fateful day, everything changed.  Cosmic Dawn emerges with a roar.  An especially massive Giant Molecular Cloud is in the final stages of contracting; the internal pressure from within, provided by the random motions of her constituent atoms and molecules, guides the hand of gravity to re-shape her into a critical new state.  Over-dense knots and filaments begin to form within her belly.  The knots continue to coalesce, becoming ever hotter and denser.  Finally, new life emerges.  Deep within one such dense knot, the massive protostar \rmmaia is born, weighing in at a whopping 5 times the mass of the Sun, or 4.5 M$_{\odot}$ .

\par \nar With her birth, comes Dawn.  Protostars spew out radiation in the form of photons at a thunderous pace; enough to make an unfathomable mound of radioactive waste pale by comparison.  Seven siblings, all due to be born within the narrow window of a million years. Their Mother, \rmpleione, a particularly compelling Giant Molecular Cloud, now begins her journey through Motherhood.  But it's not yet over; she's still in the process of yielding to gravity's nurturing might, slowly contracting and compressing, forming over-dense filaments and birthing new stars within her bosom.  

\par \Maia Hello to you, Mother!

%\levelone
\begin{tcolorbox}[sharp corners, colback=red!30, colframe=red!80!blue, title=Stellar Communication]
\par \textcolor{red} {Stars of course cannot speak.  But they can communicate with each other, even over very large distances.  They communicate by modulating their luminosities on short timescales, brightening and dimming, brightening and dimming, in whatever cadence properly communicates their intended message.  Humans are unable to speak, write or even read the language of the stars.  Throughout this book, all communications between stars will be expressed in English.} 
\end{tcolorbox}

\par \Pleione Hello to you as well, my child.  My young new protostar!

\par \Maia Wow!  The Universe is so amazing and pretty.  Are all those twinkling things off in the distance other protostars, like me?

\par \Pleione Mostly yes, child.  But stars live long lives, and the protostellar phase does not last long; only a few million years.  So most of the far off stars you are looking at are much older than you.  Stars spew out light at a colossal rate, and this is how you are able to see them.  Light makes stars shine.  They can even appear to twinkle when very far away.
%\footnote{EXPLAIN TWINKLING!!!}

\par \Maia I see, stars have very long lives...  I'll take it!  

\par \Pleione How many stars do you see?

\par \Maia Uh... How \textit{many}?  I don't understand...

\par \Pleione Well, let's start at the beginning.  When it comes to counting, that is usually a good idea.

\par \Maia Counting?

\par \Pleione  Counting is a way to keep track of something.  For example, how much of something you have or can see, or how often something should happen.  In order to count, the first thing you will need is a numbering system.

%level one:
\begin{tcolorbox}[sharp corners, colback=red!30, colframe=red!80!blue, title=Numbering Systems]
\par \textcolor{red} {INSERT HISTORICAL CONTEXT/INFO ABOUT DEVELOPMENT OF NUMBERING SYSTEMS HERE.}
\end{tcolorbox}

\par \Maia What is that?

\par \Pleione A numbering system uses symbols and rules to do the keeping track of things.  The symbols are called "numbers".  The rules decide how to use those symbols to do the counting.  For example, if you add two positive numbers together, you always get a bigger number.  If you add two negative numbers, you always get a smaller more negative number.  

\par \Maia I'm not sure I'm following you anymore...

\par \Pleione. Okay, well, before giving up on me, let me try to explain two very key concepts in any numbering system.  The first is "zero" (i.e., 0).  This number just means that you are without any of whatever it is that you are counting.  None.  The second is "one" (i.e., 1).  This number is pivotal to any numbering system, since it defines the unit of measure.  For example, I am your Mother.  I, and I alone, created you.  There is only one of me.  You are my child, and you are also unique since there is only one of you.  But, if I were to create another star like you, then I would have more than one child.  In this case, I would have \textit{two} children.  This follows from defining the sum as 1 + 1 = 2.

\par \Maia I think I am following you this time.  What if we go back to before I ever existed?  How many children did you have then?  Was it zero?

\par \Pleione That's right!

\par \Maia Okay.  So, there were zero children a long time ago.  Then there was one.  Then two.  Then... wait, what 's next?

\par \Pleione The next number in the sequence is three.  But, before we get there, let's take a moment to consider the concepts of addition and subtraction in a little more detail.

\par \Maia Fair enough.  What about them?

\par \Pleione These ideas are rather central to all of "mathematics", or using numbers to calculate or compute things.

%level one:
\begin{tcolorbox}[sharp corners, colback=red!30, colframe=red!80!blue, title=History of Mathematics]
\par \textcolor{red} {INSERT HISTORICAL PERSPECTIVE ON THE DEVELOPMENT OF MATH HERE.}
\end{tcolorbox}

\par \Maia You're losing me again...

\par \Pleione One way to calculate something is to add numbers together, as I have already shown.  To summarize, I start by making one star.  I then make one more star.  Now I have two stars.  But each star is itself just one star.  So, here, we are adding one star together with another one star, and this gives us two stars.  So one plus one is equal to two, or 1 $+$ 1 $=$ 2.  We are adding stars together, and counting them as we go.

\par \Maia You're winning me back...  I think I'm following you...

\par \Pleione Now let's take it one step further.  If we have two stars, and we add one more star, how many stars do we have?

%level one:
\begin{tcolorbox}[sharp corners, colback=red!30, colframe=red!80!blue, title=Mathematical Operations]
\par \textcolor{red} {SHOW MATH, DEFINE THE MOST BASIC MATH RULES RELATED TO ZERO, ADDING, SUBTRACTING, ETC...  ADD HISTORICAL CONTEXT AS WELL.  DISCUSS OTHER NUMBERING SYSTEMS?  LOGARITHMS? } 
\end{tcolorbox}
 
\par \Maia Uh... Give me a second to think about it... \textit{Three!?}

\par \Pleione That's right!  

\par \Maia Look at me, I am adding!

\par \Pleione You most definitely are, my daughter.  Now, let me ask again:  How many stars do you see, twinkling off in the distance?

\par \nar \rmmaia turns her attention back to the distant stars.  

\par \Maia I see a LOT of stars!  In fact, I see so many I do not think I could count them all.

\par \nar Something catches \rmmaia's curiosity, distracting her from counting.  She gasps in wonder.

\par \Maia Whoa, if you look closely at some of those distant stars, they appear to be arranged in interesting ways that make them resemble familiar or just weird shapes.  Like, over there, I see \textit{three} stars that are bunched close together and form a straight line.

\par \Pleione Very good, \rmmaia.  You are counting!  That is Orion's Belt.  Good eye!  If you take a larger look at him, you will notice as well a torso, arms and legs.  

\par \Maia I think I see them... Wait what are arms and legs?  

\par \Pleione Orion the Hunter is in the form of a human.  I would describe humans as resembling deformed stars; they look similar, and come in various colors and sizes.  But they also come along with many protuberances, such as arms and legs, each with their own set of functions.   Orion is but one example of the many stories depicted in the night sky, assigned by that species.  Humans call these familiar stellar configurations ``constellations'', and they are meant to tell some important story about their history.  Humans have developed many stories to explain their origins.

\par \Maia Have you ever seen one?

\par \Pleione One what?

\par \Maia A human.

\par \Pleione Oh!  Yes, once, quite some time back.  Awful, vile species.  Constantly shooting projectiles off the surface of their tiny planet, littering outer space with their garbage.

\par \Maia Yuck.  That does sound gross.  
%Well, I thought the third leg should be shorter because it's supposed to be back a bit, and I thought that gives it a 3-dimensional feel.  What did \textit{you} think it is?

%\newpara \Pleione Uh... Hey! 
\par \Pleione Did you know that Orion the Hunter even has a bow to fire arrows at his enemies!?  If you look closely, you can see he is holding it in his left hand, and it forms a large arc in the sky.

\par \Maia I see it!... Wait... Enemies?  What kinds of enemies?

\par \Pleione Well, if you follow Orion's Belt from left to right, you will pretty quickly notice a very bright red star.  That is the star \rmaldebarran, the Eye of Taurus the Bull.  According to the myth, Orion the Hunter fought Taurus the Bull to save the Seven Sisters.  

%level one:
\begin{tcolorbox}[sharp corners, colback=red!30, colframe=red!80!blue, title=Aldebaran]
\par \textcolor{red} {Aldebaran is the brightest star in the constellation Taurus.  At a mass of 1.7 M$_{\odot}$, Aldebaran is both older and redder than the Sun, and intrinsically brighter.  Since it is the process of ascending the giant branch of stellar evolution, which comes after the main-sequence phase, it is preparing to fuse the hydrogen and helium in its core into more massive elements.  Aldebaran shines red, since its surface temperature exceeds 3,700 C.}
\end{tcolorbox}
%Describe what kind of star is Aldebaran...a red giant?  Describe the red giant phase of evolution.}

\par \Maia Wow, that sounds very dramatic.

\par \Pleione I suppose it must have been.

\par \nar \rmpleione~ was growing weary.  Each of her children emits a wind of charged particles and photons escaping wildly from their surface.

%\levelone
\begin{tcolorbox}[sharp corners, colback=red!30, colframe=red!80!blue, title=Stellar luminosity]
\par \textcolor{red} {Over 10$^{38}$ photons are emitted every second.  Any tiny patch on the surface of a star is like the narrow entrance to a dark cave, with a swarm of bats flocking out from it.  In order to calculate the total energy emitted per second by the Sun, we can assume that every photon escaping from its surface has an energy of 12.86 Mev, corresponding to the highest energy photons produced at the end of the proton-proton-chain (thus our estimate here for the total number of photons should be regarded as a strict lower limit), which is the nuclear reaction process responsible for converting hydrogen in to helium.  Assuming 1 J $=$ 1.602 $\times$ 10$^{-13}$ MeV, we then calculate a solar luminosity of about 2.1 ${\times}$ 10$^{26}$ J s$^{-1}$.  This is very close to the total value of 3.828 $\times$ 10$^{26}$ J s$^{-1}$ computed for the luminosity of the Sun, calculated from integrated observations of the solar spectrum (REF). Finally, we note that [J s$^{-1}$] = [Watts].}  
\end{tcolorbox}

\par \nar As the winds collide with the loving embrace of their Mother, they provide an outward pressure and she begins to disperse.  The birth of \rmmaia had initiated the demise of her mother.  

\par \Maia Wait, Mother, where are you going?  

\par \nar \rmmaia, the second most massive of her soon-to-be-born siblings, wears a worried expression upon her face that begets deep concern for her fleeting Mother.  

%\Maia And why is the entire Universe spinning?  Ugh... I think I might be sick.

%\Mother My child, it is not the Universe that spins, but you.  All stars are born rotating, and you are no excpetion.  But fear not, you were also born with a magnetic field, and this will spin you down over the next several tens of millions of years.

%\Maia Oh, thank goodness.  For a second there, I was worried I would be rapidly rotating forever.  That's a relief.  But, wait, back to where you are going...? 

%\Maia Mother!  Mother!  Don't go!

\par \Pleione Oh, young one.  There is nothing to worry about.  I will be with you always, no matter what adventures befall you.  
%Please, take me with you, with an open heart.

\par \Maia That sounds suspiciously like a goodbye...

\par \Pleione Shhhh, little one.  All will be well, you will see.  You are only just now born and still contracting, as gravity continues to find its balance with the fires that now rage within you.  Hydrostatic equilibrium awaits.

%\levelone
\begin{tcolorbox}[sharp corners, colback=red!30, colframe=red!80!blue, title=Hydrostatic Equilibrium]
\par \textcolor{red} {Hydrostatic equilibrium is what ultimately decides the size or radius of a star.  The term refers to the balance between the outward pressure supplied by the energy released in the core via nuclear reactions (e.g., the proton-proton-chain, which is what burns hydrogen into helium) and the inward pull of gravity.}
\end{tcolorbox}

\par \Maia Um... You're leaving me with a complicated technical term like "hyrdrostatic equilibrium"...?  What does that even mean?

\par \Pleione It means that the energy produced within your belly must be strong enough to balance the inward force of gravity, to ensure stability.  You can think of it as a sign that you are healthy.  So long as you are in hydrostatic equilibrium, you never need to worry about your health.

%level two:
\begin{tcolorbox}[sharp corners, colback=blue!30, colframe=blue!80!blue, title=What determines the radius of a star?]
\par \textcolor{blue} {Stars out of hydrostatic equilibrium will either expand or contract.  If the pressure exceeds the gravity, then they expand.  If the gravity exceeds the pressure, then they contract.  This re-adjustment of the stellar radius is needed to set the right balance between pressure and gravity which.  The radius of a star can only really be defined for stars in hydrostatic equilibrium, since the radius remain approximately constant in this state.  Conversely, for stars out of equilibrium, the radius is dynamically changing as the star re-adjusts itself to find a new stable steady-state balance between pressure and gravity.}
\end{tcolorbox}

\par \Maia Okay, I will remember that one.

\par \Pleione Take care of your siblings for me...

%\newpara \Maia Uh... Okay...  Still kind of confused over here...
\par \nar \rmmaia and \rmpleione continue their conversation, \rmpleione doing her best to prepare her child for her inevitable journey through the Galaxy.  

\par \nar Shortly later, one of \rmmaia's siblings awakens.  \rmelectra~ emits a long, sleepy yawn.  The least massive of her siblings, \rmelectra weighs a mere 0.4 M$_{\odot}$.  \rmpleione turns her attention toward her newly born daughter.

\par \Pleione Behold!  Your sister awakens!

\par \Electra  Uh... Hi.

\par \Maia Oh, wow.  She is \textit{super} bright and, well, shiny.  It hurts my eyes if I stare right at her... Wait, \textit{is} this hurting my eyes?  Like, could I go blind?

\par \Pleione Only if you look directly at her.

%\levelone
\begin{tcolorbox}[sharp corners, colback=red!30, colframe=red!80!blue, title=Stellar Emission]
\par \textcolor{red} {Stars emit light spanning a wide range of energies.  The shielding effects of the Earth's atmosphere protect human eyes from very high-energy photons that would otherwise contribute to the degradation of the human eye.  From the surface of the Earth, we only see those photons within the visible portion of the electromagnetic spectrum.  But from space, our eyes would not be protected.  If stars' eyes are also sensitive to high-energy photons, then looking directly at other stars, especially very close ones, is anything but a good idea.}
\end{tcolorbox}

\par \Maia But I already did that!

\par \Pleione Are you blind?

\par \Maia I don't think so.

\par \Pleione How can you be sure?

\par \Maia Well, I can see you wincing, for one thing.  You are looking at me as if I just got in to a fight with a much more massive star and lost.

\par \Pleione Uh... I'm sure you'll be fine. Plus, it could have been worse.  It could have been a black hole!  They pack a far greater punch than any star, let me tell you.

\par \Maia That was less than convincing.  Wait, what is a ``black hole''?

%\levelone{
\begin{tcolorbox}[sharp corners, colback=red!30, colframe=red!80!blue, title=Black Holes]
\par \textcolor{red} {A good question.  We will learn a great deal about black holes over the course of this book.  For now, let us suffice it to say that many black holes are simply dead stars. Their progenitor stars ran out of nuclear fuel, which provided the star with the outward pressure it needed to resist gravity's inward pulling might.  With no source of outward-directed pressure, gravity won and the progenitor star collapsed to form a new, much denser object.  If the progenitor star was sufficiently massive, it would have collapsed to form a black hole.  From death, comes new life.  Black holes are so dense that the strength of gravity forbids the escape of light from their interiors.  Thus, they are black, and do not emit light.  They are only detectable by humans indirectly, via their gravitational influence on surrounding matter and stars.  Of course, this only applies to low-mass black holes, comparable in total mass to massive stars.  The origins of super-massive black holes, on the other hand, are thought to be much more complicated, and to this day remain shrouded in mystery...}
\end{tcolorbox}

%level two
\begin{tcolorbox}[sharp corners, colback=blue!30, colframe=blue!80!blue, title=Light From Accreting Black Holes]
\par \textcolor{red} {EXPLAIN HOW HIGH-ENERGY RADIATION BEING EMITTED WHEN BLACK HOLES ACCRETE MATTER!!!}
\end{tcolorbox}


\par \Pleione Let's save that one for another day, child.  You've already had quite an eventful day.

\par \nar \rmpleione's gas tendrils, swirling and coalescing around her birthing children, gently touching and massaging their young faces, continue to dissipate, faster and faster with the birth of each new star.

%level two
\begin{tcolorbox}[sharp corners, colback=blue!30, colframe=blue!80!blue, title=Radiation Pressure]
\par \textcolor{blue} {More stars means more photons, and hence more overall radiation pressure applied to dust particles.  This accelerates the rate of dispersal of a surrounding cloud of gas and dust.  But how does this work?  DEFINE RADIATION PRESSURE.}
\end{tcolorbox}

\par \nar \rmelectra~ interrupts them suddenly, belching loudly.  Plasma is ejected from her surface, emanating from above the equator.  

\par \nar Her daughters now all born and slowly coming to life, \rmpleione's time has arrived. \rmpleione~ bestows one last kiss upon her daughters, before floating off and dispersing in to the infinite vacuum.

\par \Electra It looks to be a lovely day we have on our hands here. ...I feel as though I just missed something important.  Please do fill me in at your earliest convenience.  Wait, what are those two whispering about?

\par \nar Both \rmmaia~ and \rmelectra~ turn their gaze toward \rmtaygete~ and \rmalcyone~, who together form a compact binary star system.

%\leveltwo
\begin{tcolorbox}[sharp corners, colback=blue!30, colframe=blue!80!blue, title=Gravitationally Bound]
\par \textcolor{blue} {Two objects are said to be gravitationally bound if their total relative energy (i.e., the sum of their kinetic and potential energies) is negative.  In this case, the objects orbit their mutual center of mass, carving out circular or elliptic trajectories in a plane.  The Earth is gravitationally bound to the Sun, as is the moon to the Earth.  Technically, the moon is also gravitationally bound to the Sun.  But gravity gets weaker with increasing distance, and the moon is close enough to the Earth and far enough away from the Sun that it orbits the former instead of the latter.}  
\end{tcolorbox}

\par \nar Gravitationally bound, the sisters orbit their mutual center of mass in harmony.  The two sisters are in fact twins, each with a total mass of 2.1 M$_{\odot}$.  Needless to say, they were close.  \rmtaygete~ and \rmalcyone~ quietly conferred about the topic at hand, namely which of the two of them is the brightest.

%MUST DO:
%PIPE:  CAN WE MAKE A MAP OF THE GALAXY WITH THE LOCATION OF THE PLEIADES INDICATED, AND EACH STAR'S TRAJECTORY THROUGH THE GALAXY?  FOR CHAPTER ONE?

\par \Taygete I think it goes without saying that I am brighter than you are.

\par \Alcyone Dream on!  I outshine you for sure.

\par \Taygete Alright, tough stuff.  Want to know how I know that I am brighter than you are?

\par \Alcyone Sure.  Amuse me.

\par \Taygete I'm definitely fatter than you are, and bigger.  Both contribute to making me brighter, relative to \textit{you}.  At least, I'm pretty sure that is how it works...  

%\leveltwo{
\begin{tcolorbox}[sharp corners, colback=blue!30, colframe=blue!80!blue, title=Dependence of Luminosity on Mass and Radius]
\par \textcolor{blue} {The luminosity of a star increases steeply with both increasing mass and radius.  This is the case during the main-sequence phase of a star's lifetime, during which time stars are burning hydrogen into helium in their cores.  All stars, once finished with the protostellar phase, become main-sequence stars.}
\end{tcolorbox}

\par \Alcyone Oh shut up...  You are neither fatter nor bigger than I am.  You \textit{are} way more delusional than I am.  I'll give you that.

\par \nar Rotating their inner cores ever faster, \rmtaygete~ and \rmalcyone~ strengthen their magnetic fields.  Long filaments of plasma wave out from their respective surfaces, not unlike arms reaching outward.  They begin flailing at each other violently, intent on a fight.  But they lie outside of each other's grasp, unable to reach with even the longest tendril of plasma either can muster; a sisters' quarrel unrealized.  Their efforts futile, they quickly give up.  

\par \nar \rmtaygete~ and \rmalcyone~ are in fact two members of a triplet.  \rmcelaeno, the third component of the triplet, lies much farther away than the other two.  This, combined with the fact that she is a more massive star weighing in at 4.0 M$_{\odot}$, makes \rmcelaeno a little bit different from her twin sisters.  She is often ridiculed by her fellow twins because of it.

%\levelone{
\begin{tcolorbox}[sharp corners, colback=red!30, colframe=red!80!blue, title=Dynamical Stability]
\par \textcolor{red} {This distance is absolutely necessary to ensure the long-term dynamical stability of the triple;  if the inner pair becomes too wide, the gravity exerted by the outer object will pull them apart.  Chaos will ensue.  All Hell will break loose.  This chaos can mediate the ejection of one or more stars from the triple, and even direct collisions.}  
\end{tcolorbox}

\par \nar The triplets' current configuration, hierarchical and dynamically stable, is nothing short of fate; binding them to each other indefinitely.

%\levelone{
\begin{tcolorbox}[sharp corners, colback=red!30, colframe=red!80!blue, title=Orbital Hierarchies]
\par \textcolor{red} {The easiest way to explain a "hierarchy" in a triplet is if two of the stars form a very compact binary, and the third star orbits at a very large distance from this compact pair.  In such a scenario, the inner compact binary can be regarded as a single star from the point of view of the outer triple companion, for all practical purposes.  Said another way, when it comes to a hierarchical triple star system, there are two orbits with almost opposing properties; the inner binary is compact, whereas the outer triple is on a very wide orbit.}
\end{tcolorbox}

\par \Maia Well, I think you are almost certainly identical twins.  I cannot see any real difference between you.  I mean, look at $\rmcelaeno$ over there; she's blue, whereas the two of you are clearly more of a yellow color.  She's also \textit{much} fatter and bigger than the two of you combined.  

\par \Celaeno Alright, I see your point.

\par \Taygete Agreed.  \rmalcyone, I'd extend a hand in offer of peace, but I don't have one.

\par \nar \rmtaygete once again churns her inner dynamo, brightening in response.  Tendrils of hot plasma shoot outward from her surface, mimicking arms and hands that wave at her sister.

%\levelone{
\begin{tcolorbox}[sharp corners, colback=red!30, colframe=red!80!blue, title=Magnetic Fields and Dynamos]
\par \textcolor{red} {By making the inner core of a star rotate faster, an existing magnetic field can be amplified.   This can in turn trigger substantial chromospheric activity in the star, with filaments of hot plasma extending radially outward and coronal mass ejections bursting through its surface.}  
\end{tcolorbox}

\par \nar Meanwhile, \rmcelaeno~ had inspecting her midsection meticulously.  Yep, fatter.  Unsure as to whether or not this was a good thing, \rmcelaeno~ wore a pensive expression, clearly trying to work it out in her head.  She was distracted from this self-introspection when she noticed a dense whisp of \Pleione passing between her and her twin sisters.  

\par \Celaeno What's that?

\par \Maia The fleeting remains of our Mother, I am afraid.

\par \Electra That's \textit{Mom}?!

\par \Maia Well, what's left of her.

\par \nar The siblings continued to accrete mass from what remained of their Mother, \rmpleione.  They each grew and grew, until eventually they reached a steady-state configuration.

%\levelone{
\begin{tcolorbox}[sharp corners, colback=red!30, colframe=red!80!blue, title=Steady-State]
\par \textcolor{red} {The term ``steady-state'' here implies that the stars are maintaining an approximately constant mass.} 
\end{tcolorbox}

\par \nar This marked the end of their growth, and ultimately the end of the protostellar phase of their lives.  In the end, the outward pressure produced from within due to the thermonuclear reactions brewing in their bellies had grown sufficiently strong to balance the inward pull of gravity.  Hydrostatic equilibrium achieved!  With this balance in place, the siblings would endure most of their lives in this approximate steady-state configuration, slowly fusing the lowest mass nucleon (hydrogen) into the next best thing (helium).

\par \nar \rmsterope~ came to life suddenly, announcing her appearance with a high-pitched scream.  With a mass of 2.93 M$_{\odot}$,\rmsterope appears most similar to \rmtaygete and \rmalcyone relative to her other siblings.

\par \Sterope AAAAAAAHHHHHHhhhh!!!!!  What the Hell, man?  Where am I?  What am I?  When...?  You get the idea.

\par \Maia It's okay, sister.  You are one of us.  We are stars born of the gas and dust of our Mother, a particularly glamorous Giant Molecular Cloud, if I do say so.  She has left us now, but not without first bestowing her deepest gift upon us all, along with all of her love.

\par \Sterope  Uh... You are all my sisters?  We are a family?

\par \Maia Yes!

\par \Sterope  In that case, there remains a slim chance that the rest of this conversation will proceed without me feeling the need to scream again.

\par \Maia Progress!

\par \Sterope Uh, yeah, right.  Progress.  Let's get down to brass tacks.  Who are all you strangers?  You are my sisters, that much I have gathered.  But what else?  Wait, who am \textit{I}?  More importantly, \textit{what} am I...?  I'm starting to feel another scream coming on...

\par \Maia Relax, young one.  You're in good company here.  Familiar company.  \textit{Familial} company, even. We are your siblings and we are stars.  Thus and therefore, you too are a star.  

\par \Sterope Okay.  But what is a star?  And does it have anything to do with why I am feeling so bloated?

\par \Alcyone I wasn't going to say anything, \rmsterope, but you do look a little red in the face.  Is everything okay over there?  Wait, you asked a good question.  What the heck is a star, anyways?

\par \Maia We are born of our Mother.  Plain and simple.  We formed out of the gas and dust she left behind, after gravity coalesced us into the beautiful burning spheres of hydrogen you see before you.  Inside, we home a nuclear furnace that generates energy and emits light.  Our insides are so hot, that hydrogen is converted in to helium, releasing photons and hence energy in the process.  The hydrogen is our food!  Outside, gravity pushes inward, but it cannot surpass the outward push provided by our internal metabolisms.  Protostars will continue to contract in to a denser state with a hotter core, until a critical balance is achieved, called hydrostratic equilibrium.  This will also get rid of the reddish hue you currently find yourself with, \rmsterope.

%\leveltwo
\begin{tcolorbox}[sharp corners, colback=blue!30, colframe=blue!80!blue, title=Blackbody]
\par \textcolor{blue} {A blackbody is an idealized object that absorbs all the incident radiation that falls on it at all frequencies.  Hence, incident visible light will be absorbed and not reflected, causing the surface of the object to appear black.  An ideal blackbody in thermal equilibrium adheres to two important properties.  First, it is an ideal emitter.  This means that, at every frequency, it emits as much or more thermal radiation compared to any other body at the same temperature.  Second, it is a diffuse emitter.  This means that the radiation is emitted isotropically or, equivalently, the emitted radiation is the same in all directions, when the radiation is measured per unit area perpendicular to a specified direction.}
%EXPLAIN BLACKBODY.  EXPLAIN WIEN'S LAW AND THE RELATIONSHIP BETWEEN COLOR AND TEMPERATURE.}
%%\par \textcolor{blue} {\nar Discuss/define blackbodies, their properties, and how this relates to second law of thermodynamics.  Could include some math showing how to derive each limiting approximaiton (e.g., Rayleigh-Taylor Law, Wien's Law, etc.).}
\end{tcolorbox}

\par \Sterope Well, that's a relief: the bloating is only temporary.

\par \Alcyone I think I am following what you are saying, at least so far.  What do we need to eat to keep ourselves going?  I mean, we must need energy?

\par \Maia You have plenty of energy to keep you going for billions of years!  You're consuming the hydrogen you were born with; converting it in to helium right there in your belly.  It's a gift, just enjoy it.  The consequence of this act of consuming is that you shine very bright.  Photons are emitted every time four hydrogen atoms are consumed to produce helium, and they leak through your body and emanate from your surface.  Bright as a light!  The nuclear fuel already stored within you is sufficient to last millions, even billions of years.  You'll be shinning practically forever!

\par \Celaeno Sounds to me like an awful lot of time to kill.

\par \Maia There will be plenty of adventures along the way to keep you distracted, I have no doubt.

\par \Sterope  Like what?  

\par \Maia Only time will tell.  But each star inevitably follows its own path through the Cosmos, and realizes its own fate.  We are individuals, after all.

\par \Sterope \rmmaia, how do you know so much?

\par \Maia Well, I don't really. I know what Mother told me.  I am the oldest of us, after all, and she explained as much as she could to me before dispersing.  

\par \Sterope I'm grateful for your efforts.  Mother dispersed so quickly, it must have been hard for her to convey a lot of detailed information to you before dispersing so completely.

\par \Maia It was.  She spoke really fast.

\par \Sterope And you remembered all of it?

\par \Maia Yep.  No problem!  

\par \Sterope Reeeeaaaaalllly, \rmmaia?  All of it?

\par \Maia \textit{Sigh}.  Fine!  Mother only told me a few things.  I don't want anybody to panic, so I'm trying to convey that Mother left me feeling confident, like we are more than capable of figuring it out for ourselves.  I'm only trying to help!  
%So there's a non-negligible chance that I'm making a lot of it up as I go along.  But \textit{most} of it is right, and straight from Mother.  At least... I'm pretty sure.  Take the filler with a grain of salt though.  

\par \Sterope Fair enough.  It sounds like you are doing your best.

\par \Maia I'm trying to motivate the lot of you, make you feel loved, important, etc.  Look, you get the idea. 

\par \nar \rmmaia's shoulders slumped as she let out a prolonged sigh.

\par \Sterope It's okay, \rmmaia.  We love you too.

\par \nar \rmsterope~ flashes a warm smile at \rmmaia.  \rmmaia~ smiles back, relieved.  \rmalcyone~ belches loudly.

\par \Sterope \rmalcyone!

\par \Alcyone I'm sorry!  It was an accident.  The magnetic field lines are churning and wrapping around themselves inside my belly.  They keep breaking out and reconnecting, as if of their own free will.  I'm learning that spontaneous emissions are, unfortunately, inevitable.  Way out of \textit{my} control, at least.

%\levelone{
\begin{tcolorbox}[sharp corners, colback=red!30, colframe=red!80!blue, title=Reconnection of Magnetic Field Lines]
\par \textcolor{red} {Most main-sequence stars have magnetic fields that typically emanate from their poles; the younger the star, the more powerful the magnetic field.  When two or more magnetic field lines intersect, they ``reconnect'' to form new, disconnected field lines.  This ``reconnection'' is usually an energetic event, accompanied by a burst of high-energy photons (i.e., gamma rays and x-rays).}   
\end{tcolorbox}

\par \nar Synchronized to the microsecond, \rmmaia~ and \rmsterope~ both roll their eyes.

\par \Maia Just do your best to keep your spontaneous emissions to yourself.

\par \Alcyone Will do. 

\par \Electra Uh...\rmmaia, I definitely don't mean to startle you, but some freaky, ominous stuff is going on right behind you.

\par \Maia Your goal there was to \textit{avoid} startling me?

\par \Electra Yep.  How'd I do?

\par \Maia Not very well at all.  I'm currently terrified of what might be lurking behind me.  Okay, I am turning around now...

\par \nar \rmmaia~ turns to see gas and dust had coalesced into a dense knot behind her.  She recognized right away the familiar dynamical dance choreographed by gravity; the final stages of the birth of yet another star, another sibling.

\par \Maia Oh, how wonderful!  We are witnessing the birth of our seventh sibling.  It would seem that Mother is not yet finished.

\par \nar \rmmerope~ came to life with a sudden jolt.  And the hiccups.  
%She is the second most massive of her siblings, weighing in at 4.5 M$_{\odot}$.

\par \Merope \textbf{Hiccup!}  Excuse me.  That whole being born thing was a little weird, and \textbf{Hiccup!} kind of uncomfortable.  It left with me extra gas in my belly, or \textbf{Hiccup!} something else that has given me the hiccups.  

\par \nar \rmmerope~ takes a minute to relax and compose herself.

\par \Merope Okay.  I'm feeling better now.  

\begin{figure}
\includegraphics[width=\columnwidth,angle=270,origin=c]{fig1.pdf}
\caption{\rmmaia~ and \rmmerope, who together form a gravitationally bound binary star system.  Illustration by Andre Pipe Oliva.
\label{fig:fig1}}
\end{figure}

\par \Sterope Super!  I'll try to find solace in your comfort as I struggle to ignore the lingering stench of your quasi-belches...  Wait, who are you?

\par \Merope Oh right.  Introductions! I knew I was forgetting something.   Hi!  I'm \rmmerope!

\par \rmmerope~ was the second most massive of her siblings, weighing in at 4.5 solar masses.  Gaseous emissions aside, her presence was hard to ignore amidst the seven sisters.  

\par \Maia It's wonderful to meet you, sister.  It would seem that you and I form a bound pair.  A binary star system!  How fortunate that gravity is an attractive force.  Our mutual gravitational attraction will keep us in this configuration practically forever.  Well, at least until one of us explodes or something.

%\levelone{
\begin{tcolorbox}[sharp corners, colback=red!30, colframe=red!80!blue, title=Supernovae and Stellar Lifetimes]
\par \textcolor{red} {\nar The most massive stars end their lives with a dramatic explosion, called a supernova.  In one go, the explosion can liberate roughly as much energy as the Sun over its entire 10 billion year lifetime.  At their peak, supernovae shine 10$^{10}$ times brighter than the Sun.  There is a simple formula that allows us to calculate the expected lifetime of a main-sequence star, which is $\tau_{\rm MS} =$ 10(m/M$_{\odot}$)$^{-2.8}$ Gyr (REF? ADRIAN?).  Using this formula, for Merope, Maia, Calaeno, Sterope, Taygete, Alcyone and Elektra, we find MS lifetimes of, respectively, 0.1, 0.1, 0.2, 0.5, 1.3, 1.3, 130 Gyr.  As we will show later on, these lifetimes should exceed the timescale for the dispersal of the Pleiades open cluster.}
\end{tcolorbox}

\par \Merope Wait, what!?  Who's exploding!?  Is it me!?  I don't want to explode!

\par \Maia Shhhh....  Relax, sister.  Nobody is exploding today.  
%, or tomorrow or any other day close enough to the present that we can count the number of days between now and then.  

\par \Merope Today!?  What about tomorrow?  

\par \Maia Nobody will be exploding tomorrow either.

\par \Merope And the day after that?

\par \Maia Nobody.

\par \Merope And the day after that?

\par \Maia Certainly not.

\par \Merope And the day after...?

\par \nar \rmmaia~ interjected before \rmmerope~ could finish.

\par \Maia Nobody will be exploding for a very long time, if ever.

\par \Merope Okay.  It doesn't seem immediately urgent, I guess. But we are \textit{definitely} circling back around to this exploding business at some point...

\begin{figure}
\includegraphics[width=\columnwidth]{fig2.jpg}
\caption{The seven sisters together.  Illustration by Andre Pipe Oliva.
\label{fig:fig2}}
\end{figure}

\par \Maia We will, I am sure.  But for the moment it seems we have a family to become acquainted with.

\par \nar \rmmaia~ turned to address her siblings.

%\textbf{THIS IS A GOOD PLACE TO INCLUDE ANY PLANETARY SYSEMS THAT ARE ALSO BEING BORN AROUND ANY STARS, IF ANYBODY WANTS TO INTRODUCE SUCH CHARACTERS.  NL:  I put several planets orbiting Electra... This was decided arbitrarily though, and we can change however we like moving forward.} 

\par \Maia Greetings to you all!  I cannot express how happy I am on this day, the day of our mutual births.  The matter that forms our bodies comes from the same Mother, and to her we owe homage!  Our existence is blessed by her great sacrifice, having largely spent herself to birth us few.  Seven massive siblings, and countless more familial satellites in the form of planets, moons, comets and even asteroids.  I see \rmelectra has a number of planets with stable orbits about her center of mass!  

\par \Electra I'm as surprised as you are!  But I am eagerly looking forward to getting to know them better.

%\levelone{
\begin{tcolorbox}[sharp corners, colback=red!30, colframe=red!80!blue, title=Satellites]
\par \textcolor{red} {The term satellite refers to any celestial body that is gravitationally bound to, but much less massive than, the seven sisters.  Very small stars, brown dwarfs, planets, moons, comets, asteroids, etc.} 
\end{tcolorbox} 

\par \Maia All born of the same stuff, in the same place, and at about the same time.  It is truly a time to celebrate.  But we are all weary of a prolonged dawn, and should now rest.  When we awake, we will celebrate properly!

\par \Merope  Count me in!  

\par \Electra A party sounds great. \textit{Yawn}. Just after I get a little shut eye.

\par \Sterope I could go for a nap. Then a party.  I'm in too.

\par \nar Meanwhile, \rmtaygete, \rmalcyone~ and \rmcelaeno~ had fallen asleep, and were snoring loudly.  Seven siblings, all born within the narrow window of a million years.  Their futures look bright.

\end{document}
