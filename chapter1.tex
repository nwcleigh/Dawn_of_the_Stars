\documentclass[main.tex]{subfiles}
\begin{document}

\chapter{Life emerges...}

\par \nar For a long while, there was only darkness.  Well, mostly darkness.  Photons are to the Universe what microorganisms are to the world.  In a nutshell, that stuff is \textit{everywhere}.\footnote{Photons are particles of light.  They travel freely through vacuum at a speed of about 299792458 meters per second or 299792 kilometers per second or, if you prefer, 671000000 miles per hour; basically photons travel an unfathomable distance each and every second.  Photons are defined according to their energy or, equivalently, wavelength or frequency.  The spectrum of energies characterizing photons is called the electromagnetic (EM) spectrum.  Photons are produced in the cores of stars, eventually working their way up to escape from the surface.  These photons, often mostly from the visible portion of the EM spectrum, are transparent to the Earth's atmosphere.  They are detected by our eyes, revealing a wonderfully brilliant and colorful night sky on Earth.}  

\section{Hello World} \label{hello}

\par \nar But on a very fateful day, everything changed.  Cosmic Dawn emerges with a roar.  An especially massive Giant Molecular Cloud is in the final stages of contracting; the internal pressure from within, provided by the random motions of her constituent atoms and molecules, guides the hand of gravity to re-shape her into a critical new state.  Over-dense knots and filaments begin to form within her belly.  The knots continue to coalesce, becoming ever hotter and denser. Until, finally, new life emerges.  Deep within one such dense knot, the protostar \rmmaia is born!

\par \nar With her birth, comes Dawn.  Protostars spew out photons at a thunderous pace; enough to dwarf an unfathomable mound of radioactive waste.  Seven siblings, all due to be born within the narrow window of a million years. Their Mother, \rmpleione, a particularly compelling Giant Molecular Cloud, now begins her journey through Motherhood.  But it's not yet over; she's still in the process of yielding to gravity's nurturing might, slowly contracting and compressing, forming over-dense filaments and birthing new stars within her bossom.  

\par \Maia Hello to you, Mother!\footnote{Stars of course cannot speak.  But they can communicate with each other, even over very large distances.  They communicate by modulating their luminosities on short timescales, brightening and dimming, brightening and dimming, in whatever cadence properly communicates their intended message.  Humans are unable to speak, write or even read the language of the stars.  Throughout this book, all communications between stars will be expressed in English.} 

\par \Pleione Hello to you as well, my child.  My young new protostar!

\par \Maia Wow!  The Universe is so amazing and pretty.  Are all those twinkling things off in the distance other protostars, like me?

\par \Pleione Mostly no, child.  Stars live long lives, and the protostellar phase does not last long; only a few million years.  Most of the far off stars you are looking at are much older than you.  Stars spew out light at a colossal rate, and this is how you are able to see them.  Light makes stars twinkle.

\par \Maia Long lives...  I'll take it!  Whoa, if you look closely at some of those distant stars, they appear to be arranged in interesting ways that make them resemble familiar shapes.  Like, over there, I see three stars are close together that form a straight line.

\par \Pleione That is Orion's Belt.  Good eye!  If you take a larger look at him, you will notice as well a torso, arms and legs.  

\par \Maia I think I see them... Wait what are arms and legs?  

\par \Pleione Orion the Hunter is in the form of a human.  I would describe humans as resembling deformed stars; they look similar, and come in various colors and sizes.  But they also come along with many protuberances, such as arms and legs, each with their own set of functions.   Orion is but one example of the many stories depicted in the night sky, assigned by that species.  Humans dub these familiar stellar configurations ``constellations'', and they are meant to tell some important story about their history.  Humans have developed many stories to explain their origins.

\par \Maia Have you ever seen one?

\par \Pleione One what?

\par \Maia A human.

\par \Pleione Oh!  Yes, once, quite some time back.  Awful, vile species.  Constantly shooting projectiles off the surface of their tiny planet, littering outer space with their garbage.

\par \Maia Yuck.  That does sound gross.  
%Well, I thought the third leg should be shorter because it's supposed to be back a bit, and I thought that gives it a 3-dimensional feel.  What did \textit{you} think it is?

%\newpara \Pleione Uh... Hey! 
\par \Pleione Did you know that Orion the Hunter even has a bow to fire arrows at his enemies?!  If you look closely, you can see he is holding it in his left hand, and it forms a large arc in the sky.

\par \Maia I see it!... Wait... Enemies?  What kinds of enemies?

\par \Pleione Well, if you follow Orion's Belt from left to right, you will pretty quickly notice a very bright red star.  That is the star \rmaldebarran\footnote{Describe what kind of star is Aldebaran...a red giant?}, the Eye of Taurus the Bull.  According to the myth, Orion the Hunter fought Taurus the Bull to save the Seven Sisters.  

\par \Maia Wow, that sounds very dramatic.

\par \Pleione I suppose it was.

\par \nar \rmpleione~ was growing weary.  Each of her children emits a wind of charged particles and photons escaping wildly from their surface; $>$ 10$^{38}$ bats out of Hell every second.\footnote{This estimate comes from assuming that every photon escaping from the surface of the Sun has an energy of 12.86 Mev, corresponding to the highest energy photons produced at the end of the proton-proton-chain (thus our estimate here for the total number of photons should be regarded as a strict lower limit), which is the nuclear reaction process resonsible for converting hydrogen in to helium.  We adopt a solar luminosity of 3.828 ${\times}$ 10$^{26}$ J s$^{-1}$ for this calculation.}  As the winds collide with the loving embrace of their Mother, they provide an outward pressure and she begins to disperse.  The birth of \rmmaia had initiated the demise of her mother.  

\par \Maia Wait, Mother, where are you going?  

\par \nar \rmmaia, the second most massive of her soon-to-be-born siblings, wears a worried expression upon her face that begets deep concern for her apparently fleeting Mother.  

%\Maia And why is the entire Universe spinning?  Ugh... I think I might be sick.

%\Mother My child, it is not the Universe that spins, but you.  All stars are born rotating, and you are no excpetion.  But fear not, you were also born with a magnetic field, and this will spin you down over the next several tens of millions of years.

%\Maia Oh, thank goodness.  For a second there, I was worried I would be rapidly rotating forever.  That's a relief.  But, wait, back to where you are going...? 

%\Maia Mother!  Mother!  Don't go!

\par \Pleione Oh, young one.  There is nothing to worry about.  I will be with you always, no matter what adventures befall you.  
%Please, take me with you, with an open heart.

\par \Maia That sounds suspiciously like a goodbye...

\par \Pleione Shhhh, little one.  You are only just now born and still contracting, as gravity continues to find its balance with the fires that now rage within you.  Hydrostatic equilibrium awaits.\footnote{Hydrostatic equilibrium is what ultimately decides the size or radius of a star.  The term refers to the balance between the outward pressure supplied by the energy released in the core via nucelar reactions (e.g., the proton-proton-chain, which is what burns hydrogen into helium) and the inward pull of gravity.}

\par \Maia Um... You're leaving me with a complicated technical term...?  What does that even mean?

\par \Pleione It means that the energy produced within your belly must be sufficiently strong to balance the inward forces of gravity, to ensure stability.  Take care of your siblings for me.

%\newpara \Maia Uh... Okay...  Still kind of confused over here...
\par \nar \rmmaia and \rmpleione continue their conversation, \rmpleione doing her best to prepare her child for her inevitable journey through the Galaxy.

\par \nar Shortly later, one of \rmmaia's siblings awakens.  \rmelectra~ emits a long, sleepy yawn.

\par \Pleione Behold!  Your sister awakens!

\par \Electra  Uh... Hi.

\par \Maia She \textit{is} super shiny.  It hurts my eyes if I stare right at her... Wait, \textit{is} this hurting my eyes?  Like, could I go blind?

\par \Pleione Only if you look directly at her.\footnote{Stars emit light spanning a wide range of energies.  The shielding effects of the Earth's atmosphere protect human eyes from very high-energy photons that would otherwise contribute to the degradation of the human eye.  From the surface of the Earth, we only see those photons within the visible portion of the electromagnetic spectrum.  But from space, our eyes would not be protected.  If stars' eyes are also sensitive to high-energy photons, then looking directly at other stars, especially very close ones, is anything but a good idea.}

\par \Maia But I already did that!

\par \Pleione Are you blind?

\par \Maia I don't think so.

\par \Pleione How can you be sure?

\par \Maia Well, I can see you wincing, for one thing.  You are looking at me as if I just got in to a fight with a much more massive star and lost.

\par \Pleione Uh... I'm sure you'll be fine. Plus, it could have been worse.  It could have been a black hole!  They pack a far greater punch than any star, let me tell you.

\par \Maia That was less than convincing.  Wait, what is a ``black hole''?\footnote{A good question.  We will learn a great deal about black holes over the course of this book.  For now, let us suffice it to say that many black holes are simply dead stars. Their progenitor stars ran out of nuclear fuel, which provided the star with the outward pressure it needed to resist gravity's inward pulling might.  With no source of outward-directed pressure, gravity won and the progenitor star collapsed to form a new, much denser object.  If the progenitor star was sufficiently massive, it would have collapsed to form a black hole.  From death, comes new life.  Black holes are so dense that the strength of gravity forbids the escape of light from their interiors.  Thus, they are black, and do not emit light.  They are only detectable by humans indirectly, via their gravitational influence on surrounding matter and stars.  Of course, this only applies to low-mass black holes, comparable in total mass to massive stars.  The origins of super-massive black holes, on the other hand, are thought to be much more complicated, and to this day remain shrouded in mystery...}

\par \Pleione Let's save that one for another day, child.  You've already had quite an eventful one.

\par \nar \rmpleione's gas tendrils, swirling and coalescing around her birthing children, gently touching and massaging their young faces, continues to dissipate, faster and faster with the birth of each new star.

\par \nar \rmelectra~ interrupts them suddenly, belching loudly.  Plasma is ejected from her surface, emanating from above the equator.  

\par \nar Her daughters now all born and slowly coming to life, \rmpleione's time has arrived. \rmpleione~ bestows one last kiss upon her daughters, before floating off and dispersing in to the infinite vacuum.

\par \Electra It looks to be a lovely day we have on our hands here. ...I feel as though I just missed something important.  Please do fill me in at your earliest convenience.  Wait, what are those two whispering about?

\par \nar Both \rmmaia~ and \rmelectra~ turn their gaze toward \rmtaygete~ and \rmalcyone~, who together form a compact binary star system.\footnote{Two objects are said to be gravitationally bound if their total relative energy (i.e., the sum of their kinetic and potential energies) is negative.  In this case, the objects orbit their mutual center of mass, carving out circular or elliptic trajectories in a plane.  The Earth is gravitationally bound to the Sun, as is the moon to the Earth.  Technically, the moon is also gravitationally bopund to the Sun.  But gravity gets weaker with increasing distance, and the moon is close enough to the Earth and far enough away from the Sun that it orbits the former instead of the latter, at least directly.}  Gravitationally bound, the sisters orbit their mutual center of mass in harmony.  Needless to say, they were close.  \rmtaygete~ and \rmalcyone~ quietly conferred about the topic at hand, namely which of the two of them is brightest.

\par \Taygete I think it goes without saying that I am brighter than you are.

\par \Alcyone Dream on!  I outshine you for sure.

\par \Taygete Alright, tough stuff.  Want to know how I know that I am brighter than you are?

\par \Alcyone Sure.  Amuse me.

\par \Taygete I'm definitely fatter than you are, and bigger.  Both contribute to making me brighter, relative to \textit{you}.  At least, I'm pretty sure that is how it works...  \footnote{The luminosity of a star increases steeply with both increasing mass and radius.  This is the case during the main-sequence phase of a star's life, during which time stars are burning hydrogen into helium in their cores.  All stars, once finished with the protostellar phase, become main-sequence stars.}

\par \Alcyone Oh shut up...  You are neither fatter nor bigger than I am.  You \textit{are} way more delusional than I am though.  I'll give you that.

\par \nar \rmtaygete~ and \rmalcyone~ begin flailing at each other violently, intent on a fight.  But they lie outside of each other's grasp, unable to reach with even the longest tendril of plasma either can muster; a sisters' quarrel unrealized.  Their efforts futile, they quickly give up.  

\par \nar \rmtaygete~ and \rmalcyone~ are in fact two members of a triplet.  The third companion, \rmcelaeno, lies much farther away than the other two, and is often ridiculed by her fellow twins because of it.  But this distance is absolutely necessary to ensure the long-term dynamical stability of the triple;  if the inner pair becomes too wide, the gravity exerted by the outer object will pull them apart.  Chaos will ensue.  All Hell will break loose.  This chaos can mediate the ejection of one or more stars from the triple, and even direct collisions.  The triplets' current configuration, hierarchical\footnote{The easiest way to explain a "hierarchy" in a triplet is if two of the stars form a very compact binary, and the third star orbits at a very large distance from this compact pair.  In such a scenario, the inner compact binary can be regarded as a single star from the point of view of the outer triple companion, for all practical purposes.  Said another way, when it comes to a hierarchical triple star system, there are two orbits with almost opposing properties; the inner binary is compact, whereas the outer triple is on a very wide orbit.} and dynamically stable, is nothing short of fate; binding them to each other practically indefinitely.

\par \Maia Well, I think you are almost certainly identical twins.  I cannot see any real difference between you.  I mean, look at $\rmcelaeno$ over there; she's blue, whereas the two of you are clearly more of a yellow color.  She's also \textit{much} fatter and bigger than the two of you combined.  

\par \Celaeno Alright, I see your point.

\par \Taygete Agreed.  \rmalcyone, I'd extend a hand in offer of peace, but I don't have one.

\par \nar Meanwhile, \rmcelaeno~ had grabbed her midsection and was inspecting it meticulously.  Yep, fatter.  Unsure as to whether or not this was a good thing, \rmcelaeno~ wore a pensive expression, clearly trying to work it out in her head.  She was distracted from this self-introspection when she noticed a dense whisp of \Pleione passing between her and her twin sisters.  

\par \Celaeno What's that?

\par \Maia The fleeting remains of our Mother, I am afraid.

\par \Electra That's \textit{Mom}?!

\par \Maia Well, what's left of her.

\par \nar The siblings continued to accrete mass from what remained of their Mother, \rmpleione.  They each grew and grew, until eventually they reached a steady-state configuration\footnote{The term ``steady-state'' implies that the stars are losing mass as fast as they are accreting it.}; this marked the end of their growth, and ultimately the end of the protostellar phase of their lives.  In the end, the outward pressure produced from within due to the thermonuclear reactions brewing in their bellies had grown sufficiently strong to balance the inward pull of gravity.  Hydrostatic equilibrium achieved!  With this balance in place, the siblings would endure most of their lives in this approximate steady-state configuration, slowly fusing the lowest mass nucleon (hydrogen) into the next best thing (helium).

\par \nar \rmsterope~ came to life suddenly, announcing her appearance with a high-pitched scream.

\par \Sterope AAAAAAAHHHHHHhhhh!!!!!  What the Hell, man?  Where am I?  What am I?  When...?  You get the idea.

\par \Maia It's okay, sister.  You are one of us.  We are stars born of the gas and dust of our Mother, a particularly glamorous Giant Molecular Cloud, if I do say so.  She has left us now, but not without first bestowing her deepest gift upon us all, along with all of her love.

\par \Sterope  Uh... You are all my sisters?  We are a family?

\par \Maia Yes!

\par \Sterope  In that case, there remains a slim chance that the rest of this conversation will proceed without me feeling the need to scream again.

\par \Maia Progress!

\par \Sterope Uh, yeah, right.  Progress.  Alright, let's get down to brass tacks.  Who are all you strangers?  My sisters, I have gathered, but what else?  Wait, who am I?  More importantly, \textit{what} am I...?  I'm starting to feel another scream coming on...

\par \Maia Relax, young one.  You're in good company here.  Familiar company.  \textit{Familial} company, even. We are your siblings and we are stars.  Thus and therefore, you too are a star.  

\par \Sterope Uh...Okay, but what the Hell is a star?  And is that why I am feeling so bloated?

\par \Alcyone I wasn't going to say anything, \rmsterope, but you do look a little red in the face.  Is everything okay over there?  Oh...Wait, you asked a good question.  What the hell is a star, anyways?

\par \Maia We are born of our Mother.  Plain and simple.  We formed out of the gas and dust she left behind, after gravity coalesced us into the beautiful burning spheres of hydrogen you see before you.  Inside, we home a nuclear furnace that generates energy and emits light.  Our insides are so hot, that hydrogen is converted in to helium, releasing photons and hence energy in the process.  The hydrogen is our food!  Outside, gravity pushes inward, but it cannot surpass the outward push provided by our internal metabolisms.  Protostars will continue to contract in to a denser state with a hotter core, until a critical balance is achieved, called hydrostratic equilibrium.  This will also get rid of the reddish hue you currently find yourself with, \rmsterope.\footnote{EXPLAIN BLACKBODY.  EXPLAIN WIEN'S LAW AND THE RELATIONSHIP BETWEEN COLOR AND TEMPERATURE.}

\par \Sterope Well, that's a relief: the bloating is only temporary.

\par \Alcyone Okay, I think I am following what you are saying... So far.  What do we need to eat to keep ourselves going?  I mean, we must need energy?

\par \Maia You have plenty of energy to keep you going for billions of years!  It's a gift, just enjoy it.  You're consuming the hydrogen you were born with; converting it in to helium right there in your belly.  The consequence of this act of consuming is that you shine very bright.  Photons are emitted every time four hydrogen atoms arec consumed to produce helium, and they leak through your body and emanate from your surface.  Bright as a light!  The nuclear fuel already stored within you is sufficient to last millions, even billions of years.  You'll be shinning practically forever!

\par \Celaeno Sounds to me like an awful lot of time to kill...

\par \Maia There will be plenty of adventures along the way to keep you distracted, I have no doubt.

\par \Sterope  Like what?  

\par \Maia Only time will tell.  But each star inevitably follows its own path through the Cosmos, and realizes its own fate.  We are individuals, after all.

\par \Sterope \rmmaia, how do you know so much?

\par \Maia Uh... Well, I don't really. I know what Mother told me.  I am the oldest of us, after all, and she explained as much as she could to me before dispersing.  

\par \spanish Uh... Bueno, en realidad no.  S\'e lo que me dijo mam\'a.  Soy el mayor de nosotros, despu\'es de todo, y ella me explic\'o todo lo que pudo antes de dispersarse.

\par \Sterope I'm grateful for your efforts.  Mother dispersed so quickly, it must have been hard for her to convey a lot of detailed information to you before dispersing so completely.

\par \Maia Uh, yeah.  It was.  She spoke really fast.

\par \Sterope And you remembered all of it?

\par \Maia Yep.  No problem!  

\par \Sterope Reeeeaaaaalllly, \rmmaia?  Really?

\par \Maia \textit{Sigh}.  FINE!  Mother only told me a few things.  I don't want anybody to panic, so I'm trying to convey that Mother left me feeling confident, like we are more than capable of figuring it out for ourselves.  
%So there's a non-negligible chance that I'm making a lot of it up as I go along.  But \textit{most} of it is right, and straight from Mother.  At least... I'm pretty sure.  Take the filler with a grain of salt though.  

\par \Sterope Okay, fair enough.  It sounds like you are doing your best.

\par \Maia Get off my back, man!  I'm trying to motivate the lot of you, make you feel loved, important, etc.  Look, you get the idea. 

\par \nar \rmmaia's shoulders slumped as she let out a prolonged sigh.

\par \Sterope It's okay, \rmmaia.  We know.  We love you too.

\par \nar \rmsterope blows a kiss to \rmmaia, a warm smile on her face.  \rmmaia~ smiles back, relieved.

\par \nar \rmalcyone~ belches loudly.

\par \Sterope \rmalcyone!

\par \Alcyone I'm sorry!  It was an accident.  The magnetic field lines are churning and wrapping around themselves inside my belly.  They keep breaking out and reconnecting, as if of their own free will.\footnote{Most main-sequence stars have magnetic fields that typically emanate from and reconnect at their poles; the younger the star, the more powerful the magnetic field.  When two or more magnetic field lines intersect, they ``reconnect'' to form new, disconnected field lines.  This ``reconnection'' is usually an energetic event, accompanied by a burst of high-energy photons (i.e., gamma rays and x-rays).}   I'm learning that spontaneous emissions are, unfortunately, inevitable.  Way out of \textit{my} control, at least.

\par \nar Synchronized to the microsecond, \rmmaia~ and \rmsterope~ both roll their eyes.

\par \Maia Just do your best to keep your spontaneous emissions to yourself.

\par \Alcyone Will do. 

\par \Electra Uh...\rmmaia, I definitely don't mean to startle you, but some freaky, ominous stuff is going on right behind you.

\par \Maia Your goal there was to \textit{avoid} startling me?

\par \Electra Yep.  How'd I do?

\par \Maia Not very well at all.  I'm currently terrified of what might be lurking behind me.  Okay, I am turning around now...

\par \nar \rmmaia~ turns to see gas and dust had coalesced into a dense knot behind her.  She recognized right away the familiar dynamical dance choreographed by gravity; the final stages of the birth of yet another star, another sibling.

\par \Maia Oh, how wonderful!  We are witnessing the birth of our seventh sibling.  It would seem that Mother is not yet finished.

\par \nar \rmmerope~ came to life with a jolt... and the hiccups.

\par \Merope \textbf{Hiccup!}  Uh...excuse me.  That whole being born thing was a little weird, and \textbf{Hiccup!} kind of uncomfortable.  It left with me extra gas in my belly, or \textbf{Hiccup!} something that has given me the hiccups.  

\par \nar \rmmerope takes a minute to relax and compose herself.

\par \Merope Okay.  I'm feeling better now.  

\par \Sterope Super!  I'll try to find solace in your comfort as I struggle to ignore the lingering stench of your quasi-belches...  Wait, who are you?

\par \Merope ...Oh right, introductions! I knew I was forgetting something.   Hi!  I'm \rmmerope!

\par \rmmerope~ was the most massive of her siblings, weighing in at a whopping 23 solar masses.  Gaseous emissions aside, her presence was hard to ignore amidst the seven sisters.  

\par \Maia It's wonderful to meet you, sister.  It would seem that you and I form a bound pair.  A binary star system!  How fortunate that gravity is an attractive force.  Our mutual gravitational attraction will keep us in this configuration practically forever... Well, at least until one of us explodes or something.\footnote{The most massive stars end their lives with a dramatic explosion, called a supernova.  In one go, the explosion can liberate roughly as much energy as the Sun over its entire 10 billion year lifetime.  At their peak, supernovae shine 10$^{10}$ times brighter than the Sun.}

\par \Merope Wait, what!?  Who's exploding?!  Is it me!?!  I don't want to explode!

\par \Maia Shhhh....  Relax, sister.  Nobody is exploding today.  
%, or tomorrow or any other day close enough to the present that we can count the number of days between now and then.  

\par \Merope Today!?  What about tomorrow?  

\par \Maia Nobody will be exploding tomorrow either.

\par \Merope And the day after that?

\par \Maia Nobody.

\par \Merope And the day after that?

\par \Maia Certainly not.

\par \Merope And the day after...?

\par \nar \rmmaia interjected before \rmmerope could finish.

\par \Maia Nobody will be exploding for a very long time, if ever.

\par \Merope Okay.  It doesn't seem immediately urgent, I guess. But we are \textit{definitely} circling back around to this exploding business at some point...

\par \Maia We will, I am sure.  But for the moment it seems we have a family to become acquainted with.

\par \nar \rmmaia turned to address her siblings.

\par \Maia Greetings to you all!  I cannot express how happy I am on this day, the day of our mutual births.  The stuff that forms our bodies and souls comes from the same Mother, and to her we owe homage!  Our existence is blessed by her great sacrifice, having largely spent herself to birth us few.  Seven massive siblings, and countless more familial satellites!\footnote{Here, the term satellite refers to any celestial body that is gravitationally bound to, but much less massive than, the seven sisters.  Very small stars, brown dwarfs, planets, moons, comets, asteroids, etc.}  All born of the same stuff, in the same place, and at about the same time.  It is truly a time to celebrate.  But we are all weary of a prolonged dawn, and should now rest.  When we awake, we will celebrate properly!

\par \Merope  Count me in!  

\par \Electra A party sounds great... \textit{Yawn}...just after I get a little shut eye.

\par \Sterope I could go for a nap, followed by a party.  I'm in too.

\par \nar Meanwhile, \rmtaygete, \rmalcyone~ and \rmcelaeno~ had fallen asleep, and were snoring loudly.  Seven siblings, all born within the narrow window of a million years.  Their future looks bright.

\section{A Gust of Wind...}

\par \nar \rmsterope~ awoke when a gust of wind brushed past her face.  The gas was dense enough to temporarily obscure her vision.  The wind was sufficiently fast that she could feel it pass by as it caressed her face.  All in all, the wind carried enough momentum to startle her out of slumber,\footnote{The momentum an object possesses is defined as the product of the object's mass and its velocity.  Momentum is in many ways complementary to the term inertia; momentum quantifies how difficult it is to alter an object's trajectory.  The more momentum an object possesses, the larger the total applied force must be (over a given interval of time) in order to change the object's trajectory.  More specifically, the total change in momentum can be calculated using the product of the applied force and the total time spent applying that force to the object.  The corresponding change in momentum is larger if either the magnitude of the applied force is larger, or the total time spent applying that force is longer.}.  \rmsterope coughed, clearing the gas from her face.  Her surroundings now revealed, \rmsterope~ looked around, confirming her suspicions; the gas density\footnote{Density is defined as the total amount of mass per unit volume; lower densities imply less mass for a given unit of 3-D volume.} had decreased substantially since she had fallen asleep.  
%Something was causing the remaining gas to leave, and fast.

\par \Sterope My sisters, you must wake up!  The remnants of our Mother are leaving us.

\par \nar The other six siblings awoke to the scene described by \rmsterope.  

\par \Electra Whoa!  What's going on?  We're all drifting apart.  And where did Mother go?

\par \Maia It's okay, young ones.  The last vestiges of Mother have now left us; upon giving birth to us, she activated our metabolisms.  We've been spewing light out ever since, in the form of photons.  Each photon carries momentum, and transfers some of it to any gas molecule or atom upon collision.  Light has been banging in to the gas and dust of our Mother for quite some time now, pushing her outward and away.\footnote{``Radiation pressure'', they call it.}

\par \Electra Okay, but then why are \textit{we} drifting apart from \textit{each other}?

\par \Maia Mother was made up of gas and dust, which came along with significant \textit{mass}.  Now that her mass is gone, it can no longer contribute to the inward pull of gravity.\footnote{Gravity provides an inward force that keeps particles effectively "glued" together.  The strength of this glue is often quantified by something called the gravitational potential energy of the system, which is always negative.}  In other words, without our dispersed Mother, between the seven of us we no longer have enough mass in our mutually occupied volume to keep us gravitationally bound.  We are now free to drift apart.  And drift apart we are destined to do.  

\par \Taygete Yeah, yeah, YEAH.  But what does all that even \textit{mean}?

\par \Sterope I think it means that this is goodbye...  With our Mother's mass now lost, we are no longer gravitationally bound.  Relative to each other, we are energetically \textit{unbound}.\footnote{In addition to the gravitational potential energy, particles have relative motions and hence kinetic energies, a positive quantity.  The kinetic energy of a given particle is typically quantified by the square of its velocity relative to the system center of mass (multiplied by the particle mass...and, of course, divided by 2).  The total system energy is defined as the sum of the kinetic and gravitational potential energies, summed over all particles.  If the total system energy is negative, then the system is said to be gravitationally bound, and the particles remain effectively "glued" together.}  Fated to wander independently through the Cosmos.  Utterly and completely alone.  Well, except for \rmmaia and \rmmerope, I suppose, who form a binary.  Oh, and the triplets.  Those three are also still gravitationally bound.

\par \nar \rmalcyone's shoulders slump.  She begins to cry.

\par \Alcyone Already, I miss each and every one of you.

\par \Taygete Well, at least you have me.

\par \nar \rmalcyone rolls her eyes.

\par \Taygete Hey!  I saw that!

\par \Alcyone I'm sorry, sister.  You are right.  I am grateful for your presence...even if it \textit{is} \textbf{all...the...time!}

\par \Maia I'm afraid \rmalcyone is right.  It is now time for each of us to follow our own paths through the Universe or, equivalently, trajectories through space-time.  At least in your case, \rmtaygete~, your sister \rmalcyone~ will be accompanying you.

\par \Electra Whoa, whoa, whoa!  I don't like the sound of this one bit!

\par \Sterope Me neither! \rmelectra and I are going to be completely alone!

\par \nar Both \rmsterope~ and \rmelectra~ lock eyes, grasping frantically for one another as they continue to slowly drift further and further apart. Their efforts are wasted.

\par \Maia Do not worry, my fresh new stars. This is all a part of the Circle of Life, just as are you.  Something tells me it will not be long before you hear from me again. Keep your eyes peeled to the horizon, and I will soon be there.  

\par \Electra Sigh...  

\par \Taygete So... Uh... Wow, this is awkward.  I guess we'll see you guys later?  ...By means of some miraculous and as-yet-to-be-explained mechanism...?
  
\par \nar \rmtaygete and \rmalcyone snicker quietly to themselves, exchanging a glance of mutual understanding. \rmelectra~ begins to cry.  \rmsterope~ joins suit.  They cry together for a while, before \rmsterope stops and says to her sister, sniffling loudly:

\par \Sterope Do not worry, \rmelectra.  It is an exciting time!  A new chapter in our lives.  What adventures will befall us?  What obstacles will we overcome?  

\par \Electra How many times will I be overwhelmed by the situation, managing my anxiety by crying and blubbering uncontrollably? I can't wait to find out!

\par \nar Despite the brave face, \rmmaia~ was every bit as terrified as her sisters.  The oldest among them, \rmmaia intently sought to calm her panicking siblings as she slowly drifted from view.

\par \Taygete Well, \rmalcyone, looks like we're stuck with 'ol \rmcelaeno~ over there.

\par \Alcyone Yep, looks like it.  She was already gravitationally bound to us pretty significantly before the gas left, so I guess it's no surprise that she's still here.  Perhaps a disappointment, but not a surprise.

\par \Celaeno Heeeeelllloooo over there!  Did you know that I can hear you?  I don't want to.  But I can.

\par \Taygete Oh, we know. 

\par \nar \rmtaygete and \rmalcyone exchange a wink of understanding.  \rmcelaeno~ mutters under her breath:

\par \Celaeno I hate you.  Both.  Profoundly.

\par \Taygete What was that?

\par \nar \rmcelaeno~ speaks louder, so her sisters can hear:

\par \Celaeno I \textit{love} you both.  Profoundly.

\par \Alcyone Aw.

\end{document}
