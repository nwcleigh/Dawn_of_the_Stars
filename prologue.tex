\documentclass[main.tex]{subfiles}
%\usepackage[T1]{fontenc}
%\usepackage{ae,aecompl}


%%%%% AUTHORS - PLACE YOUR OWN PACKAGES HERE %%%%%

% Only include extra packages if you really need them. Common packages are:
%\usepackage{graphicx}	% Including figure files
%\usepackage{amsmath}	% Advanced maths commands
%\usepackage{amssymb}	% Extra maths symbols
%\usepackage{lipsum}
%\usepackage{lmodern}
%\usepackage{tcolorbox}

%\newcommand\levelone[1]{{\color{red}\bf}}
%\newcommand\leveltwo[1]{{\color{blue}\bf}}
%\newcommand\levelthree[1]{{\color{green}\bf}}
\begin{document}
\newpage
\addcontentsline{toc}{chapter}{Guidelines}
\section*{Guidelines} \label{guide}

\par \nar Hello!  Welcome to {\it Dawn of the Stars}, the story of the Seven Sisters.  Maia, Merope, Taygete, Alcyone, Celaeno, Sterope and Elektra are sibling stars, all born together at about the same place and time.  They are seven members of a much larger group of around 100 stars, forming a star cluster called the Pleiades.  In this book, we will follow the story of each of the Seven Sisters, beginning at their birth and following them as they each leave the Pleiades, embarking on a perilous journey to explore the Milky Way and, one day, hopefully find her sisters again.  Each of the Seven Sisters experiences a unique journey, meeting many interesting characters along the way.  As you read on, you will join them as they discover the Galaxy, exploring its many wonders.  You will learn about stars and the physics that decides their properties, answering questions like:  If stars are born, can they die too?  If so, how long do they live?  Why do stars have the colors that they do?  Why are there no green or purple stars?  Why do the blue stars tend to be bigger than the red stars?  

\par \nar Before we begin, let us first consider a few features in this book implemented to help communicate its contents to the reader.  We have gone to great care to clearly separate the story line from the educational content, so that the reader can choose for themselves when they are interested in reading more about a particular topic when it is presented in the story line, or when they would prefer to ignore the additional material and continue reading on about the adventures of the Seven Sisters.  Below, we describe the "rules" defined and implemented throughout the book to ensure that this goal is met.

\par \nar The first thing you will notice is that the text and story line have been segmented into a “box” structure.  This is used to define and communicate different levels of education needed as a minimum by the reader to understand certain key concepts contained within each “level”.  For example, level zero is shown in black text, and is not confined to any such boxes.  This “zeroth level” contains the underlying story line following our main characters, and can be understood by readers having a base level of education attained in grade school (i.e., up to and including high school).  Our focus here is to keep the language accessible to readers with a high school education in the sciences, and to focus on visualizing the story line as it unfolds.  Level 0 does not require any knowledge of any text appearing in any other “levels” in order to understand its contents.  If the reader chooses to read only the black text, then they will be able to follow along with the story but will neglect access to some of the higher-level education materials presented in this book.  

\par \nar Subsequent “higher” levels require more education to understand their contents.  All of the corresponding “higher level” education materials are contained within, and isolated to, the box format described above.  Level one is indicated by red boxes and lettering, and requires a basic undergraduate STEM-based education to understand the content.  Level 1 contains no equations, and is focused on explaining the important physical concepts that appear recurrently throughout the book.  All text appearing in red boxes assumes that the reader is following along with all text appearing in both red and black fonts.  Level two is indicated via blue boxes, and requires an education in undergraduate physics to comprehend.  This where equations will begin to be introduced to further explain the important physical concepts.  We keep it simple, by avoiding complex derivations.  Instead we focus on order-of-magnitude calculations.  All text appearing in blue boxes assumes that the reader is following along with all text appearing in blue, red and black boxes.  Level three is indicated via green boxes and lettering, and is aimed at the graduate level.  This is the expert level, requiring a graduate level education in physics.  There is no ceiling to the complexity of the content, and this is where derivations and even snippets of programming code will be introduced, in case the reader wants to try running some of their own simulations of the relevant $N$-body dynamics appearing throughout the book.  %And so on.
%Do we want to include any higher levels?  

\par \nar In summary, each level of text, indicated by a different type of box and lettering, and requiring the indicated background education, requires knowledge of the content of each level below it, but no knowledge of any level above it.  In principle, the reader can begin to appreciate the book from a very young age and, if they so choose, return to it again and again over the course of their life as their own personal education develops.  Each time the reader re-discovers this book, we hope they will discover new things, with new content emerging as it becomes accessible to them.

\par \nar Now, we know what you are thinking:  "How on Earth am I supposed to safely navigate my way through Level 0, when I'll be dodging around or even fumbling right through Level 1, or Level 2 or, even worse, Level 3!  This sounds sort of like playing a video game that is impossible to win...." 

\par \nar Rest assured, we have taken great care to ensure for every reader a safe, secure and pleasant journey through each and every chapter, independent of their level of education.  This means that ignoring any higher level content by reading only Level 0 text is straight-forward, since any obstacles introduced by higher levels are clearly identifiable and isolated from the rest of the story, such that they are easy to avoid.  
%This is accomplished by clearly segmenting off any text associated with higher levels that require a more specialized knowledge to access or understand.  To this end, 
To summarize, the key point to keep in mind is that we use colored text to indicate to the reader the level associated with that text, and boxes to indicate that the text should only be regarded by readers at the appropriate level, and ignored otherwise.  So if you only read black text, you can safely ignore any higher level material without having to worry about missing important developments in the story line, since no additional aspects of the story are introduced in higher level text.
%For example, if you are only reading Level 0, then you can safely ignore any and all colored text and boxes, without any details being lost from the principle story line following our main characters.  
If you are still feeling reluctant, try to keep in mind that acquiring a paper cut from turning the pages is probably the worst thing that could happen to you upon reading our book.  Of course, this assumes you are not carrying any explosive devices on your person. In that case, exploding is probably the worst thing that could happen to you.

%PYRAMID DIAGRAM SHOWING THE COLORS AND CORRESPONDING EDUCATION LEVELS?  Obviously, the highest level of education goes at the top, indicating all levels below it are required to follow the story development in these sections.  The base of the pyramid should be in black, given the above descriptions.

\par \nar The second thing you will notice is that the characters in the story appear in corresponding illustrations.  These are included not just for entertainment purposes, but are instead primarily used to help convey the underlying story and certain key concepts.  We do our best to ensure that the illustrations are accurate and precise in terms of the underlying physics, to help communicate important concepts typically related to the physics governing the birth and subsequent evolution of stars.  For example, the sizes and colors of the stars in the story are accurate given the masses of the stars as constrained from observational data, combined with our knowledge of stellar evolution and how the properties of stars change over time.

\par \nar The third rule pertains to astrophysical accuracy throughout the story line.  For example, it can often occur that the travel times for stars to venture from Point A to Point B in the Milky Way galaxy can drastically exceed the stellar evolution timescales.  In other words, it can often occur that stars will run out of nuclear fuel and collapse to form compact objects, such as white dwarfs, neutron stars and black holes, on a timescale shorter than the indicated travel time.  We ignore this effect at times, so as to avoid our characters changing from stars to compact objects within only the first few chapters.  To compensate for this, we include Level 1 boxes to explain these competing timescales, and to make it clear how adhering strictly to these competing timescales would affect the story line and the underlying character development (see below).

\par \nar The fourth rule is that we stick to the characters defined in the Greek mythology of the Seven Sisters, but consistently inform the reader of what is actually now known about their properties from centuries of observing those stars living in the Pleiades cluster.  This is more complicated than it sounds, due to a long history related to constellation creation and the development of telescopes with superior resolution abilities.  With the naked eye, the Pleiades cluster is visible very close to the Orion and Taurus constellations.  However, the naked eye can only resolve a small subset of the total stars in the Pleiades cluster.  Specifically, we can only see the most massive and hence brightest objects in the cluster.  But, as telescope technology improved over the years, it quickly became apparent that many of the Seven Sisters, as originally named by the Greek mythology, are actually previously unresolved multiple star systems.  So, in a few cases, a given Seven Sister is actually a collection of gravitationally bound stars.  To adapt to this complication, we stick to the original naming conventions as decided by the Greek mythology, and include Level 1 boxes after each character is introduced, to update the reader on the current state of the observations for each of our Seven Sisters.

\par \nar Last but not least, we assume that when our stars merge or evolve to become compact objects (i.e., white dwarfs, neutron stars or black holes), they become new characters with a new personality.  In the case of compact objects, the tendency is for the resulting character to be more evil relative to the original Seven Sister.  However, we go to great efforts to impose a unique personality to each character, which we clearly define and adhere to when a given character is communicating with other stars.  For example, black holes often consume stars whole, if the stars venture too close to them.  Some black holes in our story are committed to enticing stars to come sufficiently close for consumption, whereas other black holes are more conflicted by their need to destroy other stars in order to "eat" and grow in mass.

\par \nar In closing, we sincerely hope you enjoy reading our book.  The rules outlined here will be repeated continuously throughout the book, primarily at Level 0.  The rules should become more familiar to you as you read on, deeper and deeper into the unfolding story line and the individual adventures that befall our cast of characters.  We hope you enjoy following the development of our characters as they learn more and more about our home, the Milky Way Galaxy, and the Universe around them, embracing the learning process as you read on.  We hope you are able to take away more knowledge than we gained upon writing the book.  Of course, this is no easy task, since we learned a lot!  But we have gone to great lengths to ensure that this end goal is reached for the vast majority of our readers.  See you at the end of the book!

%\textbf{NL:  I tried to define a few "rules" above, which would also extend to the color-coding of the text in footnotes, or on a different page or in the margin or whatever.  I propose we try to define these rules, and decide on the corresponding education levels, before we even begin working on the book.  Thoughts?  I also tried to include stuff where Pleione teaches Maia how to count, add, subtract, etc.  My hope is that this introduces the key concepts of "mathematics and numbers" and "computing", and will help you guys having fun doing whatever the hell you want with this and/or subsequent chapters.  With all of us operating under the same over-arching rules, I think it should be fine to do things like:  contribute at whatever level you want (and even request of me trying to integrate the underlying concepts into the story, from the bottom up, and such things.  I mean, I hope this will turn into "our book" instead of "my project" pretty quickly, but please keep honest.  That is my intention, anyways, or at least to be as sincere as possible when asking if you guys wanted to work on this.  Okay, crap, this got "sappy"...  Also, I still need to go through the rest of this, including this chapter, in more detail to correct for these ideas, if we choose to keep them.  But I figured we should all decide on the rules before doing that, maybe assigning chapters.  For example, chapter 2, I think it is, might work well for Adrian since it is focused on the triplets.  There is also cluster dynamics in globulars in one chapter, which could go in lots of directions.  Ditto when the black holes appear in the story line.  There is also the possibility of including planets around one or some of the stars, and trying to integrate those characters into things.  And so on.  Just brain storming a bit, or trying to.}
%\comment
%Chapter 1 (star formation, basic stellar physics related to concepts like hydrodynamic equilibrium, introducing the main concepts we wish to build on and come back to over the course of the book, such as mathematics and computation, if we want to stick with those overarching themes)
%Main author:  Nathan

%In this chapter, we introduce the main characters appearing throughout the book, namely the Seven Sisters.  A giant molecular cloud %named Pleione births seven stars, namely Maia, Merope, Sterope, Taygete, Celaeno, Elektra and Alcyone.  Maia and Merope form a %binary pair, and are the two most massive of the seven siblings (Merope > Maia).  Taygete and Alcyone form a compact binary, with %Celaeno orbiting the pair as an outer tertiary companion.  Elektra and Sterope are two isolated single stars.  Merope (= 23 solar masses) %> Maia > Celaeno > Sterope > Taygete = Alcyone > Elektra (DOUBLE CHECK THIS!).  

%This chapter begins with the births of the seven sisters, and their mother Pleione instructing them on the basics of communication, counting and numbers, and the development of basic arithmetic.  The chapter ends with the dissociation of the newly formed cluster, once the gas has been dispersed due to radiation pressure, lowering the gravitational potential in the process and causing the natal stellar cluster to expand.  It expands to the point that the seven sisters become gravitationally unbound, drifting apart and hence beginning their individual journeys throughout the Galaxy.

%In this chapter, we must decide and describe how the stars will communicate.  As it is, we are assuming that they will use the EM spectrum to communicate, modulating their luminosities temporally to effectively communicate in an analogous way to using morse code.  BUT, this immediately presents the challenge of having to worry about delays in said communication due to the light travel time becoming exceedingly long.  Of course, I would view this as an opportunity to explain certain key concepts while maintaining an interesting and/or useful story on which we can draw out the physical concepts we want to get into.
%	We also introduce and explain the concepts of numbering systems, counting and basic arithmetic (i.e., computation).  This is needed so that the main characters will be able to perform calculations later in the book. 
%\comment


%PROPERTIES OF STARS ARE GIVEN BELOW.
%In this chapter, we begin with our introduction of the main characters appearing throughout the book, namely the Seven Sisters.  A giant molecular cloud named Pleione births seven stars, namely Maia, Merope, Sterope, Taygete, Celaeno, Elektra and Alcyone.  Maia and Merope form a binary pair, and are the two most massive of the seven siblings (Merope > Maia).  Taygete and Alcyone form a compact binary, with Celaeno orbiting the pair as an outer tertiary companion.  Elektra and Sterope are two isolated single stars.  Merope (= 23 solar masses; the true mass seems to be about 4.5 M_sun with a radius of 5.1 R_sun, from Wikipedia, also called 23 Tau) > Maia (= 5 M_sun, 6 R_sun, spectral type B8III, 850 L_sun, also called 20 Tau) > Celaeno (4 M_sun, 4.4 R_sun, 344 L_sun, also called 16 Tau) > Sterope (2.93 M_sun, B8V star; REF?) > Taygete (2.1 M_sun; Taygete is actually a binary star system or Tau Aa and Ab, with true masses of about 4.5 M_sun for Aa and 3.2 M_sun for Ab) = Alcyone (2.1 M_sun; Alcyone is actually a triple star system...it seems the brightest star is a B-type giant in the inner binary with a very low-mass companion and an orbital period of about 4 days, whereas the outer companion is roughly half the mass of the B-type giant with an orbital period of about 830 days) > Electra (0.4 M_sun; also called 17 Tau I do not see a mass...i.e., Wikipedia failed me). NOTE:  Only Sterope has the technically correct mass, thanks to Maxwell, and the rest I made up.  We also only mention Merope’s mass in the story line, so this needs to be edited.  I guess we should dig into the correct masses a bit better, and make sure we check for consistency with the colors and sizes of the stars in the illustrations.  
%Adrian: maybe be more specific and specify the masses of all the stars?
%Nathan:  Agreed.  This is on the todo list!  I did make up the rest of the masses now, except for Sterope’s mass.  Still wondering if we should just look up what is actually known about these stars in the literature, and then maybe edit components of the story line accordingly.
%Nathan: Note:  We need to edit this section, to make sure we birth planets around at least some of the stars.  How many stars should have planets?  How many planets should each star have?  This is another place I think it would be cool to try to be accurate, or at least consistent with current data.  Advice/thoughts welcome.
%Maxwell: Good idea to mention the masses indeed. If we use MSun, should we somehow explain why this is a more commonly used unit than kg? The introduction of the MSun can be done in a box, since the seven sisters may not be aware of our Sun and therefore it is unnatural for them to normalize their masses to our Sun. The introduction to planet formation seems to be a bit too fast, but we will see how the story develops.
%Javier: will it make sense to indicate that the seven Pleiades do exist in nature even though the fictitious storyline takes off from there? That may help readers understand that although the story is fictitious the properties of the stars are based on real numbers.
%Nathan: I started looking into the details of these stars, and have come to realize I should have done this a long time ago.  Sincerest apologies.  I wrote in some rough notes using mostly wikipedia, which does give specific references but they need to be checked.  We should discuss how we want to handle the details of these stars, and how accurate we want to be,  It would be much cooler to be accurate, in my opinion, but this will require some over-arching decisions to go along with the “rules” we decide on.  Once decided, wew then need to think about to edit or re-structure the book to accommodate the more accurate or observed stellar properties.  I mean, both Taygete and Alcyone are actually unresolved multiples, the latter a binary and the former a triplet.  So we could just work with that directly, which might be easiest.
%
%This chapter begins with the births of the seven sisters, and their mother Pleione instructing them on the basics of communication, counting and numbers, and the development of basic arithmetic.  The focus is on the first born os the siblings, namely Maia.  The chapter should end with the birth of the next sister(s) in line, but save the births of the remaining siblings for the next chapter.  Pleione disperses at the end of this chapter.



%\section{Guidelines} \label{guide}

%\par \nar Before we begin, let us first consider a few features in this book implemented to help communicate its contents.  


%\par \nar The first thing you will notice is that the text and story line have been segmented into a “box” structure.  This is used to define and communicate different levels of education needed as a minimum by the reader to understand certain key concepts contained within each “level”.  For example, level zero is shown in black text, and is not confined to any such boxes.  This “zeroth level” contains most of the underlying story line, and can be understood by readers having a base level of education attained in grade school (i.e., up to and including high school).  It does not require any knowledge of any text appearing in any other “levels” in order to understand its contents.  If the reader chooses to read only the black text, then they will be able to follow along with the story but will neglect access to some of the higher-level education materials presented in this book.  

%\par \nar Subsequent “higher” levels require more education to understand their contents.  All of the corresponding “higher level” education materials are contained within, and isolated to, the box format described above.  Level one is indicated by red boxes and lettering, and requires a basic undergraduate STEM-based education to understand the content.  All text appearing in red boxes assumes that the reader is following along with all text appearing in both red and black fonts.  Level two is indicated via blue boxes, and requires an education in undergraduate physics to comprehend.  All text appearing in blue boxes assumes that the reader is following along with all text appearing in blue, red and black boxes.  Level three is indicated via green boxes and lettering, and is aimed at the graduate level.  And so on.  

%\par \nar In summary, each level of text, indicated by a different type of box and lettering, and requiring the indicated background education, requires knowledge of the content of each level below it, but no knowledge of any level above it.  In principle, the reader can begin to appreciate the book from a very young age and, if they so choose, return to it again and again over the course of their life as their own personal education develops.  Each time the reader re-discovers this book, we hope they will discover new things, with new content emerging as it becomes accessible to them.

%\par \nar Now, we know what you are thinking:  "How on Earth am I supposed to safely navigate my way through Level 0, when I'll be dodging around or even fumbling right through Level 1, or Level 2 or, have mercy, Level 3!  This sounds hard.  If reading this book is anything like playing a video game, players on Level 0 are going to need an infinite number of lives to complete it." 

%\par \nar Rest assured, we have taken great care to ensure for every reader a safe, secure and pleasant journey through each and every chapter, independent of their level of education.  This includes granting users on Level 0 an infinite number of lives.  
%This is accomplished by clearly segmenting off any text associated with higher levels that require a more specialized knowledge to access or understand.  To this end, 
%The key point to keep in mind is that we use colored text to indicate to the reader the level associated with that text, and boxes to indicate that the text should only be regarded by readers at the appropriate level, and ignored otherwise.  For example, if you are only reading Level 0, then you can safely ignore any and all colored text and boxes, without any details being lost from the principle story line following our main characters.  If you are still feeling reluctant, try to keep in mind that a paper cut is probably the worst thing that could happen to you upon reading our book.  Of course, this assumes you are not carrying any explosive devices on your person. In that case, exploding is probably the worst thing that could happen to you.

%PYRAMID DIAGRAM SHOWING THE COLORS AND CORRESPONDING EDUCATION LEVELS?  Obviously, the highest level of education goes at the top, indicating all levels below it are required to follow the story development in these sections.  The base of the pyramid should be in black, given the above descriptions.

%\par \nar The second thing you will notice is that the characters in the story appear in corresponding illustrations.  These are included not just for entertainment purposes, but are instead primarily used to help convey the underlying story and certain key concepts.  We do our best to ensure that the illustrations are accurate and precise in terms of the underlying physics, to help communicate important concepts typically related to the physics governing the birth and subsequent evolution of stars.  For example, the sizes and colors of the stars in the story are accurate, given their relative masses.

\end{document}
