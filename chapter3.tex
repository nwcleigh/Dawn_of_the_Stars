\section{From a lonely road, to a crowded cluster}

$\Sterope$ watched in sadness as her family slowly drifted out of view.  A tear, or more accurately plasma, streamed down the length of her face.  $\Sterope$ has a lonely road ahead.  But, little did she know, her isolation would not last for long.

She drifted aimlessly through the Cosmos for several tens of thousands of years.  The scenery was nice for the most part, since she remained close to the disk of the Milky Way where most stars are currently being born.  The nebulae from which they are birthed are often beautiful to behold, illuminated by their children to produce light of many wavelengths.  A true symphony of color.  

$\Sterope$ noticed that she happened to be moving in a very similar direction, with a very similar velocity, to a particularly far off but bright blob she couldn't quite resolve with the naked eye.  She could infer a rough distance based on having monitored her relative motion to it, however.  It was far away and extended, no doubt about it.  Too big to be a star.  Too bright to be a few stars or even a gas cloud.  So what could it be?  That doesn't leave much...

As the days, weeks and months passed, the far off blob drifted ever closer, occupying a larger and larger fraction of $\Sterope$'s view of the sky.  Regardless, the blob eluded resolution to the bitter end.

\subsection{A new home...?}

One day, $\Sterope$ awoke surrounded by stars, creatures very much like her.  There were thousands, perhaps millions.  She suddenly recalled she had been drifting ever closer to something bright and dense.  If her rough calculations had been right, she was about due for a very close encounter with whatever the bright blurry object turned out to be.  To her surprise, not only was it a compact cluster of stars, but before awakening she had inadvertently wandered right in to the middle of it.

The denizens of this cluster were typically old (i.e., billions of years in age), but the range of stellar ages was large; clearly, multiple episodes of stellar birth had occurred here at one time.  All together, a very larg number of stars occupied a volume roughly 10$^6$ times smaller than she had come to be familar with in the field of the Milky Way, for the same number of stars.  

Obviously, traveling through the vast low-density and largely empty space of the Galactic field occupied the most time for any traveler, due its vast extent.  This is No Man's Land, a sea of now dead star-forming regions long forgotten, exhausted of their gas supply.  Out in the Galactic halo and far from the disk of the Galaxy where most stars are currently being born, only old massive and dense globular clusters exist, fossil records of a very early phase in the formation of the Galaxy.  Apart from these, stars are about as rare as diamonds in a McDonald's restaurant. 

\Jane Hello there!  I see you are passing through in something of a hurry.  Uh...and there you go again.  Bye!

$\Sterope$ realized she was moving a little faster than the other stars in the cluster.  Suddenly, she whipped past another one.

\Dan Hey!  Watch where you're going!

\Sterope So sorry!

$\Sterope$ exclaimed after the scary incident, as she quickly faded in to the distance.  She looked back at the scene of the near-collision, grateful to fate for those few kilometers that spared her from a merger.

%\Marie So sorry!

$\Sterope$ turned back around to discover that she was about to collide with yet another star, albeit a smaller one.  Frightened, she closed her eyes and hoped for the best.  

\Louise Whoa! Whoa! WHOA!

It's a close call.  But the pair of stars manage to avoid a direct collision.  Instead, $\Sterope$ flies past $\Louise$, too fast to say hello.

As $\Sterope$ whips by, her gravitational influence is felt by $\Louise$, and vice versa.  No doubt about it.  $\Sterope$ is more massive than the other stars in the cluster, since most are much older than her.  When she undergoes a close encounter with a much less massive $\Louise$, momentum conservation dictates that she induces a strong deflection to $\Louise$'s trajectory through the cluster.  In this case, $\Louise$ is flung off and escapes from the cluster, causing $\Sterope$ to end up gravitationally bound to it in $\Louise$'s stead.  Momentum conservation strikes again!  Needless to say, $\Sterope$ felt terrible.

\Louise  AAAAAAHHHH!!! Help!  I'm floating away!

\Sterope I'm \textit{so} sorry!  I didn't mean to do it!  It was an accident.  

\Louise Uh...I don't think that helps me.  ...Nope, I'm still escaping to infinity.  Crap.  This is all your faaauuu...

$\Sterope$ could barely hear $\Louise$ now, as she retreated beyond the tidal boundary of their host cluster and in to the space beyond.  

\Sterope WHAT?! I can't hear you?!?

$\Sterope$'s shoulders slumped.  Another one bites the dust.  Saddened, she mutters solemnly to herself:

\Sterope I am more sorry than I could ever say.  Well...good luck, I guess.  

\subsection{The neighbors, introductions, and the next step}

$\Sterope$ suddenly realized she was completely surrounded by stars.  They were so close she could make out the faces of hundreds, even thousands, of them with the naked eye.  This was a little less close to comfort than $\Sterope$ had grown accustomed to over the last few years.  In the neighborhod of the $\Sun$, the approximate inter-stellar distance is of order one parsec.  Said another way, the distance between the $\Sun$ and its next nearest neighbor in the Galaxy, Proximi Centarui, is about one parsec.  Way out in the halo of the Galaxy, the Galactic field, No Man's Land, the inter-star distance drops by many orders of magnitude relative to the Solar neighborhood.  $\Sterope$ had been traveling for thousands of years, a short span of time relative to her expected lifespan, but long enough for her to travel out of the Galactic Disk, and in to the Galactic Halo; a sparse sphere of old stars surrounding the Galactic Disk and Bulge.  Here, dense gravitationally-bound bundles of stars, called globular clusters, reside.  And not much else.

$\Sterope$ now found herself in the very core of just such a globular cluster, having inadvertently collided with it while escaping from the Galaxy, via the Galactic Halo.  Gravity had acted to focus $\Sterope$'s trajectory, drawing her inward toward the million solar mass globular cluster.  Within the cluster core, the stellar density was now about a million times higher than in the Solar neighborhood; the average distance separating $\Sterope$ from her closest neighbor had gone from about a parsec, to about 1/100's of a parsec.  

Startled and overwhelmed by all the staring faces, $\Sterope$ gasped.  It sure was a lot of personalities to introduce yourself to, get to know and, let's face it, tolerate.  $\Sterope$ wasn't so sure she was up for the job.

\Enrico Why hello there!  I am $\Enrico$, it's nice to meet you.

$\Sterope$ turned suddenly, toward the mysterious voice.

\Sterope Uh...hi.  My name is $\Sterope$.  If you don't mind me asking, where am I exactly?

\Enrico You find yourself in an old star cluster.  Most of the million or so stars spanning the roughly twenty parsecs of our cluster, where the outer reaches can be found, and where stars slowly bleed back in to No Man's Land, were born at more or less the same time as the rest of the older stars in the Galaxy.  From the same Mother Cloud.  I, on the other hand, am older and come from a different generation of stars.  

\Sterope All the stars are packed so close together here...

\Enrico Yep, it's crammed in here all right.  You get used to it pretty quickly.  Mostly you don't notice it.  But, every now and then, two stars do smash in to each other directly.  They collide.  BOOM.  More frequently though, two stars undergo strong deflections, during which one star passes by another star so closely that their stellar surfaces are almost touching.  Of course, if their surfaces don't touch, they tend to bestow a strong defletion, one to the other.  This deflects their trajectories, and can either slow or speed them up.  

\Sterope Well, which is it?  Do they speed up or slow down?

\Enrico More massive stars tend to accelerate lower mass stars more easily, for a net gain of linear momentum. Conversely, lower mass stars tend to \textit{be} accelerated by more massive stars such that, by conservation of linear momentum, the more massive perturbers tend to be decelerated. So, in the end, more massive stars tend to be slowed down, especially if they have large velocities, whereas lower mass stars tend to be sped up, on average.  Not much you can do about any of it, really, except enjoy the show...which by the way, I highly recommend.

$\Enrico$ winks at $\Sterope$, who laughs.

\Sterope How old are you...if you don't mind me asking?

$\Enrico$ thinks on the question for a moment...

\Enrico How old is the Universe now?  Wait, no, if I remember right, most stars yammer on about the number 13.7 Gyr for the age of the Universe, give or take a few hundred million years.  That sound about right to you?

\Sterope I guess, but I really don't know.  I was not born long ago compared to you; it hasn't even even been a Gyr since my birth.

\Enrico You are a young one.  Good luck on all of life's many adventures, child.  

\Sterope Uh...thanks?

Suddenly and without warning, plasma shot out of $\Enrico$'s surface, below his equator.  A coronal mass ejection.  An old star, convective cells toiled at $\Enrico$'s surface, stirring upward the plasma from deep within his belly.  The plasma hit $\Enrico$'s photosphere with enough force to escape, using the outward force supplied by nearby magnetic field lines.  An explosion of plasma emerged from $\Enrico$'s lower hemisphere, not unlike a volcano erupting.

\Enrico Uh... My apologies.  This old star is at the mercy of surface convection!

\Sterope Oh, that's quite alright.  Could happen to anybody.

\Enrico Back to the question at hand!  Okay, so if the Universe is a little less than 14 billion years old, that must make me...twelve and a half billion years old!  Yep, that's the number.  Most of these other stars are more like nine billion years old...give or take a billion years or so.  Young pups, and they mostly keep to themselves.  I'm sorry if they seem rude that way, but in fairnes over 90\% of this cluster is comprised of their generation, so I guess it's no surprise they mostly keep to themselves.  

\Sterope Actually, I'm a little relieved.  This is a \texit{lot} of stars in a very crammed space.  I'm really not used to it, and was worrying how I would even begin getting to know all of these faces.

\Enrico Oh, don't worry about that.  I'm happy to chat any time you like, and to defer any time you prefer not to.  But these others will mostly inter-mingle with their own kind.  You'll be lucky if even one of them strikes up a serious conversation with you.  I mean, they are polite, I give them that.  They just prefer not to engage with outsiders directly.  

\Sterope Well, I'm perfectly fine with that for the time being.  Hmmm... How do you suppose I might find my way out of here?

\Enrico You want out, do ya?  Not the first time I've heard that.  But I warn you, accomplishing that feat is anything but easy.  Can I ask:  Where are you heading in such a hurry?

\Sterope Well, nowhere, really.  I guess I was just hoping to explore the Galaxy, and maybe even stumble across my missing siblings.  We were all born from the same Mother Cloud.  Ever since our birth cluster dispersed, I've been a little worried about them.  Now though, I'm more than a little worried that it's a really big Galaxy out there, and if they have been traveling as I have, then finding them within my lifespan could be impossible.  Still, I have to to try.

\Enrico Well, it sounds to me like you want to make your way back to the Galactic Bulge, which surrounds the central nuclear cluster and a non-negligible fraction of the Galactic Disk.  Unless I miss my guess, if you started out in the Galactic Disk somewhere, which is most likely for a young pup such as yourself, then you're best bet for finding your siblings is in and around that central region of our Galaxy.  I know that doesn't narrow it down as much as you'd probably like, but at least it's a start.

\Sterope Thank you!  I really do appreciate it.  Yes, it is a \textit{definite} start.  Wait, just one more question:  How the heck do I get out of this cluster?

\Enrico Oh, right.  \textit{That} question.  Well, you're not going to like the answer.

\Sterope I don't care, try me anyway.

\Enrico To do that, you'll need to find not one, but \textit{two} black holes, lurking around most likely somewhere here in the core.  

\Sterope Wait, what's a black hole?  And why do I need \textit{two} of them?

\Enrico Well, to answer the second question, you need \textit{mass} and \textit{lots of it} confined to a small volume if you want to be able to achieve the acceleration you need to escape from the core of this cluster.  Looking at you, I'd guess you're, what, 3 maybe 4 solar masses?

\Sterope Hey! I really don't see how my weight is relevant to this discussion...

\Enrico Because I have some idea how massive the most massive black holes in this cluster might be, and you need \textit{two} that are each more massive than you.  The more massive they are relative to you, the better.

\Sterope Gotcha.  Alright, fine.  Last time I checked I was...

$\Sterope$'s voice drops to a whisper...

\Sterope ...3.3 solar masses or so.  Buuuuuut I've been budgeting my nuclear fuel more sensibly lately, so I think it's probably a bit less than that.  

\Enrico Nothing to be ashemed of as far as I am concerned.

\Sterope Oh please, what do you weigh?  Half a solar mass?

\Enrico Nah, more like a quarter solar mass, last time I checked.

\Sterope I hate you.  Alright, so what are these ``black holes'' you speak about, and how do we find them?

\Enrico Well, I guess the most important thing is that they are as dark as they come.  They don't shine.  So finding them is obviously a pretty serious challenge.  As to what they are, technically, they are the corpses of stars once much more massive than yourself.  Now wandering unseen through the Cosmos...usually, anyways.

\Sterope  Dead stars?  Really?  Okay, whatever.  So, basically, I am looking for ghosts haunting the cluster, which I rather conveniently cannot see.  And what is it you expect me to do with these dark ghosts once I find them?

\Enrico You'll have to capture 'em.  Well, actually, first you'll have to convince two of them to partner up and form a bound binary system.  Then, you're going need to convince them to let you take a run at them.  You'll have to work out the details on your own, which I warn you are not as straight-forward as you might expect, especially when chaos rears its ugly head and enters the picture.  But, in principle, two black holes bound in a compact binary should have enough binding energy to give you the acceleration you need to escape the cluster.  

\Sterope Wow, that is a \textit{super} complicated plan.  Sigh.  Well, I guess I'll have to find a way to make it work, which brings me to my last question:  How, in the name of Hell, do I find these black holes?

\Enrico There is only one sure fire way, child.  You must search for a star orbiting within a companion-less binary star system.  If the black hole forms a binary star system with another luminous star, \textit{any} luminous star, then it becomes possible to observe a star in orbit about something that cannot be seen.  This immediately implies the unseen presence of a dark compact object binary companion.  You can then use the orbital speed of the luminous companion at different distances from its location from the unseen black hole to calculate the approximate mass of the black hole.  

\Sterope Wooooooooow.  This sounds like a loooooot of work.  I don't know about this... I mean, what are black holes even like?  \Textit{If} I can find not one, but \textit{two} of them, do you think they will agree to help me escape from this cluster?

\Enrico Hmmmm... A fair and good question.  Few stars have gotten to really know a black hole and survived to tell the tale, to be honest with you... But they do have one weakness:  their constant hunger to grow.  Perhaps if you have food to offer in exchange for their services, they would be more inclined to help you out.  

\Sterope  So...bribery?  You are suggesting that I bribe them?  

$\Enrico$ shrugs rather non-chalantly.

\Enrico It often works, I have to say.

\Sterope With what?

\Enrico Uh, well, mass.  Mass-energy to be precise, but from your perspective I think it's going to be easiest to just aim for mass.

\Sterope Okay...Again though, with what?

\Enrico Other stars?

\Sterope WHAT?! You want me to deliver other stars to these black holes so that they can eat them?  And the stars will die?

\Enrico No! No! I was just saying such a scenario \textit{could} work, at least in principle.  But, yes, those stars would surely die.  There are other options though!  Murder is not the only one.  Any mass will do.  The more of it you have, the better your position to bargain.

\Sterope Okay, okay. So the mass could just as easily be the random crud out in No Man's Land?

\Enrico I suppose so, yes.  Provided somebody could collect it all in to one place.

\Sterope Alright.  Now we're getting somewhere.  ...Wait, how do you suppose I find and collect the mass?

\Enrico Off the top of my head, by finding those stars on the verge of evolving off the main-sequence and somehow getting your self close enough to them (i.e., in a binary system) that, when they evolve off the main-sequence and expand to become giants, they transfer the mass in their expanded envelope over to you.

\Sterope  Uh...Okay, so basically I am going to have to somehow figure out a way to swap myself in to \textit{and} out of at least two normal stellar binaries...in addition to two black hole binaries so that I can eventually use their mass to accelerate me to above the escape speed from the cluster?  Just trying to wrap my head around this.  That seems like a lot.  Hmmmmm... Where do I even begin?

\Enrico Well, child, by my calculations, you need to participate in at least ten direct dynamical interactions with other singles or binaries in the cluster.  Two things can help with that:  increasing your mass, and reducing your velocity relative to the cluster average.  Both of these increase the rate of colliding with other objects.  The reason is because gravity helps you out a lot; it serves to focus on the relative trajectories of two colliding particles, making it so that they are more likely to collide.  Hence, slower incoming singles and binaries on wide orbits should also be able to collide directly with a binary due to this gravitational focussing, whereas without gravity it would not occur.

\Sterope Okay, got it...I think.  So...what do I do now?

\Enrico Not much you can do, but wait.  Don't worry though, gravity will do all the work for you.  It will carry you throughout this cluster, and deliver you close to other stars and binaries.  You are most likely to run in to the most massive objects in the cluster first, since it is more massive objects that exert the strongest gravitational force and, without even intending to, draw you in from further afar.  Usually the end result is only a close approach, but often the interaction will be direct...as you will find out for yourself!

$\Sterope$ noticed she had drifted farther from $\Enrico$ than before.  She realized the process would continue, and they would soon part ways.

\Sterope I notice I am drifting away from you.  I can barely hear you anymore, in fact.  Thank you so much, $\Enrico$, for all your help.  I'm off to find those two massive black holes!

\Enrico Good luck, young one.  It will take some time, but I have no doubt you will realize your goal of escaping eventually.

$\Sterope$ began her long journey through the cluster.  A sea of faces came in to and faded out of view.  One thing quickly became apparent to $\Sterope$ about her temporary neighbors:  they were highly skilled at avoiding eye contact.  So she continued on, mostly in silence, in search of the two massive black holes whose help she seeked.




