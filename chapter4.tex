\documentclass[main.tex]{subfiles}
%\usepackage[T1]{fontenc}
%\usepackage{ae,aecompl}
%\usepackage{graphicx}	% Including figure files
%\usepackage{amsmath}	% Advanced maths commands
%\usepackage{amssymb}
%\usepackage{caption}
\begin{document}

\chapter{From a Lonely Road, to a Crowded Cluster}

\par \nar \rmsterope~watched in sadness as her family slowly drifted out of view.  A tear made of plasma streamed down the length of her face.  \rmsterope~ had a lonely road ahead.  But, little did she know, her isolation would not last for long.

\begin{tcolorbox}[sharp corners, colback=red!30, colframe=red!80!blue, title=The Galactic Field]
\par \textcolor{red} {The field of the Milky Way, or No Man's Land, effectively denotes those regions in our Galaxy outside star clusters, the Galactic disk and its inner, denser central regions.  The Galactic field is where the stellar density is at its lowest in our Galaxy, and the time for light to travel from one star to the next can be many many years. To put it into context, the Sun's nearest neighbor, Proxima Centauri, is located roughly one parsec away (or about 3.26 light years).}
\end{tcolorbox}

\par \nar She drifted aimlessly through the Cosmos for several tens of thousands of years.  The scenery was nice for the most part, since she remained close to the disk of the Milky Way where most stars are currently being born.  The nebulae from which they are birthed are often beautiful to behold, illuminated by the light of the stars they birth.  A true symphony of color.  



\par \nar \rmsterope~  noticed that she happened to be moving in the general direction of a far off but bright blob.  The object was too far away to be resolved by the naked eye.  \rmsterope~ was unsure of the nature of the object, but she could infer a rough distance based on having monitored her relative motion to it.

%\footnote{}  
\begin{tcolorbox}[sharp corners, colback=blue!30, colframe=blue!80!blue, title=Parallax]
\par \textcolor{blue} {The term ``parallax'' refers to a technique used by astronomers to calculate the distances to the nearest objects to the Sun.  It is the angular separation on the sky between two independent measurements of an object's angular position, performed from two distinct locations with a large displacement or baseline.  The largest possible baseline is given by the motion of the Earth about the Sun, such that the two different measurements of the object's angular position are separated in time by $\sim$ 6 months, and the length of the corresponding baseline is twice the Earth's orbital distance from the Sun.  The basic idea behind the concept is most easily conveyed by extending one's arm outward in front, perpendicular to the torso.  Extend one finger in to the air and focus on it.  Close one eye.  Then close the other eye, and open the originally closed eye.  Repeat this process.  You will notice that your finger appears to suddenly shift positions relative to the background.  The scale of this shift depends on the relative distance between your eye and your finger, as well as between your finger and the background.  Hence, measuring the size of the angular displacement in degrees or radians can be used to compute the object's distance from the observer.  The concept is identical to that underlying stellar parallax. PICTURE OR DIAGRAM?} 
\end{tcolorbox} 


\par \nar The object was \textit{very} far away, no doubt about it.  Too big to be a star.  Too bright to be a few stars or even a gas cloud.  So what could it be? 
%NL: Maybe we should define terms as ``basic'' as ``velocity'', as in the above?  But I am guessing this is not its first appearance...

\begin{tcolorbox}[sharp corners, colback=red!30, colframe=red!80!blue, title=Globular Clusters]
\par \textcolor{red} {Globular clusters are massive star clusters, much more massive than the open cluster in which \rmsterope~ and her siblings were born.  They can be home to more than a million stars, all old and born a very long time ago.  The ages of globular clusters rival that of the Milky Way itself, having been estimated to exceed 10 billion years in some cases.  They contain the fossil record of a very early episode of star formation in our Galaxy, and understanding their origins is crucial to understanding how galaxies are born and grow.  Most galaxies are home to hundreds or even thousands of globular clusters.  They live in the halos of galaxies, far away from galactic disks where gas is still being converted into stars in young star-forming regions.  Globular clusters are home to some of the highest stellar densities ever observed in the Universe, exceeding 10$^6$ stars per cubic parsec in their cores in many cases.  For comparison, The density in the solar neighborhood is roughly 10$^6$ times smaller.  This means that, if you take the Sun and its nearest neighbor Proxima Centauri and place them in their current configuration in the core of a dense globular cluster, roughly one hundred stars would lie between the pair.  Crowded does not do it justice.} 
\end{tcolorbox} 
%This means that, if we took the Sun and its nearest neighbor Proxima Centauri and put them in the core of a dense globular cluster, roughly 100 stars would lie between the pair. 

\par \nar As the days, weeks, months, years and centuries passed, the far off blob drifted ever closer, occupying a larger and larger fraction of \rmsterope's view of the sky.  Regardless, the blob eluded resolution for what seemed to \rmsterope~ like a very long time.  But it continued to draw closer....
%NL: What about words like ``resolution''?  This seems a little high-level for early high school and younger.  Crap.  I guess we need to formally decide on a level of education to aim at, and go from there.  Just hoping to flag stuff like this for later...

\begin{figure}
\includegraphics[width=\columnwidth,angle=270,origin=c]{ch4_1.pdf}
\caption{\rmsterope~ noticing a fuzzy blob far off in the distance.  As she will soon discover, she is looking at an unresolved globular star cluster.  Illustration by Andre Pipe Oliva.
\label{fig:fig1}}
\end{figure}

\begin{tcolorbox}[sharp corners, colback=red!30, colframe=red!80!blue, title=Travel Time]
\par \textcolor{red} {How long would it take \rmsterope~ to travel from the disk of our Galaxy out into the halo where globular clusters reside?  To answer this question, consider a globular cluster residing in the halo, roughly 10,000 parsec, or 10 kpc, away from \rmsterope~.  A typical stellar velocity in the solar neighborhood is 30 km s$^{-1}$.  Taking this as the relative velocity between \rmsterope~ and the target globular cluster, it would take roughly 10,000 pc/(30 km s$^{-1}$) $\sim$ 330 Myr.  This is much shorter than the expected age of \rmsterope~, as discussed in Chapter 1.}
\end{tcolorbox} 


\section{A New Home...?}

\par \nar One day, \rmsterope~  awoke surrounded by stars.  There were thousands, perhaps millions, within her immediate view.  She suddenly recalled drifting ever closer to something bright and blurry that she had never quite been able to see before falling asleep.  If her rough calculations were right, she was about due for a very close encounter with whatever the bright blurry object happened to be.  To her surprise, the mysterious object turned out to be a compact very dense cluster of stars.  Before awakening she had inadvertently wandered right in to the middle of it.
%Moved to chapter 1:
%\footnote{As was the case for the Seven Sisters, most, if not all, stars are thought to be born in "clusters", gravitationally-bound and compact groupings of stars, ranging in numbers from a few tens to several million, with central densities in the range 10 M$_{\odot}$ pc$^{-3}$ - 10$^6$ M$_{\odot}$ pc$^{-3}$.  These stars are all thought to form from the same Giant Molecular Cloud, as gravity shaped it into denser knots and filaments, creating the ideal environment for star formation.  However, the most massive star clusters, may be the products of the mergers of several lower mass clusters.  The answer to this question is still an active area of research.}

\par \nar The denizens of this cluster were typically old.  In fact, billions of years in age. But the range of stellar ages was large.  Clearly, multiple episodes of stellar birth had occurred here at one time or another.  All together, a very large number of stars occupied a volume so compact the stellar density reached a million stars per cubic parsec at its center.

\par \nar \rmsterope~ considered her surroundings carefully, trying to connect it to whatever she could remember before falling asleep.  She could at least recall that she had traveled through the Galactic field to get to her current location.  Obviously, traveling through the sparsely populated space of the Galactic field occupied the most time for any traveler, due its vast extent.  After all, it was No Man's Land, a sea of now dead and dispersed star-forming regions long forgotten, exhausted of their gas supplies.  Out in the Galactic halo and far from the disk of the Galaxy where most stars are currently being born, it would seem that only old massive and dense globular clusters exist.  Apart from these, stars are extremely scarce in these barren outer regions of our Galaxy. 
%\footnote{NEEDS A PICTURE, with a ``you are here'' showing the Sun. Point out the gas, star-forming regions, etc.

%NL: Adopt names of "old" people for the older stars:  i.e., Louise, Peggy, Daisy, Rose, Norma, etc.

\par \nar Suddenly, a strange voice interrupts \rmsterope~'s train of thought...

\par \Jane Hello there!  I see you are passing through in something of a hurry.  ...and there you go again.  Bye!

\par \nar \rmsterope~ realized she was moving a little faster than the other stars in the cluster.  Suddenly, she whipped past another one.

%\footnote{}  
\begin{tcolorbox}[sharp corners, colback=blue!30, colframe=blue!80!blue, title=Velocity Dispersion]
\par \textcolor{blue} {What are typical relative velocities for stars in globular clusters?  To answer this question, it is useful to introduce the concept of the root-mean-square velocity, which provides a useful approximate estimate of the typical velocity of a star in a cluster relative to the cluster center of mass.  We can use the Virial Theorem, or $2T + U = 0$, to obtain an estimate for the typical stellar velocity.  Assuming that our star cluster is well approximated as a Plummer sphere, we set $T = \frac{1}{2}m_{\rm i}N<\sqrt{v^2}>^2$ and $U = -\frac{GN^2m_{\rm i}^2}{\sqrt{r^2 + a^2}}$, where $r$ is the distance from the cluster center and a is the Plummer or core radius.  This is turn gives:
\begin{equation}
<\sqrt{v^2}> = \Big( \frac{GNm_{\rm i}}{\sqrt{r^2 + a^2}} \Big)^{1/2}.
\end{equation}
For a total cluster mass of order 10$^6$ M$_{\odot}$ and a characteristic cluster size of about 10 pc, the root-mean-square stellar speed tends to be of order 10 km s$^{-1}$.  In other words, in massive dense star clusters, stars typically travel of order 10 kilometers each and every second.  Less massive clusters tend to have smaller root-mean-square speeds, reaching all the way down to 0.1 km s$^{-1}$ in open clusters.  Stars in the Galactic field tend to move at several tens of km s$^{-1}$, so \rmsterope~ is traveling through her new host cluster a fair bit faster than are the other stars in the cluster.
} 
\end{tcolorbox} 

\par \Gene Hey!  Watch where you're going!

\par \Sterope So sorry!

\par \nar \rmsterope~ glanced back at the scene of the near-collision, grateful for those few million kilometers separating them at her distance of closest approach, which \textit{barely} spared her from a direct collision.

\begin{tcolorbox}[sharp corners, colback=red!30, colframe=red!80!blue, title=Stellar Collisions and Blue Stragglers]
\par \textcolor{red} {In the dense cores of globular clusters, the time for two single stars to undergo a direct collision is thought to be much shorter than the age of the host cluster.  This implies that collisions do occur in globular clusters, and their products remain lurking somewhere within the cluster.  But what happens when two ordinary main-sequence stars undergo a direct collision?  One very likely possibility is that the two stars will merge and in the process become a single new star.  Such collision products are often calle "blue stragglers".  This nickname comes from the fact blue stragglers are more massive and hence bluer than most other cluster stars, and the fresh hydrogen mixed into their cores during the collision prolongs their main-sequence lifetimes.  Hence they straggle in evolving away from the main-sequence phase of evolution, before evolving on to become a giant star.}
\end{tcolorbox} 

\par \nar \rmsterope~ turned back around to face her forward motion, only to discover that she was about to collide with yet another star.  Frightened, she closed her eyes and hoped for the best.  

\par \Louise Whoa! Whoa! WHOA!

\par \nar It's a close call.  But the pair of stars manage to avoid a direct collision.  Instead, \rmsterope~ flies past \rmlouise, too fast to say hello.

%\footnote{}  
\begin{tcolorbox}[sharp corners, colback=blue!30, colframe=blue!80!blue, title=Collision Rate I]
\par \textcolor{blue}{A simple estimate for the rate of direct collisions between identical single stars in a star cluster, borrowed from chemistry by considering a particle traveling through a uniform gaseous medium, comes from the mean free path (MFP) approximation.  Crudely, the MFP can be estimated by dimensional analysis, such that the mean free path $l$ is $l \sim$ 1/$n\sigma$, where $n$ is the mean particle number density and $\sigma$ is the collisional cross-section (i.e., an area corresponding to the direct overlap of two stars' radii; if a star passes within this area, a collision occurs).  If the mean particle velocity is $v$, then the rate of direct collisions is $\Gamma \sim n\sigma{v}$ and the mean time between collisions is $\tau \sim$ 1/$\Gamma$.}  
\end{tcolorbox} 

%\footnote{}  
\begin{tcolorbox}[sharp corners, colback=green!30, colframe=green!80!blue, title=Mean Free Path Approximation]
\par \textcolor{green} {On average, how far do we expect a star to travel in a cluster before undergoing a collision with another star?  Consider the core of a cluster of identical stars (where the core is loosely defined as the central regions of a cluster containing roughly 10\% of the total cluster mass), each having radius $R$.  Then, then geometric cross-section for collision is $\sigma = \pi(2R)^2 = 4{\pi}R^2$.  Now, let us consider a star moving at velocity $v$ through a fixed stellar background with stellar number density $n$. In a time $t$, the geometric cross-section $\sigma$ will sweep out a volume $V = 4{\pi}R^2vt$.  Hence, the mean distance traveled between collision events, called the mean free path $l$, can be taken as the length of the path divided by the number of collisions:
\begin{equation}
l = \frac{vt}{4{\pi}R^2vtn} = \frac{1}{4{\pi}R^2n} = \frac{1}{n\sigma}.
\end{equation}
The above simple derivation hopefully conveys the important concepts.  A more detailed derivation should relax the assumption of a fixed background of stars, and properly factor in the dependence on their velocity distribution.} 
\end{tcolorbox} 

%\footnote{}  
\begin{tcolorbox}[sharp corners, colback=green!30, colframe=green!80!blue, title=Collision Rate II]
\par \textcolor{green} {In order to evaluate the expected rate of stellar collisions in the dense core of a star cluster, let us assume that the system forms a gravitationally-bound, approximately spherical cluster, which we assume obeys a number density profile $n(r)$ and typical relative particle velocity $v$.  The mean time between direct collisions in the cluster core (where the density is highest and the interaction rate dominates in the cluster) can then be expressed using the mean free path approximation:
\begin{equation}
\label{eqn:mfp}
\Gamma = \frac{N_{\rm i}v_{\rm i+j}}{l_{\rm i+j}} = N_{\rm i}n_{\rm j}{\sigma_{\rm i+j}}v_{\rm i+j},
\end{equation} 
where $N_{\rm i}$ is the number of objects of type $i$, $n_{\rm j}$ is the number density of single stars of type $j$, $v_{\rm i+j}$ is the relative velocity at infinity between the target and incoming objects and $\sigma_{\rm i+j}$ is the corresponding cross-section for collision.
} 
\end{tcolorbox} 


\par \nar As \rmsterope~ whips by, her gravitational influence is felt by \rmlouise, and vice versa.  \rmsterope~ is more massive than the other stars in the cluster, since most are much older than her.  When she undergoes a close encounter with a much less massive \rmlouise, momentum conservation dictates that she induces a strong deflection to \rmlouise's trajectory through the cluster.  In this case, \rmlouise~ is flung off and escapes from the cluster, causing \rmsterope~ to end up gravitationally bound to it in \rmlouise's stead.  Momentum conservation strikes again!  Needless to say, \rmsterope~ felt terrible.
%\footnote{To remind the reader, the linear momentum of a particular is given by the product of its mass and velocity in 1D.  To change a particle's momentum, a net force must be applied for a finite interval of time.  Particles with more momentum require either a larger applied force or for a given force to be applied for a longer interval of time in order to significantly change their momentum.}


\par \Louise  AAAAAAHHHH!!! Help!  I'm floating away!

\par \Sterope I'm \textit{so} sorry!  I didn't mean to do it!  It was an accident.  

\par \Louise Uh...I don't think that helps me.  ...Nope, I'm still escaping to infinity.  This is all your faaauuullllttt...

\par \nar \rmsterope~ could barely hear \rmlouise~ now, as she retreated beyond the outer boundary of their host cluster and in to the empty space beyond, No Man's Land.

\par \Sterope WHAT?! I can't hear you?!?

\par \nar \rmsterope's shoulders slumped.  Another one bites the dust.  Saddened, she mutters defeatedly:

\par \Sterope I am more sorry than I could ever say.  Well... Good luck, I guess.  

\section{\rmsterope's New Neighbors}

\par \nar \rmsterope~ suddenly realized she was completely surrounded by stars.  They were so close she could make out the faces of hundreds, even thousands, of them.  This was a little too close for comfort for \rmsterope~, relative to what she had grown accustomed to over the last few million years traveling through the sparse Galactic field.  
%In the neighborhood of the Sun, the approximate inter-stellar distance is of order one parsec.  Said another way, the distance between the Sun and its next nearest neighbor in the Galaxy, Proxima Centauri, is about one parsec (or about 3.26 light years).  Way out in the halo of the Galaxy, the Galactic field, No Man's Land, the inter-star distance drops by many orders of magnitude relative to the Solar neighborhood.  \rmsterope~ had been traveling for millions of years, a short span of time relative to her expected lifespan, but long enough for her to travel out of the Galactic Disk, and in to the Galactic Halo; a sparse sphere of old stars surrounding the Galactic Disk and Bulge.  Here, dense gravitationally-bound bundles of stars, called globular clusters, reside.  And not much else.

\par \nar \rmsterope~ was now in the very core of just such a globular cluster, having inadvertently collided with it while escaping from the Galaxy, via the Galactic Halo.  Gravity had acted to focus \rmsterope's trajectory, drawing her inward toward the million solar mass globular cluster.  Within the cluster core, the stellar density was now about a million times higher than in the Solar neighborhood; the average distance separating \rmsterope~ from her closest neighbor had gone from about a parsec, to about 1/100's of a parsec.  

\begin{tcolorbox}[sharp corners, colback=blue!30, colframe=blue!80!blue, title=Gravitational Focusing I]
\par \textcolor{blue} {We are typically used to thinking of the collisional cross-section simply as the geometric surface area corresponding to the radii of two particles overlapping at closest approach.  That is, if the particle radius is R, n the geometrical cross-section for collision is $\pi$R$^2$.  The geometric cross-section can be enhanced when gravity is at work, and its effects in altering the velocities and trajectories of the interacting particles are non-negligible.  This new cross-section, called the gravitationally-focused cross-section for collision, can be calculated using conservation of energy and angular momentum. }
\end{tcolorbox} 

\begin{tcolorbox}[sharp corners, colback=green!30, colframe=green!80!blue, title=Gravitational Focusing II]
\par \textcolor{green} {If we include the contribution from gravitational focusing to the collisional cross-section for collision, we obtain:
\begin{equation}
\label{eqn:gf}
\sigma_{\rm i+j} = {\pi}b^2 = {\pi}p^2\Big[1 + \frac{2G(m_{\rm i} + m_{\rm j})}{pv_{\rm i+j}^2}\Big],
\end{equation}
where $b$ is the impact parameter for a pericenter distance $p = 2R$ (i.e., distance at closest approach corresponding to a collision for identical particles) between two bodies with masses $m_{\rm i}$ and $m_{\rm j}$ that approach each other with a relative velocity at infinity of $v_{\rm i+j}$.  Equation~\ref{eqn:gf} is derived using conservation of energy and (linear and angular) momentum during a two-body encounter.  For the single-single case, we 
%assume %m$_{\rm 1} =$ m and m$_{\rm 2} =$ 2m and 
set $p$ $\sim$ ($R_{\rm i} +$ $R_{\rm j}$)  where $R_{\rm i}$ is the radius of particle type $i$.  Equation~\ref{eqn:gf} then becomes:
\begin{equation}
\label{eqn:gf12}
\sigma_{\rm i+j} = {\pi}(R_{\rm i} + R_{\rm j})^2\Big[1 + \frac{2G(m_{\rm i} + m_{\rm j})}{(R_{\rm i} + R_{\rm j})v_{\rm i+j}^2}\Big] \approx \frac{2{\pi}G(m_{\rm i} + m_{\rm j})(R_{\rm i} + R_{\rm j})}{v_{\rm i+j}^2}.
\end{equation}
In order to evaluate whether or not gravitational-focusing should dominate the collisional cross-section, such that it is a non-negligible effect, we can compute the Safronov number.  This is the ratio of the escape velocity at the surface of star to the local root-mean-square velocity, or:
\begin{equation}
\label{eqn:safro}
%\Theta_{\rm i} = \frac{{v_{\rm esc}^{2}_{\rm i }}}{4\sigma^2} = \frac{Gm_{\rm i}}{2\sigma^2R_{\rm i}}.
\Theta_{\rm i} = \frac{{v_{\rm esc,i}^{2}}}{4\sigma^2} = \frac{Gm_{\rm i}}{2\sigma^2R_{\rm i}},
\end{equation}
where $\sigma$ corresponds to the velocity dispersion of a Maxwellian velocity dispersion with root-mean-square velocity $<\sqrt{v^2}>$.  If $\Theta_{\rm i} \gg 1$, then gravitational focusing dominates.  In the opposite limit, if $\Theta_{\rm i} \ll 1$, then gravitational focusing is unimportant and the geometric cross-section dominates.  
}
\end{tcolorbox} 

\par \nar Startled and overwhelmed by all the staring faces, \rmsterope~ gasped.  It sure was a lot of personalities to introduce yourself to, get to know and, let's face it, tolerate.  \rmsterope~ wasn't so sure she was up for the job.

\par \Enrico Why hello there!  I am \rmenrico, it's nice to meet you.

\begin{figure}
\includegraphics[width=\columnwidth,angle=270,origin=c]{ch4_2.pdf}
\caption{\rmsterope~ as she meets \rmenrico~ for the first time.  Illustration by Andre Pipe Oliva.
\label{fig:fig1}}
\end{figure}

\par \nar \rmsterope~ turned suddenly, toward the mysterious voice.

\par \Sterope Uh...  Hi!  My name is \rmsterope.  If you don't mind me asking, where am I exactly?

\par \Enrico You find yourself in an old star cluster.  A globular cluster!  Most of the million or so stars spanning the roughly twenty parsecs of our cluster, where its outer reaches can be found, and where stars slowly bleed back in to No Man's Land, were born at more or less the same time as the rest of the older stars in the Galaxy.  From the same Mother Cloud.  I, on the other hand, am older and come from a different generation of stars.  

\par \Sterope All the stars are packed so close together here...

\par \Enrico Yep, it's crammed in here all right.  You get used to it pretty quickly though.  Mostly you don't notice it.  But, every now and then, two stars do smash in to each other.  They collide head-on.  BOOM!  More often though, two stars undergo strong deflections, during which one star passes by another star so closely that their stellar surfaces are almost touching.  Of course, if their surfaces don't touch, then instead of colliding they tend to bestow a strong deflection, one to the other.  This causes a deflection in each star's trajectory, and can either slow or speed them up.  

\par \Sterope Well, which is it?  Do they speed up or slow down?

\begin{tcolorbox}[sharp corners, colback=blue!30, colframe=blue!80!blue, title=Energy Equipartition]
\par \textcolor{blue} {An important thermodynamics-based analogy can be drawn between star clusters and a gas in a container.  The mean velocity of a star can be regarded as a proxy for its temperature:  larger mean motions imply hotter temperatures.  Since all particles in the system are free to interact and exchange energy (via mostly weak deflections induced by gravity in star clusters, and direct collisions between atoms or molecules in a gas), the system tends to evolve toward a "thermalized" state in which all particles have comparable kinetic energies or temperatures.  This state is typically called "energy equipartition".  More massive stars exert a stronger gravitational force at a given distance, and so will typically accelerate a less massive star during a close approach, imparting unto it additional kinetic energy.  Conversely, the more massive star loses kinetic energy, and slows down.  Given enough time for many such close energy-exchanging interactions to have occurred, a star cluster will tend toward a state of energy equipartition, at least within certain subsets of the cluster (e.g., the core).}
\end{tcolorbox} 


\par \Enrico More massive stars tend to accelerate lower mass stars more easily, for a net gain of linear momentum.  Conversely, lower mass stars tend to \textit{be} accelerated by more massive stars such that, by conservation of linear momentum, the more massive perturbers tend to be decelerated. So, in the end, more massive stars tend to be slowed down, whereas lower mass stars tend to be sped up, on average.  Not much you can do about any of it, really, except enjoy the show...which by the way, I highly recommend.
%footnote{An important thermodynamics-based analogy can be drawn between star clusters and a gas in a container.  The mean velocity of a star can be regarded as a proxy for its temperature:  larger mean motions imply hotter temperatures.  Since all particles in the system are free to interact and exchange energy (via mostly weak deflections induced by gravity in star clusters, and direct collisions between atoms or molecules in a gas), the system tends to evolve toward a "thermalized" state in which all particles have comparable kinetic energies or temperatures.}

\begin{figure}
\includegraphics[width=\columnwidth,angle=270,origin=c]{ch4_3.pdf}
\caption{\rmenrico~ informs \rmsterope~ of what she must do in order to escape from the globular cluster she presently finds herself in.  Illustration by Andre Pipe Oliva.
\label{fig:fig1}}
\end{figure}

\par \nar \rmenrico~ winks at \rmsterope, who laughs.

\par \Sterope How old are you...if you don't mind me asking?

\par \nar \rmenrico~ thinks about the question for a moment...

\par \Enrico How old is the Universe now?  Wait...if I remember right, stars older than me used to yammer on about the number 13.7 Gyr for the age of the Universe, give or take a few hundred million years... I think.  Does that sound about right to you?

\begin{tcolorbox}[sharp corners, colback=red!30, colframe=red!80!blue, title=The Age of the Universe]
\par \textcolor{red} {All the available empirical data point to and are roughly consistent with an age of $\sim$ 13.7 Gyr  for the Universe.  For example, no star older than this critical upper limit has ever been observed.}
\end{tcolorbox} 


\par \Sterope I guess so, but I don't really know, to be honest.  I was not born long ago compared to you; it hasn't even been a billion years since my birth.

\par \Enrico You are a young one!  Good luck on all of life's many adventures, child.  

\par \Sterope Uh... Thanks?

\par \nar Suddenly and without warning, plasma shot out of \rmenrico's surface, below his equator.  A coronal mass ejection.  An old star, convective cells toiled at \rmenrico's surface, stirring upward the plasma from deep within his belly.  The plasma hit the outer layers of \rmenrico's atmosphere with enough force to escape, drawing on the outward force supplied by nearby magnetic field lines.  Another explosion of plasma emerged from \rmenrico's lower hemisphere, not unlike a volcano erupting.

%\footnote{The photosphere is the outer shell of a star from which light is radiated and emerges.}
\begin{tcolorbox}[sharp corners, colback=red!30, colframe=red!80!blue, title=Photosphere]
\par \textcolor{red} {The photosphere lies in the outer layers of a star's atmosphere.  It is the outer shell of a star from which light is radiated and emerges.  At larger distances from the star's center of mass, the photosphere gives way to the stellar corona, which is much hotter than the photosphere.  The physics underlying the reasons why the corona is so hot remain poorly understood and active area of research.}
\end{tcolorbox}

\begin{tcolorbox}[sharp corners, colback=red!30, colframe=red!80!blue, title=Coronal Mass Ejections]
\par \textcolor{red} {A coronal mass ejection (CME) refers to the ejection of plasma from the surface of a star.  Large amounts of matter (made up of protons and electrons, mostly) and electromagnetic radiation are launched out into space, and can either remain close to the stellar surface or be flung out through the corona and into the Solar System and beyond.  CMEs are known to be associated with very energetic shifts and changes in the star's outer magnetic field and a reorganization of its magnetic field lines.} 
\end{tcolorbox} 

\par \Enrico My apologies!  This old star is at the mercy of surface convection!
%!\footnote{If a steep enough temperature gradient (i.e., the temperature increases rapidly with increasing distance from the star's center of mass) exists in the interior of a star, an instability can arise in which heated plasma ascends and cooled plasma descends. This can also occur if the gas has a high heat capacity, implying that it cools slowly as it expands.  As a bubble of gas ascends, it finds itself in a region of lower pressure, allowing it to expand and cool.  The bubble must remain cooler and less dense than its surroundings in order to remain buoyant and continue to ascend.  Otherwise, if the bubble cools below the temperature of the new ambient plasma, its density will rise above that of the surrounding plasma and it will lose its buoyancy, sinking back down.}

\begin{tcolorbox}[sharp corners, colback=red!30, colframe=red!80!blue, title=Convection]
\par \textcolor{red} {If a steep enough temperature gradient (i.e., the temperature increases rapidly with increasing distance from the star's center of mass) exists in the interior of a star, an instability can arise in which heated plasma ascends and cooled plasma descends. This can also occur if the gas has a high heat capacity, implying that it cools slowly as it expands.  As a bubble of gas ascends, it finds itself in a region of lower pressure, allowing it to expand and cool.  The bubble must remain cooler and less dense than its surroundings in order to remain buoyant and continue to ascend.  Otherwise, if the bubble cools below the temperature of the new ambient plasma, its density will rise above that of the surrounding plasma and it will lose its buoyancy, sinking back down.} 
\end{tcolorbox} 


\par \Sterope Oh, that's quite alright.  It could happen to anybody.

\par \Enrico Back to the question at hand.  Okay, so if the Universe is a little less than 14 billion years old, that must make me...twelve and a half billion years old?  Yep, that's the number!  Most of these other stars are more like nine billion years old...give or take a billion years or so.  Young pups, and they mostly keep to themselves.  I'm sorry if they seem rude but, in fairness, over 90\% of this cluster is comprised of their generation, so I guess it's no surprise they mostly keep to themselves.  Strength, and confidence, in numbers.

\par \Sterope Actually, I'm a little relieved.  This is a \textit{lot} of stars in a very crammed space.  I'm really not used to it, and was worrying how I would even begin getting to know all of these faces.

\par \Enrico Oh, don't worry about that.  I'm happy to chat any time you like, and to defer any time you prefer not to.  But these others will mostly inter-mingle with their own kind.  You'll be lucky if even one of them strikes up a serious conversation with you.  I mean, they are polite, I give them that.  They just prefer not to engage with outsiders directly, which are very rare around here.  You're the first one in well over a billion years!  

\par \Sterope Well, I'm perfectly fine with them keeping to themselves for the time being.  I think I will do the same... Hmmm... How do you suppose I might find my way out of here?

\par \Enrico You want out, huh?  I suppose I cannot blame you.  It is crowded in here.  But I warn you, accomplishing that feat is anything but easy.  Can I ask:  Where are you headed in such a hurry?

\par \Sterope Well, nowhere, really.  I guess I was just hoping to explore the Galaxy, and maybe even stumble across my missing siblings.  We were all born from the same Mother Cloud.  Ever since our birth cluster dispersed, I've been a little worried about them.  Now though, I'm more than a little worried that it's a really big Galaxy out there, and if they have been traveling as I have, then finding them within my lifespan could be impossible.  Still, I have to try!
%\footnote{SOMEWHERE IN THE BOOK, INCLUDE A MAP OF THE GALAXY, ALONG WITH ALL SEVEN SIBLINGS' PATHS THROUGHOUT IT!}

\par \Enrico Well, it sounds to me like you want to make your way back to the Galactic Bulge, which surrounds the central nuclear cluster and a non-negligible fraction of the Galactic Disk.  Unless I miss my guess, if you started out in the Galactic Disk somewhere, which is most likely the case for a young pup such as yourself, then your best bet for finding your siblings is in and around that central region of our Galaxy.  I know that doesn't narrow it down as much as you'd probably like, but at least it's a start.

\par \Sterope Thank you!  I really do appreciate it.  Yes, it is a \textit{definite} start.  Wait, just one more question:  How the heck do I get out of this cluster?

\par \Enrico Oh, right.  \textit{That} question.  Well, you're not going to like the answer.

\par \Sterope I don't care, try me anyway.

\par \Enrico To do that, you'll need to find not one, but \textit{two} black holes, lurking around somewhere here in the core.  I know they are here...somewhere.  

\par \Sterope Wait, what's a black hole?  And why do I need \textit{two} of them?

\par \Enrico Well, to answer the second question, you need \textit{mass} and \textit{lots of it} confined to a small volume if you want to be able to achieve the acceleration you will need to escape from this cluster.  This is a basic requirement in order for gravity to impart the required total force needed to achieve such a high velocity.  That is, a sufficiently large gravitational acceleration must be imparted over a given timescale in order to accelerate an object to a final velocity that exceeds the local escape velocity of the cluster.  Looking at you, I'd guess you're, what, 2 maybe 3 solar masses?

\begin{tcolorbox}[sharp corners, colback=blue!30, colframe=blue!80!blue, title=Escape Velocity]
\par \textcolor{blue} {The local escape velocity is defined as the minimum velocity required at a given distance from the cluster center of mass for the total energy of the escaper to exceed zero or, equivalently, to become gravitationally unbound.  A simple way to calculate the local escape velocity, using conservation of energy, is to equate the kinetic energy of an object to its local gravitational binding energy, keeping the object velocity as a free parameter.  If you solve this equation for the object velocity, you will obtain the minimum speed needed at that position in the host cluster gravitational potential for the object to become gravitationally unbound and escape to spatial infinity (asymptoting to zero velocity at spatial infinity in the context of this simple idealized equality). The escape velocity from the surface of a sphere of radius $R$ and total mass $M$ is:
\begin{equation}
\label{eqn:vesc}
v_{\rm esc} = \Big( \frac{GM}{R} \Big)^{1/2}.
\end{equation}
} 
\end{tcolorbox} 

\par \Sterope Whoa whoa WHOA!  My appearance is pretty much the last thing I wanted to discuss with you.  Besides, I really don't see how my weight is relevant to this discussion...

\par \Enrico Because I have some idea how massive the most massive black holes in this cluster might be, and you need \textit{two} that are each more massive than you.  The more massive they are relative to you, the easier it will be for you to achieve the required acceleration.

\begin{tcolorbox}[sharp corners, colback=blue!30, colframe=blue!80!blue, title=Impulse]
\par \textcolor{blue} {In order for a star to be accelerated to above the escape velocity of a star cluster, it must somehow be imparted with enough additional momentum and kinetic energy to overcome the gravitational potential at the star's location in the cluster.  If this is achieved, then the star has positive total energy in the center of mass reference frame of the cluster, and becomes unbound.  The impulse is a useful quantity to understand how much momentum can be imparted for a given close approach between two stars.  The impulse, which has units of momentum, is defined as:
\begin{equation}
I = F_{\rm g}{\Delta}t,
\end{equation}
where $F_{\rm g}$ is the imparted gravitational force and $\Delta{t}$ is the time over which the force is applied.  Hence, for a given distance of closest approach, more massive perturbers will impart a larger acceleration, and slower relative velocities will increase the time $\Delta{t}$ over which the force is applied, resulting in a larger increase in momentum.  Thus, according to the imparted impulse, \rmsterope~ should seek out much more massive stars to acquire the needed acceleration to escape from the cluster, and she should aim to get as close to the stars as possible (since $F_{\rm g} \propto r^{-2}$, where $r$ is the distance separating the stars' center of mass) with a low relative velocity.
} 
\end{tcolorbox} 

\par \Sterope Gotcha.  Alright, fine.  Last time I checked I was...

\par \nar \rmsterope's voice drops to a whisper...

\par \Sterope ...2.9 solar masses or so.  Buuuuuut I've been blowing off winds for quite some time during my travels, so I think it's probably a bit less than that now.  

\begin{tcolorbox}[sharp corners, colback=blue!30, colframe=blue!80!blue, title=Stellar Winds]
\par \textcolor{blue} {
} 
\end{tcolorbox} 

\par \Enrico Nothing to be ashamed of as far as I am concerned.

\par \Sterope Oh please, what do you weigh?  Half a solar mass?

\par \Enrico Nah, more like a quarter of a solar mass, last time I checked.

\begin{tcolorbox}[sharp corners, colback=blue!30, colframe=blue!80!blue, title=Main-Sequence Lifetimes]
\par \textcolor{blue} {Stars in the main-sequence phase of lives share a relationship between their total mass and the duration of their lifespan spent on the main-sequence.  This relation is such that the most massive MS stars evolve the fastest, and have the shortest lifespans.  In fact, the evolution is so slow for the least massive MS stars that those with masses of only a few tenths of a solar mass should have total ages that exceed the current age of the Universe.  That is, some of these stars have been burning hydrogen into helium in their cores for over 13 billion years, and they are not yet done!
} 
\end{tcolorbox} 

\par \Sterope I am so jealous right now.  Alright, so what are these ``black holes'' you speak about, and how do we find them?

%DEFINE A BLACK HOLE HERE!!! TAKE FROM CHAPTER 1 !!!


\par \Enrico Well, I guess the most important thing to know is that they are as dark as they come.  They don't shine.  At all.  So finding them is obviously a pretty serious challenge.  As to what they are, technically, they are the corpses of stars once much more massive than yourself, now wandering unseen through the Cosmos.

\par \Sterope  Dead stars?  Really?  So, basically, I am looking for ghosts haunting this cluster, which I rather conveniently cannot see?  And what is it you expect me to do with these dark ghosts, once I find them?

\par \Enrico You'll have to capture 'em.  Well, actually, first you'll have to convince two of them to partner up and form a bound binary system.  Then, you're going to need to convince them to let you take a run at them.  You'll have to work out the details on your own, which I warn you are not as straight-forward as you might expect, especially when chaos rears its ugly head and enters the picture.

\begin{tcolorbox}[sharp corners, colback=blue!30, colframe=blue!80!blue, title=Chaos and the Three-Body Problem]
\par \textcolor{blue} {The chaotic three-body problem has evaded a solution for centuries.  The reason is simple:  small perturbations to the initial conditions compound over time to change the very outcome of the interaction (e.g., which of the three particles is ejected).  A well known term for this is the "butterfly effect".
} 
\end{tcolorbox} 


\par \Enrico But, in principle, two black holes bound in a compact binary should have enough binding energy to give you the acceleration you need to escape the cluster.

\begin{tcolorbox}[sharp corners, colback=blue!30, colframe=blue!80!blue, title=The Energetics of Escape]
\par \textcolor{blue} {Recall that the binding energy of a binary star defines its internal reservoir of \textit{negative} energy.  The more negative the binary binding energy, the closer are the companions (for a given pair of companion masses).  Liouville's Theorem tells us that the total volume in phase space is conserved during the time evolution of self-gravitating N-body systems.  In practice, what this means is that the more negative the total energy, the more likely it is particles will be accelerated to high velocities.  Thus, it is easier for binaries with more binding energy to accelerate interloping single stars to above the local escape velocity of the host star cluster, even deep in the cluster core where the escape velocity is at its highest.
} 
\end{tcolorbox}   

\par \Sterope Wow, that is a \textit{super} complicated plan.  Sigh.  Well, I guess I'll have to find a way to make it work, which brings me to my last question:  How, in the name of Hell, do I find these black holes?

\par \Enrico There is only one sure fire way, child.  You must search for a star orbiting within what appears to be a companion-less binary star system.  If the black hole forms a binary star system with another luminous star, \textit{any} luminous star, then it becomes possible to observe a star in orbit about something that cannot be seen.  This immediately implies the unseen presence of a dark compact object (i.e., a dead star) binary companion.  You can then use the orbital speed of the luminous companion as a function of distance from the unseen black hole to calculate the approximate mass of the black hole.  Just measure the time it takes the luminous companion to orbit the black hole once, and measure the distance from the center of mass that the companion appears to be orbiting.  The orbit should trace out an ellipse.  Calculate the typical orbital velocity by dividing the circumference of this circle by the orbital period.

\par \Sterope Wooooooooow.  This sounds like a loooooot of work.  I don't know about this... I mean, what are black holes even like?  \textit{If} I can find not one, but \textit{two} of them, do you think they will agree to help me escape from this cluster?

\par \Enrico Hmmmm... A fair and good question.  Few stars have gotten to really know a black hole and survived to tell the tale, to be honest with you... But they do have one weakness:  they have a constant hunger to grow.  Perhaps if you have food to offer in exchange for their services, they would be more inclined to help you out.  

\par \Sterope  So...bribery?  You are suggesting that I bribe them?  

\par \nar \rmenrico~ shrugs rather non-chalantly.

\par \Enrico It often works, I have to say.

\par \Sterope With what?

\par \Enrico Uh, well, mass.  Any mass will do.

\par \Sterope Okay...Again though, with what?

\par \Enrico Other stars?

\par \Sterope WHAT?! You want me to deliver other stars to these black holes so that they can eat them?  And the stars will die?

\par \Enrico No! No! I was just saying such a scenario \textit{could} work, at least in principle.  But, yes, those stars would surely die.  There are other options though!  Murder is not the only one.  Any mass will do.  The more of it you have, the better your position to bargain.  You could even trade some of your own mass in exchange for a boost!

\par \Sterope Okay, okay. So the mass could just as easily be the random crud out in No Man's Land?  

\par \Enrico I suppose so, yes.  Provided somebody could collect it all in to one place.

\par \Sterope So I could even take it from my own belly, or that of some other star?  Like, leave them alive, but take a little bit of their mass?

\par \Enrico I think that will work.  Good idea!

\par \Sterope Alright.  Now we're getting somewhere.  ...Wait, how do you suppose I collect mass?

\par \Enrico Off the top of my head, by finding those stars on the verge of evolving off the main-sequence and somehow getting your self close enough to them (i.e., in a binary system) that, when they evolve off the main-sequence and expand to become red giants, they transfer the mass in their expanded envelope over to you.

%\footnote{During the main-sequence (MS), stars are converting hydrogen into helium in their cores.  This is their primary source of energy, and makes them shine.  The MS is the first phase in the lifetime of a star, right after the protostellar phase.  It also tends to be the longest phase in the lifetime of a star, often lasting many hundreds or even thousands of times longer than later phases (e.g., the red giant branch phase).} 
%\footnote{Red giant branch stars (RGB) are more evolved than MS stars, converting hydrogen into helium outside the core in a shell.  During the red giant branch phase, stars can expand by up to a factor of several hundred times their former size on the MS.  The outer envelope is only tenuously bound, and can easily be stripped by a binary companion as the star expands.}
\begin{tcolorbox}[sharp corners, colback=red!30, colframe=red!80!blue, title=Stellar Evolution Beyond the Main-Sequence]
\par \textcolor{red} {During the main-sequence (MS), stars are converting hydrogen into helium in their cores.  This is their primary source of energy, and makes them shine.  The MS is the first phase in the lifetime of a star, right after the protostellar phase.  It also tends to be the longest phase in the lifetime of a star, often lasting many hundreds or even thousands of times longer than later phases (e.g., the red giant branch phase).
Red giant branch stars (RGB) are more evolved than MS stars, converting hydrogen into helium outside the core in a shell.  During the red giant branch phase, stars can expand by up to a factor of several hundred times their former size on the MS.  The outer envelope is only tenuously bound, and can easily be stripped by a binary companion as the star expands.
} 
\end{tcolorbox} 

\par \Sterope  Uh...Okay, so basically I am going to have to somehow figure out a way to swap myself in to \textit{and} out of at least one normal stellar binary system...in addition to two black hole binaries?  Then, I can use the mass of the black holes to accelerate me to above the escape speed from the cluster?  Just trying to wrap my head around this.  That seems like a lot of work.  Hmmmmm... Where do I even begin?

\par \Enrico Well, child, by my calculations, you need to participate in at least ten direct dynamical interactions with other singles or binaries in the cluster.  Two things can help with that:  increasing your mass, and reducing your velocity relative to the cluster average.  Both of these increase the rate of collisions with other stars or binaries in the cluster.  The reason is related to something called gravitational focusing.  During the encounter, gravity helps out a lot; it serves to focus inward the relative trajectories of two colliding particles, making it so that they are more likely to collide.  Hence, slower incoming singles are more likely to collide directly with a binary due to this gravitational focusing, whereas without gravity it would not occur.
%.\footnote{A simple estimate for the rate of direct collisions between identical single stars in a star cluster, borrowed from chemistry by considering a particle traveling through a uniform gaseous medium, comes from the mean free path (MFP) approximation.  Crudely, the MFP can be estimated by dimensional analysis, such that the mean free path l is l $\sim$ 1/n$\sigma$, where n is the mean particle number density and $\sigma$ is the collisional cross-section (i.e., an area corresponding to the direct overlap of two stars' radii; if a star passes within this area, a collision occurs).  If the mean particle velocity is v, then the rate of direct collisions is $\Gamma \sim$ n$\sigma$v and the mean time between collisions is $\tau \sim$ 1/$\Gamma$.}  
%\footnote{We are typically used to thinking of the collisional cross-section simply as the geometric surface area corresponding to the radii of two particles overlapping at closest approach.  That is, if the particle radius is R, n the geometrical cross-section for collision is $\pi$R$^2$.  The geometric cross-section can be enhanced when gravity is at work, and its effects in altering the velocities and trajectories of the interacting particles are non-negligible.  This new cross-section, called the gravitationally-focused cross-section for collision, can be calculated using conservation of energy and angular momentum. FINISH USING LEONARD OR SPITZER PAPERS.}

\par \Sterope Okay, got it...I think.  So...what do I do now?

\par \Enrico Not much you can do, but wait.  Don't worry though, gravity will do all the work for you.  It will carry you throughout this cluster, and deliver you close to other stars and binaries.  You are most likely to run in to the most massive objects in the cluster first; more massive objects exert the strongest gravitational force and, without even intending to, draw you in from further afar.  But massive objects are rare, and low-mass stars make up the vast majority of this old star cluster.  Usually the end result of these interactions is only a close approach, but often the encounter will be direct and you will become at least temporarily gravitationally bound to these other objects in the cluster...as you will find out for yourself soon enough!  

\par \nar \rmsterope~ noticed she had been drifting farther away from \rmenrico~.  She realized the process would continue, and they would soon part ways.

\par \Sterope I notice I am drifting away from you.  I can barely hear you anymore, in fact.  Thank you so much, \rmenrico, for all your help.  I'm off to find those two massive black holes!

\par \Enrico Good luck, young one.  It will take some time, but I have no doubt you will realize your goal of escaping eventually.

\begin{figure}
\includegraphics[width=\columnwidth,angle=270,origin=c]{ch4_4.pdf}
\caption{Armed with a satisfactory escape plan, \rmsterope~ drifts away from \rmenrico~, off in search of a couple black holes.  Illustration by Andre Pipe Oliva.
\label{fig:fig1}}
\end{figure}

\par \nar \rmsterope~ began her long journey through the cluster.  A sea of faces came in to and faded out of view.  One thing quickly became apparent to \rmsterope~ about her temporary neighbors:  they were highly skilled at avoiding eye contact.  So she continued on, mostly in silence, in search of the two massive black holes whose help she sought.

\end{document}


