\documentclass[main.tex]{subfiles}
\begin{document}

\chapter{Taming the Beast}

%\MAIA AND \MEROPE ARE IN A BINARY!!!  NEED TO INCLUDE \MAIA IN THE NEXT CHAPTER!!!

\newpara \nar \rmelectra~'s journey began when she left her birth cluster, heading out of the Galactic Disk.  But it wasn't long before a close encounter with a giant molecular cloud, named \rmaethra~, diverted her course.  The gravitational pull of the cloud caused a nearly 90$^{\circ}$ deflection.\footnote{INCLUDE DIAGRAM SHOWING THE EVOLUTION OF THE INTERACTION.}  

\newpara \Aethra I am so sorry, child.  It is often that I under-estimate the strength of my own gravity.  It would seem I have diverted your course through the Galaxy.

\newpara \Electra Oh, that's quite alright.  Honestly, it could have happened to anyone.

\newpara \Aethra It is true, but in this particular case, I have inadvertently diverted your trajectory directly toward the Galactic Center.

\newpara \nar Turning in the direction of her bulk motion, \rmelectra~ suddenly realized that, as \rmaethra~ had pointed out, her new course through the Galaxy pointed right toward the heart of the Galaxy, typically called the Galactic Center.\footnote{DESCRIBE THE CENTRAL NUCLEAR STAR CLUSTER IN THE MW.}

\newpara \Electra Uh oh...

\newpara \Aethra In the Galactic Center, there lives a dense nuclear star cluster, jam-packed with stars spanning all sorts of masses and ages.  We are in the field of our Galaxy currently, where the mean stellar density is about 1 star per cubic parsec.  Said another way, the distance separating the infamous Sun and her next nearest neighbor, Proxima Centauri, is about a parsec.  But if you take the Sun and Proxima Centauri as they are now, and drop them in to the very central regions of the nuclear cluster at the heart of our Galaxy, then more than 100 stars will fall between the pair.  The stellar densities are that much higher!

\newpara \Electra Wow!  That is dense!  Well, I guess I will find out when I get there.

\newpara \Aethra Rest assured, young one, anything could happen.  Your fate remains your own.  But, if you aren't careful, you stand to meet the behemoth that lurks at the Galactic Center.

\newpara \Electra Wait.  What's a behemoth?

\newpara \Aethra In this case, it is a super-massive black hole, weighing in at well over a million solar masses, that lives in the central nuclear cluster surrounding the very heart of our Galaxy.\footnote{GIVE SOME BACKGROUND ABOUT WHAT WE KNOW ABOUT SMBHS - THEIR TYPICAL MASSES, SIZES, HOW THEY FORM, OBSERVATIONAL PROOF OF THEIR EXISTENCE, ETC.}  

\newpara \Electra A super-massive black hole (SMBH)?  What's their story?

\newpara \Aethra They are dark, and do not emit light.  And they are very dense; typically smaller than Jupiter in diameter.  You will only ever be aware of them from their gravitational influence.  For example, if you drift within the central parsec\footnote{MAKE SURE THE TERM PARSEC IS DEFINED UPON ITS EARLIEST APPEARANCE.} or so of the central nuclear star cluster, you will start to feel the gravitational influence of the central SMBH non-negligibly.  That is, your trajectory will start to be dictated almost entirely by the gravity of the SMBH.  The total mass of SMBHs in the Universe can range anywhere from 10$^6$ to 10$^{10}$ times the mass of the Sun.  In our own Milky Way, the super-massive black hole is only about 4 $\times$ 10$^6$ times the mass of the Sun, so it's rather puny compared to some of the other SMBHs in the Universe!

\newpara \Electra Okay, that sounds like a lot of power.

\newpara \Aethra It is.  If stars drift too close to the central behemoth, they can be torn apart.  Their corpses are briefly crushed upon approaching very close to the SMBH, and they become extremely luminous for a brief time.  Some of their corpse is eaten by the behemoth, and some escapes as high-velocity ejecta; the tenuous final remains of a once proud star.  And, once inside an SMBH's belly, it is impossible to escape!  The very laws of the Universe forbid it! \footnote{DEFINE A TIDAL DISRUPTION EVENT AND PROVIDE SOME BACKGROUND.}

\newpara \Electra Oooookay.  Gotcha.  Stay the heck away from the SMBH lurking at the heart of the Galactic Center.  I will keep my eyes peeled for it.

\newpara \Aethra I am glad, child, and wish you the best of luck.  May the Cosmos bestow great luck upon you.

\newpara \Electra That sounds pretty great, actually.  I'll take two!
%NL: Does this joke work?

\section{Journey to the Heart of the Galaxy} \label{journey}

\newpara \nar \rmelectra~ continued her inevitable journey toward the Galactic Center, now too far from \rmaethra~ to hear her.  Along the way, \rmelectra~ met a number of colorful characters.  The first of these, \rmdardanus, took a particular interest in understanding her trajectory through the Galaxy.  She came upon him while traveling through the low-density Galactic field, No Man's Land, slowly drifting by at a distance, but close enough that they could hear each other bellow.

\newpara \Dardanus Hello there!  My name is \rmdardanus~.  Where are you off to in such a hurry?!

\newpara \Electra  Hello!  I am \rmelectra.  Well, if you take my current trajectory through the Galaxy and propagate it forward through time, it looks to me like I am ultimately heading for the Galactic Center.

\newpara \Dardanus  Great!  Wait... What is a ``Galactic Center''?

\newpara \Electra  As far as I know, it is the very center of our Galaxy.  Here, a dense nuclear star cluster lives, where the stellar densities are extremely high.  If I have my math right, it is about a million times denser in nuclear star clusters than it is here!
%Here, a dense nuclear star cluster lives, jam-packed with stars spanning all sorts of masses and ages.  We are in the field of our Galaxy currently, where the mean stellar density is about 1 star per cubic parsec.  Said another way, the distance separating the infamous Sun and her next nearest neighbor, Proxima Centauri, is about a parsec.  But if you take the Sun and Proxima Centauri as they are now, and drop them in to the very central regions of the nuclear cluster at the heart of our Galaxy, then more than 100 stars will fall between the pair.  The stellar densities are that much higher!

\newpara \Dardanus  Wow!  That sounds crowded.  I bet you can hear \textit{everything} \textit{everyone} is doing, at any given moment.  

\newpara \nar \rmdardanus~ shudders at the thought of living in an environment devoid of privacy.

\newpara \Electra But the really terrifying thing, if you ask me, is the super-massive black hole lurking at the heart of the Galaxy.

\newpara \Dardanus A super-massive black hole, you say?  What is that?

\newpara \Electra It is a dense dark object that does not emit light.  If I remember right, it's total mass can range anywhere from 10$^6$ to 10$^{10}$ times the mass of the Sun.  In our own Milky Way, the super-massive black hole is something like 10$^6$ times the mass of the Sun.  So it's massive, but not compared to the rest of its siblings.

\newpara \Dardanus Holy cow!  That's a lot of mass.  In fact, couldn't things potentially get violent if you were to wander too close to an SMBH?

\newpara \Electra Uh... Yeah, it's very possible.  Technically, if you wander too close to a super-massive black hole, it can eat you whole.  And, once inside its belly, you can never escape.

\newpara \Dardanus  Oh, wonderful!  That doesn't sound even remotely terrifying...

\newpara \nar \rmdardanus~ was no stranger to sarcasm.

\newpara \Electra Um... And even if you don't get eaten, if you do wander close to a super-massive black hole, you are almost certainly going to get accelerated to extremely high velocities.  Like, we are talking thousands of kilometers per second.

\newpara \Dardanus  WHAT?!  Is that even possible?

\newpara \Electra  Yep, such fast stars have definitely been seen whipping through our very own Galaxy, presumably on their way out.  I know, I've seen a couple.

\newpara \Dardanus Well, this is all very terrifying.  Given all these hurdles, how can you possibly expect to adequately prepare yourself for a journey through the central nuclear star cluster of the Milky Way?  What I'm trying to say, and delicately if I might add, is:  You're definitely going to die in there.

\newpara \nar \rmelectra~ yawns, tired from a long journey.

\newpara \Dardanus  Do you even hear me?  \textit{You are going to die for sure!}  Like, seriously, what are you going to do?

\newpara \Electra Well, the first thing I'm going to do is take another nap, frankly.  I figure I'll need to be fresh and alert for the end of this long journey.

\newpara \Dardanus Oh my gawd, you don't have a plan!  You are so going to die!  And why am I the only one freaking out about this?

\newpara \rmelectra~ had already fallen asleep, and was now snoring loudly.

\section{Befriending the Beast} \label{beast}

\newpara \nar \rmelectra~ awoke surrounded by stars.  This definitely was not the Galactic field anymore.\footnote{MAKE SURE THE GALACTIC FIELD IS DEFINED WHEN IT FIRST APPEARS.}  Recalling her trajectory before falling asleep, \rmelectra suddenly realized where she was.

\newpara \Electra The Galactic Center!  I made it!  Hello there!  Can you tell me:  Is this the nuclear star cluster at the heart of the Milky Way?

\newpara \nar A nearby star named \rmcarystus~ turned to face \rmelectra~, the confusion evident on his old face.  

\newpara \Carystus Well, to be honest, I've lived here my entire life, and I really have no idea what stars in the rest of the Galaxy call this place.  But, what I can tell you, is that you will not find a more extreme population density anywhere else in the Galaxy.  We pride ourselves on our crazy stellar densities, reaching as high as $\sim$ 10$^7$ pc$^{-3}$.

\newpara \Electra, Yep, this \textit{must} be the Galactic Center I have been looking for.  The behemoth must be close!

\newpara \Carystus Behemoth?

\newpara \Electra Allegedly, a super-massive black hole resides here somewhere, hiding among its many luminous denizens.

\newpara \Carystus Oh!  You must mean Chiron.  He's here alright.  Knowing when you are close is one thing, but finding him without being eaten is altogether another. 

\newpara \Electra Oh wow!  He sounds dangerous but illusive.  Any tips for how to best approach him?  I'd very much like to meet and interact with him, however briefly.

\newpara \Carystus  You are a brave one.  That is a good question.  Hmmmmm...   Well, \rmchiron~ is surrounded by a dense population of massive stars, directly orbiting his significant center of mass.\footnote{DESCRIBE WHAT IS KNOWN FROM OBSERVATIONS ABOUT THE INNER ~0.1 PC OF THE MW.}  These stars all emit winds from their surfaces, at a rate of many thousands of a solar mass per year.\footnote{Describe mass loss in massive stars and describe their basic evolution compared to low-mass stars.}  This is fast enough that, within almost no time, \rmchiron~ will collect their expelled gas in to a compact disk, which he accretes from and is thus his main source of fuel.  This disk, once formed, should emit light and be luminous; it is directly visible to us all on its own, even if \rmchiron~ is not.  But, without a doubt, \rmchiron~ will reside at the center of any such disk, called an ``accretion disk''.  This is where you can be sure to find him.

\newpara \Electra  Look for a small but bright accretion disk at the very center of this nuclear cluster.  This will clearly and unambiguously identify \rmchiron~.  Gotcha.  

\newpara \Carystus  Yep, that's about it.  If you can do that, you'll behold the beast.  

\newpara \Electra That's the goal!

\newpara \Carystus Just be careful.  Super-massive black holes have a mind of their own, and can be unpredictable.  Strange physics you might not be used to can emerge and even dominate the picture in this new high-gravity regime.  When these effects become important can be hard to identify before they are on top of you.  So one must take care to safely navigate such an extreme and dangerous environment.

\newpara \Electra I read you, loud and clear.  I definitely intend to keep my wits about me.  Well, I'm off to meet \rmchiron~.  Wish me luck!

\newpara \Carystus Good luck, my child.  May the brightest of futures befall you.

\end{document}
