\documentclass[main.tex]{subfiles}
\usepackage[T1]{fontenc}
\usepackage{ae,aecompl}


%%%%% AUTHORS - PLACE YOUR OWN PACKAGES HERE %%%%%

% Only include extra packages if you really need them. Common packages are:
\usepackage{graphicx}	% Including figure files
\usepackage{amsmath}	% Advanced maths commands
\usepackage{amssymb}	% Extra maths symbols

%\newcommand\levelone[1]{{\color{red}\bf}}
%\newcommand\leveltwo[1]{{\color{blue}\bf}}
%\newcommand\levelthree[1]{{\color{green}\bf}}
\begin{document}


%INSERT NEW CHAPTER 2 HERE.

\chapter{A Gust of Wind...}

\par \nar \rmsterope~ awoke when a gust of wind brushed past her face.  The gas was dense enough to temporarily obscure her vision.  She could feel the wind on her face as it rushed past, gently caressing her fiery skin.  All in all, the wind carried enough momentum to startle her out of slumber,

%\leveltwo{
\par \textcolor{blue} {\nar The momentum an object possesses is defined as the product of the object's mass and its velocity.  Momentum is in many ways complementary to the term inertia; momentum quantifies how difficult it is to alter an object's trajectory.  The more momentum an object possesses, the larger the total applied force must be (over a given interval of time) in order to change the object's trajectory.  More specifically, the total change in momentum can be calculated using the product of the applied force and the total time spent applying that force to the object.  The corresponding change in momentum is larger if either the magnitude of the applied force is larger, or the total time spent applying that force is longer.}.  

\par \nar \rmsterope~ coughed, clearing the gas from her face.  Her surroundings now revealed, \rmsterope~ looked around, confirming her suspicions; the gas had become substantially less dense, now disconcertingly sparse, since she had fallen asleep. 

%\levelone{
\par \textcolor{red} {\nar Density is defined as the total amount of mass per unit volume; lower densities imply less mass for a given unit of 3-D volume.}  
%Something was causing the remaining gas to leave, and fast.

\par \Sterope My sisters, you must wake up!  The remnants of our Mother are leaving us.

\par \nar The other six siblings awoke to the scene described by \rmsterope.  

\par \Electra Whoa!  What's going on?  We're all drifting apart.  And where did Mother go?

\par \Maia It's okay, young ones.  The last vestiges of Mother have now left us.  Upon giving birth to us, she activated our metabolisms.  We've been spewing light out ever since, in the form of photons.  Each photon carries momentum, and transfers some of it to any gas molecule or atom upon collision.  Light has been banging in to the gas and dust of our Mother for quite some time now, pushing her outward and away.

%\levelone{
\par \textcolor{red} {\nar ``Radiation pressure'', they call it.}

\par \Electra Okay, but then why are \textit{we} drifting apart from \textit{each other}?

\par \Maia Mother was made up of gas and dust, which came along with significant \textit{mass}.  And with mass comes gravity.  Now that her mass is gone, it can no longer contribute to the inward pull of gravity.

%\levelone{
\par \textcolor{red} {\nar Gravity provides an inward force that keeps particles effectively "glued" together.  The strength of this glue is often quantified by something called the gravitational potential energy of the system, which is always negative.}  

\par \Maia In other words, without our dispersed Mother, between the seven of us we no longer have enough mass in our mutually occupied volume to keep us gravitationally bound.  We are now free to drift apart.  And drift apart we are destined to do. 

\par \textcolor{blue} {\nar DERIVE CONDITION FOR A CLUSTER TO REMAIN BOUND.  EXTEND DERIVATION SO THE SOLUTION IS GIVEN AS A FUNCTION OF THE FRACTION OF ITS INITIAL MASS LOST. Quantify the timescale, and then relate it back to the lifetimes of the seven sisters to show that the cluster will disperse on a timescale much shorter than their lifetimes.}   

\par \Taygete Yeah, yeah, yeah.  But what does all that even \textit{mean}?

\par \Sterope I think it means that this is goodbye...  With our Mother's mass now lost, we are no longer gravitationally bound.  Relative to each other, we are energetically \textit{unbound}.

%\leveltwo{
\par \textcolor{blue} {\nar In addition to the gravitational potential energy, particles have relative motions and hence kinetic energies, a positive quantity.  The kinetic energy of a given particle is typically quantified by the square of its velocity relative to the system center of mass (multiplied by the particle mass...and, of course, divided by 2).  The total system energy is defined as the sum of the kinetic and gravitational potential energies, summed over all particles.  If the total system energy is negative, then the system is said to be gravitationally bound, and the particles remain effectively "glued" together.}  

\par \Sterope Fated to wander independently through the Cosmos.  Utterly and completely alone.  Well, except for \rmmaia~ and \rmmerope, I suppose, who form a binary.  Oh, and the triplets.  Those three are also still gravitationally bound.

\par \nar \rmalcyone's shoulders slump.  She begins to cry.

\par \Alcyone Already, I miss each and every one of you.

\par \Taygete Well, at least you have me.

\par \nar \rmalcyone rolls her eyes.

\par \Taygete Hey!  I saw that!

\par \Alcyone I'm sorry, sister.  You are right.  I am grateful for your presence.  Even if it \textit{is} all the time.  Without any breaks.  Ever.

\par \Maia I'm afraid \rmalcyone~ is right.  It is now time for each of us to follow our own paths through the Universe.  Or, equivalently, to follow our own trajectories through space-time.  At least in your case, \rmtaygete~, your sister \rmalcyone~ will be accompanying you.

\par \Electra Hold on a second!  I don't like the sound of this one bit!

\par \Sterope Me neither! \rmelectra~ and I are going to be completely alone!

\par \nar Both \rmsterope~ and \rmelectra~ lock eyes, exchanging a sympathetic glance as they continue to drift apart.  
%They drift farther and farther away from their sisters and each other.

\par \Maia Do not worry, my fresh new stars. This is all a part of the Circle of Life.  As are you.  Something tells me it will not be long before you hear from me again. Keep your eyes peeled to the horizon, and I will soon be there.  

\par \Electra Sigh...  

\par \Taygete So... Uh... Wow.  This is awkward.  I guess we'll see you guys later.  I'm not sure how or when that will happen.  I can only assume it will involve some miraculous and possibly mysterious act of fate.  But I'm sure it \textit{will} happen.
  
\par \nar \rmtaygete~ and \rmalcyone~ snicker quietly to themselves, exchanging a glance of mutual understanding.  Ever the pessimists.  \rmelectra~ begins to cry.  \rmsterope~ joins suit.  They cry together for a while, before \rmsterope~ stops and says to her sister, sniffling loudly:

\par \Sterope Do not worry, \rmelectra.  It is an exciting time!  A new chapter in our lives.  What adventures will befall us?  What obstacles will we overcome?  

\par \nar \rmelectra~ interrupts her sister:

\par \Electra How many times will I be overwhelmed by the situation, unsuccessfully trying to manage my anxiety by crying and blubbering uncontrollably? I can't wait to find out!

\par \nar Despite the brave face, \rmmaia~ was every bit as terrified as her sisters.  The oldest among them, \rmmaia intently sought to calm her panicking siblings as she slowly drifted from view.

\par \Taygete Well, \rmalcyone.  It looks like we're stuck with 'ol \rmcelaeno~ over there.

\par \Alcyone Yep, looks like it.  She was already gravitationally bound to us pretty significantly before the gas left, so I guess it's no surprise that she's still here.  Perhaps a disappointment, but not a surprise.

\par \Celaeno Heeeeelllloooo over there!  Did you know that I can hear you?  I wish I couldn't.  But I can.  

\par \Taygete We know. 

\par \nar \rmtaygete~ and \rmalcyone~ exchange a wink of understanding.  \rmcelaeno~ mutters under her breath:

\par \Celaeno I hate you.  Both.  Profoundly.

\par \Taygete What was that?

\par \nar \rmcelaeno~ speaks louder, so her sisters can hear:

\par \Celaeno I \textit{love} you both.  Profoundly.

\par \Alcyone Aw.

\begin{figure}
\includegraphics[width=\columnwidth]{fig3.png}
\caption{\rmmaia~ (blue), \rmelectra~ (red), \rmtaygete~ and \rmalcyone~ (both purple), \rmcelaeno~ (green) and \rmsterope~ (orange).  Note that, as explained in the text, green and purple stars do not exist in the Universe.  Illustrations by Joshua Leigh.
\label{fig:fig3}}
\end{figure}


\end{document}
